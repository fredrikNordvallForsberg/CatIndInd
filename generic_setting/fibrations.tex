\documentclass{article}

\usepackage{amssymb}
\usepackage{amsthm}
\usepackage{amsmath}
\usepackage{amstext}

\usepackage[all]{xypic}
\usepackage{stmaryrd}
\usepackage{enumerate}
\usepackage{mathtools}

\usepackage{mathabx} % \widecheck

\input header.inc

\title{Relationship to fibrational setting}


\author{Fredrik Nordvall Forsberg}

\begin{document}
\maketitle

\section{Fibrations}

We will follow the notation from Ghani et.\
al.\ ~\cite{ghaniJohannFumex2010fibind,ghaniJohannFumex2011indexedInd}.

\begin{definition}[Cartesian morphism]
  Let $U : \mathcal{E} \to \mathcal{B}$ be a functor. A morphism
  $\cart{f}{P} : f^* P \to P$ is \emph{cartesian} over a morphism $f :
  X \to Y$ in $\mathcal{B}$ if $U(\cart{f}{P}) = f$ and, for every $g
  : Q \to P$ in $\mathcal{E}$ with $U(g) = f \circ v$ for some $v :
  U(Q) \to X$, there exists a unique $h : Q \to f^* P$ in $\mathcal{E}$
  such that $U(h) = v$ and $\cart{f}{P} \circ h = g$.
\end{definition}

The notations $\cart{f}{P}$ and $f^8 P$ are justified since the cartesian morphism
over $f$ with codomain $U(P)$ is unique up to isomorphism.

\begin{definition}[Fibration]
  A functor $U : \mathcal{E} \to \mathcal{B}$ is a \emph{fibration} if
  for every object $P$ of $\mathcal{E}$ and every morphism $f : X \to
  U(P)$ there is a cartesian morphism $\cart{f}{P} : f^* P \to P$ in
  $\mathcal{E}$ over $f$.
\end{definition}

A fibration $U : \mathcal{E} \to \mathcal{B}$ is \emph{split} if there
is a choice of cartesian liftings which is strictly functorial, i.e.\
if $\cart{\id}{} = \id$ and $\cart{(g \circ f)}{} = \cart{g}{} \circ
\cart{f}{}$.
%TODO: Induced renaming functors instead of $\cart{\cdot}{}$?

\begin{definition}[Comprehension category]
  A comprehension category is a functor $U : \mathcal{E} \to
  \mathcal{B}^{\rightarrow}$ such that
  \begin{itemize}
  \item $\cod \circ U : \mathcal{E} \to \mathcal{B}$ is a fibration, and
  \item $U(g)$ is a pullback square in $\mathcal{B}$ for every
    cartesian morphism $g$ in $\mathcal{E}$ (with respect to the
    fibration $\cod \circ U : \mathcal{E} \to \mathcal{B}$).
  \end{itemize}
\end{definition}

A comprehension category $U$ is \emph{full} if $U : \mathcal{E} \to
\mathcal{B}^{\rightarrow}$ is full and faithfull, and \emph{split}
(\emph{cloven}) if $\cod \circ U : \mathcal{E} \to \mathcal{B}$ is
split (cloven).

Given a fibration $U : \mathcal{E} \to \mathcal{B}$, the terminal
object functor $1 : \mathcal{B} \to \mathcal{E}$ (if it exists)
assigns to each $X$ in $\mathcal{B}$ the terminal object in the fibre
$\mathcal{E}_X$. It is automatically right adjoint to $U$.

\begin{definition}[Comprehension category with unit]
  A \emph{comprehension category with unit} is a fibration $U :
  \mathcal{E} \to \mathcal{B}$ with a terminal object functor $1 :
  \mathcal{B} \to \mathcal{E}$, which itself has a right adjoint
  $\{\cdot\} : \mathcal{E} \to \mathcal{B}$:
\[
U \dashv 1 \dashv \{\cdot\}
\]
\end{definition}

\section{Fibrations from CwFs}

Exercise 10.4.6 in Jacobs' book \cite{jacobs1999catlogTT} constructs a
split full comprehension category from a category with families. Even
though Jacobs \cite[p.\ 618]{jacobs1999catlogTT} warns us that a
comprehension category with a fibred terminal object is not
automatically a comprehension category with unit, it is true in this
instance.

\begin{theorem} \label{thm:comprcat}
  Given a category with families $\C$, one can construct a comprehension category $U :
  \Ty_{\C} \to \C^{\rightarrow}$.
\end{theorem}
\begin{proof} \mbox{}
  \begin{itemize}
  \item The category $\Ty_{\C}$ has objects pairs $(\Gamma, \sigma)$
    with $\Gamma \in \C$ and $\sigma \in \Ty(\Gamma)$, and morphisms
    $(\Gamma, \sigma) \to (\Delta, \tau)$ are pairs $(f, M)$ with $f :
    \Gamma \to \Delta$ in $\C$ and ${M \in \Tm(\compr{\Gamma}{\sigma},
      \tau\{f \circ \ps\})}$.

  \item Now define the comprehension category $U : \Ty_{\C} \to
    \C^{\rightarrow}$ by
  %
  \begin{align*}
  U(\Gamma, \sigma) &= (\p{\sigma} : \compr{\Gamma}{\sigma} \to \Gamma) \enspace ,\\
  U(f, M) &= (\comprmor{f \circ \ps}{M}, f) \enspace .
  \end{align*}
  % 
  One should check that $U(f, M)$ really is a morphism $\p{\sigma} \to
  \p{\tau}$ in $\C^{\rightarrow}$, which is so since the following
  diagram commutes:
\[
\xymatrix{
\compr{\Gamma}{\sigma} \ar^-{\p{\sigma}}[r] \ar_-{\comprmor{f \circ \ps}{M}}[d] & \Gamma \ar^-{f}[d] \\
\compr{\Delta}{\tau} \ar_-{\p{\tau}}[r] & \Delta
}
\]

\item Next, we should check that $P \coloneqq \cod \circ U : \Ty_{\C} \to \C$
  is a fibration. Explicitly, we have $P(\Gamma, \sigma) = \Gamma$,
  $P(f, M) = f$. Given $(\Delta, \tau)$ in $\Ty_{\C}$ and $f : X \to
  \Delta$ in $\C$, we construct the cartesian morphism
  $\cart{f}{(\Delta, \tau)} \coloneqq (f, \vvs) : (X, \tau\{f\}) \to
  (\Delta, \tau)$ over $f$. (That is, $f^* (\Delta, \tau) \coloneqq
  (\Gamma, \tau\{f\})$.)  Suppose $(g, N) : (\Gamma', \sigma') \to
  (\Delta, \tau)$ with $g = P(g, N) = f \circ u$ for some $u : \Gamma'
  \to X$.  We define the mediating morphism $h : (\Gamma', \sigma')
  \to (X, \tau\{f\})$ by $h = (u, N)$ (this typechecks since $g = f
  \circ u$). Then $P(h) = u$ and
\begin{align*}
\cart{f}{(\Delta, \tau)} \circ h
  & = (f, \vvs) \circ (u, N) \\
  & = (f \circ u, \vvs\{\comprmor{u \circ \ps}{N}) \\
  & = (g, N)
\end{align*}
as required. Uniqueness follows from $P(h) = u$ and the equation above.
\item We now confirm that $U(\cart{f}{(\Delta, \tau)})$ is a pullback
  square in $\C$. Suppose that we have $r : X \to
  \compr{\Delta}{\tau}$ and $q : X \to \Gamma$ making the diagram commute:
\[
\xymatrix{
X \ar_-{r}@/_2.5pc/[ddr] \ar^-{q}@/^2pc/[drr] & & \\
& \compr{\Gamma}{\tau\{f\}} \ar^-{\p{\sigma}}[r] \ar_-{\comprmor{f \circ \ps}{\vvs}}[d] & \Gamma \ar^-{f}[d] \\
& \compr{\Delta}{\tau} \ar_-{\p{\tau}}[r] & \Delta
}
\]
%
The unique mediating morphism is $\comprmor{q}{\vvs\{r\}} : X \to \compr{\Gamma}{\sigma}$.
\item $U$ is full (i.e. $U : \Ty_{\C} \to \C^{\rightarrow}$ is full
  and faithfull): If $$(\comprmor{f \circ \ps}{M}, f) = U(f, M) = U(g,
  N) = (\comprmor{g \circ \ps}{N}, g),$$ then $f = g$ directly, and $M
  = \vvs\{\comprmor{f \circ \ps}{M}\} = \vvs\{\comprmor{g \circ
    \ps}{N}\} = N$ so that $(f, M) = (g, N)$. Hence $U$ is faithful.

  For fullness, assume
\[
\xymatrix{
\compr{\Gamma}{\sigma} \ar^-{\ps}[r] \ar_-{\alpha}[d] & \Gamma \ar^-{\beta}[d] \\
\compr{\Delta}{\tau} \ar^-{\ps}[r] & \Delta 
}
\]
Then $(\alpha, \beta) = (\comprmor{\ps \circ \alpha}{\vvs\{\alpha\}},
\beta) = (\comprmor{\beta \circ \ps}{\vvs\{\alpha\}}, \beta) =
U(\beta, \vvs\{\alpha\})$ so $U$ is full.

\item $U$ is split (i.e.\ $\cod \circ U : \Ty_{\C} \to \C$ is split):
  $\cart{\id}{} = (\id, \vvs) = \id_{\Ty_{\C}}$, and
  \begin{align*}
    \cart{(f \circ g)}{}
 = (f \circ g, \vvs) 
& = (f \circ g, \vvs\{\comprmor{g \circ \ps}{\vvs}\}) \\
& = (f, \vvs) \circ (g, \vvs) 
 = \cart{f}{} \circ \cart{g}{}.
  \end{align*}
\end{itemize}
\end{proof}

\begin{theorem}
  The underlying fibration in the comprehension category in Theorem
  \ref{thm:comprcat} is a comprehension category with unit if $\C$ has
  constant families.
\end{theorem}
\begin{proof}
  Define the terminal object functor $1 : \C \to \Ty_{\C}$ by
  $1(\Gamma) = (\Gamma, \constFam{\one})$, $1(f) = (f, \vvs)$. We
  should check the adjunctions $\cod \circ U \dashv 1 \dashv \dom
  \circ U$. Let us concentrate on $1 \dashv \dom \circ U$. Explicitly,
  $\{\cdot\} \coloneqq \dom \circ U$ sends an object $(\Gamma,
  \sigma)$ to $\compr{\Gamma}{\sigma}$ and a morphism $(f, M)$ to
  $\comprmor{f \circ \ps}{M}$. We want a natural isomorphism
  $\Hom((\Gamma, \constFam{\one}), (\Delta, \tau)) = \Hom(1(\Gamma),
  (\Delta, \tau)) \simeq \Hom(\Gamma, \{(\Delta, \tau)\}) =
  \Hom(\Gamma, \compr{\Delta}{\tau})$.  This is achieved by sending
  $(f, M) : (\Gamma, \constFam{\one}) \to (\Delta, \tau)$ to
  $\comprmor{f}{M\{\comprmor{\id}{\cFlift{!_{\Delta}}}\}}$ and $g :
  \Gamma \to \compr{\Delta}{\tau}$ to $(\ps \circ g, \vvs\{g \circ
  \ps\})$ in the other direction. The naturality is straightforward to
  check.
\end{proof}

\section{CwFs from fibrations}

It is not surprising that comprehension categories give rise to models
of type theory -- after all, that is why they were introduced by
Jacobs! Given a comprehension category with unit, we will construct a
Category with Attributes. It is then well-known (see e.g.\
Hofmann~\cite[Sect.\ 3.2]{hofmann1997syntaxsemantics}) how to make a
CwF from the CwA by defining the terms of type $\sigma$ to be the
sections of $\p{\sigma}$ (i.e.\ $f : \Gamma \to
\compr{\Gamma}{\sigma}$ such that $\p{\sigma} \circ f =
\id_{\Gamma}$).

\begin{theorem}
  Given a comprehension category with units $U : \mathcal{E} \to
  \mathcal{B}$, with adjunctions $U \dashv 1 \dashv \{ \cdot\}$, say,
  one can construct a CwA.
\end{theorem}
\begin{proof}
  Define $\Ty(\Gamma) \coloneqq \mathcal{E}_{\Gamma} $, the fibre over
  $\Gamma$. Substitution is taken care of by reindexing: $\sigma\{f\}
  \coloneqq f^* \sigma$. The functor laws are satisfied since $f^* :
  \mathcal{E}_\Delta \to \mathcal{E}_{\Gamma}$ is a functor. Further
  define $\compr{\Gamma}{\sigma} \coloneqq \{ \sigma \}$ with
  projection $U(\epsilon_{\sigma}) : \{\sigma\} \to \Gamma$, where
  $\epsilon : 1\{\cdot\} \to \id$ is the counit of the
  adjunction. This is typecorrect, since $U1 = \id$ and $U(\sigma) =
  \Gamma$.
  
  Finally, we have to check that
  \[
  \xymatrix{
    \{f^* \sigma\} \ar^-{\{\cart{f}{\sigma}\}}[r] \ar_-{U(\epsilon_{f^* \sigma})}[d] & \{ \sigma\} \ar^-{U(\epsilon_{\sigma})}[d] \\
    B \ar_-{f}[r] & \Gamma
  }
  \]
  is a pullback square. But this is exactly Jacobs verification that
  every CCU induces a comprehension category \cite[p.\
  616]{jacobs1999catlogTT}. We repeat the argument here. First, the
  square commutes since it is $U$ applied to the naturality square for
  $\epsilon$. Suppose $u : Q \to B$ and $v : Q \to \{ \sigma\}$ are
  given such that $f \circ u = U(\epsilon_{\sigma}) \circ v$. Via the
  Hom-set isomorphism, we have a transpose $\varphi(v) =
  \epsilon_{\sigma} \circ 1(v) : 1(Q) \to \sigma$ such that
  $U(\varphi(v)) = U(\epsilon_{\sigma}) \circ U(1(v)) =
  U(\epsilon_{\sigma}) \circ v = f \circ u$. Since $\cart{f}{\sigma}$
  is cartesian, this gives a morphism $h : 1(Q) \to f^* \sigma$ over
  $u$ such that $\cart{f}{\sigma} \circ h = \varphi(v)$. Now
  $\varphi^{-1}(h) = \{h\} \circ \eta_{Q} : Q \to \{f^* \sigma\}$ is
  the required mediating morphism, since we have
  \begin{align*}
    \{\cart{f}{\sigma}\} \circ \varphi^{-1}(h)
 &= \{\cart{f}{\sigma} \circ h \} \circ \eta_Q \\ 
 &= \{\varphi(v) \} \circ \eta_Q \\
 &= \varphi^{-1}(\varphi(v)) = v
  \end{align*}
  and
  \begin{align*}
    U(\epsilon_{f^* \sigma}) \circ \varphi^{-1}(h)
 &= U(\epsilon_{f^* \sigma}) \circ \{h\} \circ \eta_Q \\
 &= U(h) \circ U(\epsilon_{1(Q)}) \circ \eta_Q \qquad \text{(by naturality)}\\
 &= u \circ U(\epsilon_{1(Q)}) \circ \eta_Q\\
 &= u \circ U(\epsilon_{1(Q)}) \circ U(1(\eta_Q)) \quad \text{($U1 = \id$)}\\
 &= u \circ U(\epsilon_{1(Q)} \circ 1(\eta_Q))\\
 &= u \circ U(\id) = u \enspace .
  \end{align*}
  For uniqueness, assume there is $h' : Q \to \{f^* \sigma\}$ with
  $U(\epsilon_{f^* \sigma}) \circ h' = u$ and $\{\cart{f}{\sigma}\}
  \circ h' = v$. Then $\varphi(h') : 1(Q) \to f^* \sigma$, and
  $\varphi(h')$ is over $u$, since $U(\varphi(h')) = U(\epsilon_{f^*
    \sigma} \circ 1(h')) = U(\epsilon_{f^* \sigma}) \circ h' =
  u$. Furthermore, we have
  \begin{align*}
    \varphi(\{\cart{f}{\sigma}\} \circ h')
 &= \epsilon_{\sigma} \circ 1(\{\cart{f}{\sigma}\} \circ h') \\
 &= \epsilon_{\sigma} \circ 1(\{\cart{f}{\sigma}\}) \circ 1(h') \\
 &= \cart{f}{\sigma} \circ \epsilon_{f^* \sigma} \circ 1(h')
 = \cart{f}{\sigma} \circ \varphi(h')
  \end{align*}
  so that $\cart{f}{\sigma} \circ \varphi(h') =
  \varphi(\{\cart{f}{\sigma}\} \circ h') = \varphi(v)$. Thus, by the
  uniqueness of cartesian liftings, we must have $h = \varphi(h')$ so
  that $\varphi^{-1}(h) = h'$ as required.
\end{proof}



\bibliographystyle{alpha}
\bibliography{../../../references/biblio}

\end{document}
