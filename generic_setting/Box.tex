\documentclass{article}

\usepackage{amssymb}
\usepackage{amsthm}
\usepackage{amsmath}
\usepackage{amstext}

\usepackage[all]{xypic}
\usepackage{stmaryrd}
\usepackage{enumerate}
\usepackage{mathtools}

\usepackage{mathabx} % \widecheck

\input header.inc

\title{Random thoughts on $\BOX_F$}


\author{Fredrik Nordvall Forsberg}

\begin{document}
\maketitle

\section{Preliminaries}

Let $F : \C \to \D$ be a functor between two categories with families
$\C$ and $\D$.

\section{Universal property}

Given $\Gamma \in \C$ and $\sigma \in \Ty_{\C}(\Gamma)$, we should
have $\BOX_F(\Gamma, \sigma) \in \Ty_{\D}(F(\Gamma))$ such that
there exist an isomorphism
\[
\varphi_F : F(\Gamma . \sigma) \to F(\Gamma).\BOX_F(\Gamma, \sigma)
\]
with $\ps \circ \varphi_F = F(\ps)$.
%$\p{\BOX_F(\Gamma, \sigma)} \circ \varphi_F = F(\p{\sigma})$.

This $\BOX_F$ is unique up to isomorphism; recall that
$\Ty(\Gamma)$ can be made into a category with objects the elements
from $\Ty(\Gamma)$, and 
\[
\Hom_{\Ty(\Gamma)}(A, B) = \Tm(\Gamma . A, B\{\ps\}) \enspace .
\]
Composition is given by $g \circ f \coloneqq g\{\langle \ps,
f\rangle\}$ (with identity $\vvs$).

\begin{proposition}
\label{thm:box_uni}
  Suppose that $\BOX_F(\Gamma, \sigma)$ and $\BOX'_F(\Gamma,
  \sigma)$ with isomorphisms $\varphi_F$ and $\varphi'_F$ as above are
  given. Then they are isomorphic as objects in $\Ty(F(\Gamma))$.
\end{proposition}
\begin{proof}
Define
 \[f \in \Tm(F(\Gamma).\BOX'_F(\Gamma, \sigma),\BOX_F(\Gamma, \sigma)\{\ps\})\]
and
\[g \in \Tm(F(\Gamma).\BOX_F(\Gamma, \sigma),\BOX'_F(\Gamma,
\sigma)\{\ps\})\]
by $f = \vvs\{\varphi_F \circ \varphi_F^{'-1}\}$ and
$g = \vvs\{\varphi'_F \circ \varphi_F^{-1}\}$. Both terms have the
right type since $\ps \circ \varphi_F \circ \varphi_F^{'-1} = F(\ps)
\circ \varphi_F^{'-1} = \ps$ (and similarly for $g$). We get
\begin{align*}
  f \circ g &= f\{\langle \ps, g\rangle\} \\
  &= \vvs\{\varphi_F \circ \varphi_F^{'-1} \circ \langle \ps, \vvs\{\varphi'_F \circ \varphi_F^{-1}\}\rangle\} \\
  &= \vvs\{\varphi_F \circ \varphi_F^{'-1} \circ \langle \ps, \vvs \rangle \circ \varphi'_F \circ \varphi_F^{-1}\} \qquad \text{since $\ps \circ \varphi_F \circ \varphi_F^{'-1} = \ps$}\\
  &= \vvs\{\id\} \\
  &= \vvs = \id_{\Ty(F(\Gamma))}
\end{align*}
and similarly for $g \circ f$. Thus $\BOX_F(\Gamma, \sigma)$ and
$\BOX'_F(\Gamma, \sigma)$ are isomorphic.
\end{proof}

We can give an alternative proof by noting that there is another
characterisation of the morphisms of $\Ty_{\C}(\Gamma)$:

\begin{proposition}
  $\Ty_{\C}(\Gamma)$ is isomorphic to the category $\Ty'_{\C}(\Gamma)$
  with the same objects, but where the morphisms from $A$ to $B$ are morphisms $f : \Gamma . A \to \Gamma . B$ in $\C$ such that $\ps \circ f = \ps$.
\end{proposition}
\begin{proof}
  Define $F : \Ty_{\C}(\Gamma) \to \Ty'_{\C}(\Gamma)$ and $G :
  \Ty'_{\C}(\Gamma) \to \Ty_{\C}(\Gamma)$ to be the identities on
  objects and by $F(M) = \langle \ps, M\rangle$, $G(f) =
  \vvs\{f\}$. Note that $\ps \circ F(M) = \ps \circ \langle \ps , M
  \rangle = \ps$. We calculate:
\[    F(G(f)) = F(\vvs\{f\}) = \langle \ps, \vvs\{f\}\rangle = \langle \ps \circ f, \vvs\{f\}\rangle = \langle \ps , \vvs\rangle \circ f = f \]\[
    G(F(M)) = G(\langle \ps, M\rangle) = v\{\langle \ps , M\rangle\} = M
\]
\end{proof}

Thus, $\varphi_F \circ \varphi_F^{'-1}$ is obviously an isomorphism in
$\Ty'_{\C}(\Gamma)$ (note as in the proof of Prop.~\ref{thm:box_uni} that $\ps \circ \varphi_F \circ
\varphi_F^{'-1} = F(\ps) \circ \varphi_F^{'-1} = \ps$) and hence in
$\Ty_{\C}(\Gamma)$.

\section{Sufficient conditions for $\BOX_F$ to exist}

\subsection{``Constant family'' types}

$<$Motivation$>$.

\begin{definition}
  A CwF $\C$ supports constant family types if the following data are given:
  \begin{itemize}
  \item For each $\Gamma \in \C$, there is a type $\constFam{\Gamma}_{\Delta} \in
    \Ty(\Delta)$ for all $\Delta \in \C$ such that
    $
    \constFam{\Gamma}_{\Delta}\{g\} = \constFam{\Gamma}_{B}
    $
    whenever $g : B \to \Delta$. (We will usually omit the subscript $\Delta$.)
  \item An isomorphism $\cFhat{\cdot} : \Tm(B, \constFam{\Gamma}\{g\})
    \to \Hom(B, \Gamma)$ with inverse $\cFcheck{\cdot} : \Hom(B,
    \Gamma) \to \Tm(B, \constFam{\Gamma}\{g\})$ such that $\cFhat{M}
    \circ g = \cFhat{M\{g\}}$.
  \end{itemize}
\end{definition}

Clairambault and Dybjer \cite{clairambaultDybjer2011lcccML} defines a
similar notion of \emph{democracy} for a CwF; a CwF is democratic if
each context is represented by a type. In detail:

\begin{definition}[{\cite[Def.~6]{clairambaultDybjer2011lcccML}}]
  A CwF $\C$ is \emph{democratic} if for each object $\Gamma$ of $\C$
  there is $\overline{\Gamma} \in \Ty(\one_{\C})$ and an isomorphism
  $\gamma_{\Gamma} : \Gamma \to \one_{\C}.\overline{\Gamma}$.
\end{definition}

Reassuringly, A CwF supports constant families if and only if it is democratic.

\begin{proposition}
  A CwF $\C$ supports constant families if and only if it is democratic.
\end{proposition}
\begin{proof}
  ($\Rightarrow$) Assume $\C$ supports constant families. Define
  $\overline{\Gamma} \coloneqq \constFam{\Gamma}_{\one_{\C}}$ and
  $\gamma_{\Gamma} = \langle !_{\Gamma}, \cFcheck{\id}\rangle : \Gamma
  \to \one_{\C}.\overline{\Gamma}$ with inverse $\gamma_{\Gamma}^{-1}
  = \cFhat{\vvs} : \one_{\C}.\overline{\Gamma} \to \Gamma$. Of course,
  we have to check that $\gamma_{\Gamma}$ and $\gamma_{\Gamma}^{-1}$
  are really inverse to each other:
  \begin{align*}
    \gamma_{\Gamma}^{-1} \circ \gamma_{\Gamma}
 =  \cFhat{\vvs} \circ \langle !_{\Gamma}, \cFcheck{\id}\rangle 
 =  \cFhat{\vvs\{\langle !_{\Gamma}, \cFcheck{}\!\id\rangle\}}
 =  \cFhat{\cFcheck{\id}}
 =  \id
  \end{align*}
  \begin{align*}
    \gamma_{\Gamma} \circ \gamma_{\Gamma}^{-1}
 =  \langle !_{\Gamma}, \cFcheck{\id}\rangle  \circ \cFhat{\vvs} 
 =  \langle !_{\Gamma} \circ \cFhat{\vvs} , \cFcheck{\id}\{\cFhat{\vvs} \}\rangle 
 =  \langle !_{\one_{\C}.\constFam{\Gamma}} , \cFcheck{\id \circ \cFhat{\vvs}} \rangle 
 =  \langle \ps , \cFcheck{\cFhat{\vvs}} \rangle 
 =  \id
  \end{align*}
  Here, we use that $\p{\constFam{\Gamma}} :
  \one_{\C}.\constFam{\Gamma} \to \one_{\C}$ must be equal to
  $!_{\one_{\C}.\constFam{\Gamma}} : \one_{\C}.\constFam{\Gamma} \to
  \one_{\C}$ by the uniqueness of $!_{\one_{\C}.\constFam{\Gamma}}$,
  and that $m\{g\} = \cFcheck{\hat{}\!\!\!m \circ g}$ since
  $\cFhat{m\{g\}} = \hat{m} \circ g$.

  ($\Leftarrow$) Assume $\C$ is democratic. Define
  $\constFam{\Gamma}_{\Delta} \coloneqq
  \overline{\Gamma}\{!_{\Delta}\}$. Then
  \[\constFam{\Gamma}_{\Delta}\{g\} = \overline{\Gamma}\{!_{\Delta}
  \circ g\} = \overline{\Gamma}\{!_{B}\} = \constFam{\Gamma}_{B}\] by
  the uniqueness of $!_{B}$. Define $\cFhat{M} \coloneqq
  \gamma_{\Gamma}^{-1} \circ \langle !_{B}, M\rangle$ and $\cFcheck{f}
  \coloneqq \vvs\{\gamma_{\Gamma} \circ f\}$. Then
  \begin{align*}
    \cFcheck{\cFhat{M}}
  = \vvs\{\gamma_{\Gamma} \circ \gamma_{\Gamma}^{-1} \circ \langle !_{B}, M\rangle\}
  = \vvs\{\langle !_{B}, M\rangle\}
  = M
  \end{align*}
  and using that $\ps \circ \gamma_{\Gamma} \circ f : B \to \one_{\C}$
  is equal to $!_{B}$ for every $f : B \to \Gamma$, we have
  \begin{align*}
    \cFhat{\cFcheck{f}}
  &= \gamma_{\Gamma}^{-1} \circ \langle !_{B}, \vvs\{\gamma_{\Gamma} \circ f\}\rangle
   = \gamma_{\Gamma}^{-1} \circ \langle \ps \circ \gamma_{\Gamma} \circ f, \vvs\{\gamma_{\Gamma} \circ f\}\rangle \\
  &= \gamma_{\Gamma}^{-1} \circ \langle \ps, \vvs\rangle \circ \gamma_{\Gamma} \circ f 
   = \gamma_{\Gamma}^{-1} \circ \gamma_{\Gamma} \circ f 
   = f \enspace .
  \end{align*}
Finally, we have
\begin{align*}
    \cFhat{M} \circ g
  &= \gamma_{\Gamma}^{-1} \circ \langle !_{B}, M\rangle \circ g
   = \gamma_{\Gamma}^{-1} \circ \langle !_{B} \circ g, M\{g\}\rangle  
   = \gamma_{\Gamma}^{-1} \circ \langle !_{\Delta}, M\{g\}\rangle \\
  &= \cFhat{M\{g\}} \enspace .
\end{align*}

\end{proof}


\subsection{The inverse image type}

\begin{definition}
  A CwF $\C$ supports inverse image types if the following data are given:
  \begin{itemize}
  \item For each $f : A \to B$ in $\C$, there is a type $\invIm{f} \in
    \Ty(B)$ such that
    \[
    \invIm{f}\{g\} = <\text{substitution}>
    \]
    whenever $g : \Delta \to B$.
  \item A morphism $\im{f} : A \to B.\invIm{f}$ such that $\ps \circ
    \im{f} = f$ and $<$substitution$>$.
  \item For every type $\rho \in Ty(B.\invIm{f})$ and term $H \in
    \Tm(A, \rho\{\im{f}\})$ a term $\RinvIm{f}{\rho}(H) \in
    \Tm(B.\invIm{f}, \rho)$ such that $\RinvIm{f}{\rho}(H)\{\im{f}\} = H$ and
    $<$substitution$>$.
  \end{itemize}
\end{definition}

Inverse image types are not part of most presentations of type theory.
However, they can be constructed from constant families, sigma types
and extensional identity types:

\begin{proposition}
  Let $\C$ be a CwF with constant families, extensional identity types
  and $\Sigma$-types. Then $\C$ supports inverse image types.
\end{proposition}
\begin{proof}
  Based on the proof of \cite[Prop.~9 and
  Lemma~25]{clairambaultDybjer2011lcccML}. [Are intensional identity
  types enough?]
\end{proof}

\begin{proposition}
  Let $\C$ be a CwF with extensional identity types and inverse image
  types.  Then $\BOX_F(\Gamma, \sigma) \approx \invIm{F(\ps)}$.
\end{proposition}
\begin{proof}
  Based on \cite[Lemma 25]{clairambaultDybjer2011lcccML}.
\end{proof}

\begin{theorem}
  Let $\C$ be a category with finite limits. Then $\C$ can be extended
  to a CwF with constant families, extensional identity types and
  $\Sigma$-types.
\end{theorem}
\begin{proof}
  \cite[Lemma 18]{clairambaultDybjer2011lcccML}.
\end{proof}

\begin{corollary}
  Let $\D$ be a category with finite limits, $\C$ a CwF. Then $\BOX_F$ exists for any $F : \C \to \D$.
\end{corollary}

\section{$\BOX_F$ as a CwF morphism}
\label{sec:box-as-cwf-morphism}

\cite[Def.~10]{clairambaultDybjer2011lcccML}.
\cite[Proof of Thm.~10]{clairambaultDybjer2011lcccML}.

\bibliographystyle{alpha}
\bibliography{../../../references/biblio}

\end{document}
