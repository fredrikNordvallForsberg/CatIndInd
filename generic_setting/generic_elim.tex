\documentclass{article}

\usepackage{amssymb}
\usepackage{amsthm}
\usepackage{amsmath}
\usepackage{amstext}

\usepackage[all]{xypic}
\usepackage{stmaryrd}
\usepackage{enumerate}
\usepackage{mathtools}

\usepackage{mathabx} % \widecheck

\input header.inc

\title{Generic elimination rules for dialgebras}


\author{Fredrik Nordvall Forsberg}

\begin{document}
\maketitle

\section{Preliminaries}

Let $F$, $G : \C \to \D$ be a functor between two categories with
families $\C$ and $\D$ such that $\BOX_F$ (and $\BOX_G$) exists. Fix
an $(F,G)$-dialgebra $(X, \inn)$.

\section{Taking care of recursive calls with $\dmap_F$}

The usual elimination rules (rather, the computation rules associated
to them) refers to an operation $\dmap_F$ taking care of recursive
calls (called \textit{mapIH} by Dybjer and
Setzer~\cite{dybjersetzer1999finax}).

In the generic setting, $\dmap_F$ should have ``type''
\[
\frac{P \in \Ty(X) \qquad g \in \Tm(G(X), \BOX_G(P))}
     {\dmap_{F}(P, g) \in \Tm(F(X), \BOX_F(P))}
\]
and satisfy 
\[
\varphi_F \circ F(\cwfbar{f}) = \cwfbar{\dmap_F(P, \vvs\{\varphi_G \circ G(\cwfbar{f})\})} \enspace ,
\]
where $f \in \Tm(X, P)$.

%REMARK: This looks hairy, is there a better way?

\section{Generic elimination rules}
\label{sec:generic-elim}

The generic elimination rule for $(X, \inn)$ is
%
\begin{equation*}
\frac{P \in \Ty(X) \qquad \step \in \Tm(\compr{F(X)}{\BOX_F(P)}, \BOX_G(P)\{\inn \circ \ps\})}{\elim(P, \step) \in \Tm(G(X), \BOX_G(P))}
\label{eq:elim-general}
\end{equation*}
%
with computation rule:
%
\[
\elim(P, \step)\{\inn\} = \step\{ \cwfbar{\dmap_F(P, \elim(P, \step))} \}
\]

\begin{example}[The elimination rule in $\Set$ with $G = \id$]
  The category $\Set$ can be extended to a category with families by
  defining
  \begin{align*}
\Ty(\Gamma) & = \{ A\ \vert\ \text{$A : \Gamma \to \Set$ is a $\Gamma$-indexed family of (small) sets}\} \enspace ,\\
A\{f\} & = A \circ f \in \Ty(\Delta) \qquad\qquad\quad \text{($f : \Delta \to \Gamma$)} \enspace,\\ 
\Tm(\Gamma, A) & = (x : \Gamma) \to A(x) \enspace,\\
a\{f\} &= a \circ f \in \Tm(\Delta, A\{f\}) \ \,\qquad \text{($f : \Delta \to \Gamma$)} \enspace,\\
\compr{\Gamma}{A} & = (\Sigma x : \Gamma)A(x) \enspace,\\
\ps(x, y) & = x , \qquad \vvs(x, y)  = y \enspace,\\
\comprmor{f}{M}(x) & = (f(x), M(x)) \enspace .
  \end{align*}
%
  Let $F : \Set \to \Set$ be an endofunctor and $\id : \Set \to \Set$
  the identity functor. Then $\BOX_{\id}(P) = P$.  For $(F,
  \id$)-dialgebras (i.e.\ ordinary $F$-algebras) the elimination rule
  becomes (after currying $\step$)
\begin{align*}
\frac{P : X \to \Set \qquad
      \step : (x : F(X)) \to  \BOX_F(P, x) \to P(\inn(x))}
    {\elim(P, \step) : (x : X) \to P(x)}
\label{eq:elim-set}
\end{align*}
 as we are used to. The computation rule becomes (for $x : F(X)$)
\[
\elim(P, \step, \inn(x)) = \step(x, \dmap_F(P, \elim(P, \step))) \enspace .
\]
%
Notice how the condition on $\dmap_F$ boils down to $\varphi_F \circ
F(\cwfbar{f}) = \cwfbar{\dmap_F(P, f)}$ when $G = \id$.
\end{example}

%TODO: Example induction-induction?

\bibliographystyle{alpha}
\bibliography{../../../references/biblio}

\end{document}
