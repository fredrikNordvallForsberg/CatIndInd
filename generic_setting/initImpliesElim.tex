\documentclass{article}

\usepackage{amssymb}
\usepackage{amsthm}
\usepackage{amsmath}
\usepackage{amstext}

\usepackage[all]{xypic}
\usepackage{stmaryrd}
\usepackage{enumerate}
\usepackage{mathtools}

\usepackage{mathabx} % \widecheck

\input header.inc

\title{Init $\implies$ elim}


\author{Fredrik Nordvall Forsberg}

\begin{document}
\maketitle

\section{Init $\implies$ elim}

\begin{theorem}
  Let $F, G : \C \to \D$ with $\C$ and $\D$ CwFs. If $(X, \inn)$ is
  initial in $\Dialg(F, G)$ then the elimination principle holds for
  $(X, \inn)$.
\end{theorem}
\begin{proof}
  Let $P \in \Ty(X)$ and $g \in \Tm(\compr{F(X)}{\BOX_F(P)},
  \BOX_G(P)\{\inn \circ \ps\})$ be given. Then $h \coloneqq
  \varphi_G^{-1} \circ \comprmor{\inn \circ p}{g} \circ \varphi_F
  : F(\compr{X}{P}) \to G(\compr{X}{P})$, so by initiality, we get $\fold(h) : X \to
  \compr{X}{P}$ such that $h \circ F(\fold(h)) = G(\fold(h)) \circ \inn$. Hence
  the following diagram commutes:
\[
\xymatrix@-1pc{
F(X) \ar^-{\inn}[rrr] \ar_-{F(\fold(h))}[d] & & & G(X) \ar^-{G(\fold(h))}[d] \\
F(\compr{X}{P}) \ar_{\varphi_F}[dr] \ar_-{F(\ps)}[dd] & & & G(\compr{X}{P}) \ar@<-1ex>_-{\varphi_G}[dl] \ar^-{G(\ps)}[dd]\\
 & F(X).\BOX_F(P) \ar^{\comprmor{\inn \circ p}{g}}[r] \ar^{\ps}[dl]& \compr{G(X)}{\BOX_G(P)} \ar@<-1ex>_-{\varphi_G^{-1}}[ur] \ar^{\ps}[dr] & \\
F(X) \ar_-{\inn}[rrr] & & & G(X)
}
\]
This means that $\ps \circ \fold(h)$ is a morphism in $\Dialg(F, G)$,
so by initiality, we must have $\ps \circ \fold(h) = \id$. We now
define $\elim(P, g) \coloneqq \vv{}\{\varphi_G \circ G(\fold(h))\}$.
We then have
\begin{align*}
  \elim(p, G) &\in \Tm(G(X), \BOX_G(P)\{ \ps \circ \varphi_G \circ G(\fold(h))\}) \\
              &=  \Tm(G(X), \BOX_G(P)\{ G(\ps) \circ G(\fold(h))\}) \\
              &=  \Tm(G(X), \BOX_G(P)\{ G(\ps \circ \fold(h))\}) \\
              &=  \Tm(G(X), \BOX_G(P)\{ G(\id)\}) \\
              &=  \Tm(G(X), \BOX_G(P))
\end{align*}
as required.

We must check that the computation rule $\elim(P, g)\{\inn\} =
g\{\overline{\dmap_F(P, \elim(P, g))}\}$ holds. Note first that since
$\ps \circ \fold(h) = \id$, we have
\[
\fold(h) = \langle \ps \circ \fold(h), \vv{}\{\fold(h)\}\rangle
         = \langle \id , \vv{}\{\fold(h)\}\rangle
         = \overline{\vv{}\{\fold(h)\}}
\]
Using this, we have
\begin{align*}
  \elim(P, g)\{\inn\} &=  \vv{}\{\varphi_G \circ G(\fold(h)) \circ \inn\} \\
                      &=  \vv{}\{\varphi_G \circ \varphi_G^{-1} \circ \langle \inn \circ \ps, g\rangle \circ \varphi_F \circ F(\fold(h))\} \\
                      &=  g\{\varphi_F \circ F(\fold(h))\} \\
                      &=  g\{\varphi_F \circ F(\overline{\vv{}\{\fold(h)\}})\} \\
                      &=  g\{\overline{\dmap_F(P, \vv{}\{\varphi_G \circ G(\overline{\vv{}\{\fold(h)\}})\})}\} \\
                      &=  g\{\overline{\dmap_F(P, \vv{}\{\varphi_G \circ G(\fold(h))\})}\} \\
                      &=  g\{\overline{\dmap_F(P, \elim(P, g))}\}
\end{align*}
as required.
\end{proof}

\bibliographystyle{alpha}
\bibliography{../../../references/biblio}

\end{document}
