\documentclass{article}

\usepackage{amssymb}
\usepackage{amsthm}
\usepackage{amsmath}
\usepackage{amstext}

\usepackage{xypic}
\usepackage{stmaryrd}
\usepackage{enumerate}
\usepackage{mathtools}

\input header.inc

\begin{document}

\section*{Problem 1: $F (\Sigma\ A\ B) \cong \Sigma\ (F A)\ (\square_F B))$}

Here is a useful little construction: let $\widehat{\cdot} : ((x : A) \to B(a)) \to A \to \Sigma\ A\ B$ be defined by $\widehat{g}(x) = \langle x , g(x)\rangle$. Notice that $\pi_0 \circ \widehat{g} = \id$ and $\pi_1 \circ \widehat{g} = g$ for any $g$.

\begin{definition}
  Given functor $F : \Set \to \Set$, we define
\begin{align*}
\square_F &: \{A : \Set\} \to (P : A \to \Set) \to F(A) \to \Set \\
\dmap_F &: \{A : \Set\} \to \{P : A \to \Set\} \to ((x : A) \to  P(x)) \to (x : F(A)) \to \square_F(P, x)
\end{align*}
  by
\begin{align*}
  \square_F(B, x) &= \{ y : F\ (\Sigma\ A\ B )\ |\ F(\pi_0)(y) = x\} \\
  \dmap_F(g, x) &= \langle F(\widehat{g})(x), \refl\rangle
\end{align*}
Notice that we are using the fact that $\pi_0 \circ \widehat{g} = \id$, so that $F(\pi_0)(F(\widehat{g})(x)) = F(\pi_0 \circ \widehat{g})(x) = F(\id)(x) = x$, and $\refl$ has the right type.
\end{definition}

\begin{problem}
  \begin{enumerate}[(i)]
  \item There is $\varphi : F (\Sigma\ A\ B) \xrightarrow{\cong} \Sigma\ (F A)\ (\square_F B))$.
  \item $\pi_0 \circ \varphi = F(\pi_0)$.
  \item For $g : (x : A) \to B(x)$, we have $\widehat{\dmap_F(g)} = \varphi \circ F(\widehat{g})$.
  \end{enumerate}
\label{problem1}
\end{problem}
\begin{proof}
  \begin{enumerate}[(i)]
  \item Define
  \[
    \varphi :  F (\Sigma\ A\ B) \to \Sigma\ (F A)\ (\square_F B))
  \]
  and
  \[
    \psi :  \Sigma\ (F A)\ (\square_F B)) \to F (\Sigma\ A\ B)
  \]
  by
  \[
  \varphi(y) = \langle F(\pi_0)(y), \langle y, \refl\rangle\rangle
  \]
  \[
  \psi(\langle x, \langle y, p\rangle\rangle) = y.
  \]
  Then $\psi(\varphi(y)) = \psi(\langle F(\pi_0)\ y, \langle y,
  \refl\rangle\rangle) = y$ and for every $\langle x, \langle y,
  p\rangle\rangle : \Sigma\ (F A)\ (\square_F B))$, we have $x =
  F(\pi_0)(y)$ by $p$ and $p = \refl$ by proof irrelevance, so that
  $\varphi(\psi(\langle x, \langle y, p\rangle\rangle)) = \varphi(y) = \langle
  F(\pi_0)(y), \langle y, \refl\rangle\rangle = \langle x, \langle y,
  p\rangle\rangle $. Hence $F (\Sigma\ A\ B) \cong \Sigma\ (F A)\
  (\square_F B))$.
  \item $\pi_0(\varphi(y)) = \pi_0(\langle F(\pi_0)(y), \langle y, \refl\rangle\rangle) = F(\pi_0)(y)$.
  \item $\widehat{\dmap_F(g)}(x) = \langle x , \dmap_F(g, x)\rangle = \langle x , \langle F(\widehat{g})(x), \refl\rangle\rangle$, and also
 $\varphi(F(\widehat{g})(x))$ $=$ $\langle F(\pi_0)(F(\widehat{g})(x)), \langle (F(\widehat{g})(x), \refl\rangle\rangle$ $=$ $\langle F(\pi_0 \circ \widehat{g})(x)), \langle (F(\widehat{g})(x), \refl\rangle\rangle$ $=$ $\langle x, \langle (F(\widehat{g})(x), \refl\rangle\rangle$ since $\pi_0 \circ \widehat{g} = \id$. Hence $\widehat{\dmap_F(g)} = \varphi \circ F(\widehat{g})$.
  \end{enumerate}
\end{proof}

\section*{Problem 2: $\square$ for containers $\cong$ $\square$ for functors}

\begin{definition}
  For a container $S \triangleleft P$, we define
  \[
  \square_{S \triangleleft P} : \{A : \Set\} \to (B : A \to \Set) \to \sem{S \triangleleft P}\ A \to \Set
  \]
  by  $\square_{S \triangleleft P}(B, \langle s , f\rangle) = (p : P(s)) \to B(f(p))$.
\end{definition}

\begin{problem}
   For all $A : \Set$, $B : A \to \Set$ and $\langle s , f \rangle : \sem{S \triangleleft P}(A)$, 
\[
\square_{S \triangleleft P}(B, \langle s , f \rangle) \cong \square_{\sem{S \triangleleft P}}(B, \langle s , f \rangle).
\] \label{problem2}
\end{problem}
\begin{proof}
  Define
  \[
    \varphi :  ((p : P(s)) \to B(f(p))) \to \{ \langle s', f'\rangle : \Sigma s' : S\ . (P(s') \to \Sigma\ A\ B )\ |\ \langle s' , \pi_0 \circ f'\rangle = \langle s, f\rangle\}
  \]
  and
  \[
 \psi : \{ \langle s', f'\rangle : \Sigma s' : S\ . (P(s') \to \Sigma\ A\ B )\ |\ \langle s' , \pi_0 \circ f'\rangle = \langle s, f\rangle\} \to ((p : P(s)) \to B(f(p)))
  \]
  by
  \[
  \varphi(g) = \langle \langle s, (\lambda p\ .\ \langle f(p), g(p)\rangle)\rangle, \langle\refl, \ext(\refl)\rangle\rangle \]\[
  \psi(\langle\langle s', f'\rangle, r \rangle) = \pi_1 \circ f'
  \]
(there's lots of invented notation for equality proofs and hidden substitutions going on here). Now
$\psi(\varphi(g)) = \psi(\langle \langle s, (\lambda p\ .\ \langle f(p), g(p)\rangle)\rangle, \langle\refl, \ext(\refl)\rangle\rangle) = \pi_1 \circ (\lambda p\ .\ \langle f(p), g(p)\rangle) = g$ and 
\begin{align*}
\varphi(\psi(\langle\langle s', f'\rangle, r \rangle)) &= \varphi(\pi_1 \circ f') \\
&= \langle \langle s, (\lambda p\ .\ \langle f(p), \pi_1(f'(p))\rangle)\rangle, \langle\refl, \ext(\refl)\rangle\rangle \\
&=  \langle \langle s', (\lambda p\ .\ \langle \pi_0(f'(p)), \pi_1(f'(p))\rangle)\rangle, r\rangle \\
&=  \langle \langle s', f'\rangle, r\rangle
\end{align*}
where we have used eta for $\Sigma$ and $\Pi$ in the last equality.

\end{proof}


We can make $\square_F$ into a functor $\boxdot_F : \Fam(\Set) \to
\Fam(\Set)$ [in general $\Fam(\mathbb{C}) \to \Fam(\mathbb{D})$?] by
defining $\boxdot_F(A, B) = (F(A), \square_F(A, B))$ for objects and
$\boxdot(f, g) = (F(f), \lambda\ x\ \langle y, \refl\rangle\ . \langle F([f,g])(y), \refl\rangle)$ where $[f,g] :
\Sigma\ A\ B \to \Sigma\ A'\ B'$ is the obvious map, and the last $\refl$ is of the right type since $\pi_0 \circ [f, g] = f \circ \pi_0$.

Similarly, we can define $\boxdot_{S \triangleleft P}(A, B) = (\sem{S
  \triangleleft P}(A), \square_{S \triangleleft P})$ for objects and
and $\boxdot_{S \triangleleft P}(f, g) = (\sem{S \triangleleft P}(f),
\lambda (s, h)\ j\ p\ .\ g(h(p), j(p) ))$ for morphisms $(f, g) : (A,
B) \to (A', B')$. The result can now be strengthened to:

\begin{proposition}
  There is a natural isomorphism $\eta : \square_{S \triangleleft P} \Rightarrow \square_{\sem{S \triangleleft P}}$.
\end{proposition}
\begin{proof}
  We have already constructed the components at each $(A, B)$ and
  shown them to be isomorphisms in the proof of Problem
  \ref{problem2}. All that is left to do is to check the naturality
  condition, but this follows from a straightforward verification:
  \[
  \xymatrix
  {
    j \ar@{|->}[r] \ar@{|->}[d] & {\langle\langle s , (\lambda\ p\ .\ \langle h(p), j(p)\rangle)\rangle, \refl\rangle}
                                   \ar@{|->}[d] \\
    {\lambda\ p\ .\ g(h(p), j(p))} \ar@{|->}[r] & {\langle\langle s , (\lambda\ p\ .\ \langle f(h(p)), g(h(p), j(p))\rangle)\rangle, \refl\rangle}
  }
  \]
  
\end{proof}


\section*{Problem 3: Init $\implies$ elim (in $\Set$)}

Let $F : \Set \to \Set$ be a functor and $(\muF, \inn_F)$ a $F$-algebra, i.e.\ 
\[
\muF : \Set \qquad \qquad \inn_F : F (\muF) \to \muF.
\]

\subsection*{$\square_F$ and $\dmap_F$}

Let us first record what we need and expect from $\square_F$ and $\dmap_F$. They should have types
\begin{align*}
\square_F &: \{A : \Set\} \to (P : A \to \Set) \to F(A) \to \Set \\
\dmap_F &: \{A : \Set\} \to \{P : A \to \Set\} \to ((x : A) \to  P(x)) \to (x : F(A)) \to \square_F(P, x)
\end{align*}
and satisfy
\begin{equation}
  \label{eq:square}
  \tag{$\ast_{\square}$}
F (\Sigma\ A\ P) \cong \Sigma\ (F A)\ (\square_F P))
\end{equation}
and this isomorphism $\varphi$ must satisfy
\begin{equation}
  \label{eq:square2}
  \tag{$\ast\ast_{\square}$}
\xymatrix
{
F(\Sigma\ A\ P) \ar^-{\varphi}[r] \ar_{F(\pi_0)}[d] & \Sigma\ (F\ A)\ (\square_F\ P) \ar^{\pi_0}[dl] \\
F\ A  & 
}
\end{equation}
i.e.\ $\pi_0 \circ \varphi = F(\pi_0)$. For $\dmap_F$, we must for all $f : (x : A) \to B(x)$ have
\begin{equation}
  \label{eq:dmap}
  \tag{$\ast_{\dmap}$}
\xymatrix{
F\ A \ar^-{F(\widehat{f})}[r] \ar_{\widehat{\dmap_F(f)}}[d]& F(\Sigma\ A\ B) \ar^{\varphi}[dl] \\
\Sigma\ (F\ A)\ (\square_F\ B) & 
}
\end{equation}
i.e.\ $\widehat{\dmap_F(f)} = \varphi \circ F(\widehat{f})$.

For $\square_F(P, x) = \{ y : F\ (\Sigma\ A\ P )\ |\ F(\pi_0)(y) = x\}$ and $\dmap_F(g, x) = \langle F(\widehat{g})(x), \refl\rangle$, this holds, as proved in Problem \ref{problem1}.

\subsection*{The equivalence of elim and init}

\begin{principle}[Elim]
  The elimination principle for $F$ says that we have
\begin{multline*}
\elim_F : (P : \muF \to \Set) \to \big( (x : F(\muF)) \to \square_F(P, x) \to P(\inn_F(x))\big) \\ \to (x : \muF) \to P(x)
\end{multline*}
and computation rule
\[
\elim_F(P, g, \inn_F(x)) = g(x, \dmap_F(\elim_F(P, g), x))
\]
\end{principle}

\begin{principle}[Init]
  The initial algebra principle for $F$ says that $(\muF, \inn_F)$ is the \emph{initial} algebra for $F$, i.e.\ for any other $F$-algebra $(X, f)$, there is $\fold(f) : \muF \to X$ such that
\[
\xymatrix
{
F(\muF) \ar^{\inn_F}[r] \ar_{F(\fold(f))}[d] & \muF \ar^{\fold(f)}[d]\\
F(X) \ar^{f}[r] & X
}
\]
commutes.
\end{principle}

\begin{problem}
  Init $\implies$ elim.
\end{problem}
\begin{proof}
  Assume that $(\muF, \inn_F)$ is initial. We must construct
\begin{multline*}
\elim_F : (P : \muF \to \Set) \to \big( (x : F(\muF)) \to \square_F(P, x) \to P(\inn_F(x))\big) \\ \to (x : \muF) \to P(x)
\end{multline*}
such that
\[
\elim_F(P, g, \inn_F(x)) = g(x, \dmap_F(\elim_F(P, g), x)).
\]
Let $P : \muF \to \Set$ and $g : (x : F(\muF)) \to \square_F(P, x) \to P(\inn_F(x))$ be given. The plan is to make $\Sigma\ \muF\ P$ into a $F$-algebra $(\Sigma\ \muF\ P, h)$ and then show that $\pi_0 \circ \fold(h) = \id$, so that $\pi_1 \circ \fold(h) : (x : \muF) \to P(x)$. We can then define $\elim_F(P, g, x) = \pi_1(\fold(f)(x))$ and must show that the computation rule holds.

First, we need to define $h : F(\Sigma\ \muF\ P) \to \Sigma\ \muF\ P$. Let $\varphi : F (\Sigma\ \muF\ P) \xrightarrow{\cong} \Sigma\ (F\ A)\ (\square_F\ P)$ be the witness of (\ref{eq:square}). We can then define
\[
h(x) = \langle \inn_F(\pi_0(\varphi(x))), g(\pi_0(\varphi(x)),\pi_1(\varphi(x)))\rangle,
\]
so that $\fold(h) : \muF \to \Sigma\ \muF\ P$. We also know that
\begin{equation}
\label{eq:diagram1}
\xymatrix
{
F(\muF) \ar^{\inn_F}[r] \ar_{F(\fold(h))}[d] & \muF \ar^{\fold(h)}[d]\\
F(\Sigma\ \muF\ P) \ar^{h}[r] & \Sigma\ \muF\ P
}
\end{equation}
commutes. Now $\pi_0(h(x)) = \inn_F(\pi_0(\varphi(x))) \stackrel{(\text{\ref{eq:square2}})}{=} \inn_F(F(\pi_0)(x))$, so that the diagram
\begin{equation}
\label{eq:diagram2}
\xymatrix
{
F(\Sigma\ \muF\ P) \ar^{h}[r] \ar_{F(\pi_0)}[d] & \Sigma\ \muF\ P \ar^{\pi_0}[d] \\
F(\muF) \ar^{\inn_F}[r] & \muF 
}
\end{equation}
commutes. Pasting Diagram (\ref{eq:diagram1}) and (\ref{eq:diagram2}) together, we get the commuting diagram
\[
\xymatrix
{
%F(\Sigma\ \muF\ P) \ar^{h}[r] \ar_{F(\pi_0)}[d] & \Sigma\ \muF\ P \ar^{\pi_0}[d] \\
F(\muF) \ar^{\inn_F}[r] \ar_{F(\fold(h))}[d] & \muF \ar^{\fold(h)}[d]\\
F(\Sigma\ \muF\ P) \ar^{h}[r] \ar^{h}[r] \ar_{F(\pi_0)}[d] & \Sigma\ \muF\ P \ar^{\pi_0}[d] \\
F(\muF) \ar^{\inn_F}[r] & \muF 
}
\]
which shows that $\pi_0 \circ \fold(h) : (\muF, \inn_F) \to (\muF,
\inn_F)$ is a morphism in the category of $F$-algebras. But also $\id_{\muF}$ is such a morphism, and since $ (\muF, \inn_F)$ is initial, we must have $\pi_0 \circ \fold(h) = \id$. Hence we can define $\elim_F(P, g) \coloneqq \pi_1 \circ \fold(h) : (x : \muF) \to P(x)$.

It remains to be shown that $\elim_F(P, g, \inn_F(x)) = g(x,
\dmap_F(\elim_F(P, g), x))$. Unfolding the definition of $\elim_F$, we
see that we must prove $\pi_1(\fold(h)(\inn_F(x))) = g(x,
\dmap_F(\pi_1\circ\fold(h), x))$. Since $\fold(h) \circ \inn_F = h \circ F(\fold(h))$ by (\ref{eq:diagram1}), the left hand side reduces to $\pi_1(h(F(\fold(h)(x))))$ which is equal to $g(\pi_0(\varphi(F(\fold(h)(x)))),$ $\pi_1(\varphi(F(\fold(h)(x)))))$ by the definition of $h$. It is thus enough to show that
\begin{enumerate}[(i)]
\item $\pi_0 \circ \varphi \circ F(\fold(h)) = \id$ \label{comprule1}
\item $\pi_1 \circ \varphi \circ F(\fold(h)) = \dmap_F(\pi_1\circ\fold(h))$. \label{comprule2}
\end{enumerate}
Identity (\ref{comprule1}) is easily taken care of: by (\ref{eq:square2}), $\pi_0 \circ \varphi = F(\pi_0)$, so that we have $\pi_0(\varphi(F(\fold(h)(x)))) = F(\pi_0)(F(\fold(h)(x)))) = F(\pi_0 \circ \fold(h))(x) = F(\id)(x) = x$.

For (\ref{comprule2}), note that eta for $\Sigma$ implies $\fold(h)(x) = \langle \pi_0(\fold(h)(x)), \pi_1(\fold(h)(x))\rangle$ $=$ $\langle x, \pi_1(\fold(h)(x))\rangle$ so that $\fold(h) = \widehat{f}$ for $f \coloneqq \pi_1 \circ \fold(h)$. Hence, using $\pi_1 \circ \widehat{g} = g$ several times for different functions $g$, we have
\begin{align*}
  \dmap_F(\pi_1 \circ \fold(h)) &=  \dmap_F(\pi_1 \circ \widehat{f}) \\
  &= \dmap_F(f) \\
  &= \pi_1 \circ \widehat{\dmap_F(f)} \\
  &\!\!\!\!\!\!\stackrel{\text{(\ref{eq:dmap})}}{=} \pi_1 \circ \varphi \circ F(\widehat{f}) = \pi_1 \circ \varphi \circ F(\fold(h))
\end{align*}
which takes care of (\ref{comprule2}), and we are done.
\end{proof}

%It might be worth reflecting on what properties of $\Set$ we have used
%in the proof. We have explicitly defined $h : F(\Sigma\ \muF\ P) \to
%\Sigma\ \muF\ P$ and used that we know what $\pi_0 \circ h$ and $\pi_1
%\circ h$ are.


\section*{Problem 4: Elim $\implies$ init (in $\Set$)}

\begin{problem}
  Elim $\implies$ init.
\end{problem}
\begin{proof}
  Let $(X. f)$ be a $F$-algebra. We must construct $\fold(f) : \muF
  \to X$ such that $\fold(f) \circ \inn_F = f \circ F(\fold(f))$. We have 
  \begin{multline*}
    \elim_F : (P : \muF \to \Set) \to \big(g : (x : F(\muF)) \to \square_F(P, x) \to P(\inn_F(x))\big) \\ \to (x : \muF) \to P(x)
  \end{multline*}
  such that $\elim_F(P, g, \inn_F(x)) = g(x, \dmap_F(\elim_F(P, g), x))$,
  so let us choose $P = K\ X$ to be constantly $X$ and
$g(x, y) = f(F(\pi_1)(\varphi^{-1}(\langle x, y\rangle))))$, and define $\fold(f) = \elim_F(K\ X, g)$. Then $\fold(f) : \muF \to X$, and
\begin{align*}
  \fold(f)(\inn_F(x)) &= \elim_F(P, g, \inn_F(x)) \\
  &= g(x, \dmap_F(\elim_F(P, g), x)) \\
  &= f(F(\pi_1)(\varphi^{-1}(\langle x, \dmap_F(\elim_F(P, g), x)\rangle)))) \\
  &= f \circ F(\pi_1) \circ \varphi^{-1} \circ (\lambda x\ .\ \langle x, \dmap_F(\fold(f), x)\rangle)(x) \\
  &= f \circ F(\pi_1) \circ \varphi^{-1} \circ \widehat{\dmap_F(\fold(f))}(x) \\
  &\!\!\!\!\!\!\stackrel{\text{(\ref{eq:dmap})}}{=} f \circ F(\pi_1) \circ \varphi^{-1} \circ \varphi \circ F(\widehat{\fold(f)})(x) \\
  &= f \circ F(\pi_1 \circ \widehat{\fold(f)})(x) \\
  &= f \circ F(\fold(f))(x),
\end{align*}
so $\fold(f) \circ \inn_F = f \circ F(\fold(f))$ as required.
\end{proof}

\section*{Problem 5: $\sem{\elim_T}_{\Fam} \cong \elim$ for simple ind.-ind.}

If we consider ``simple'' induction-induction where the constructors
for $B$ do not refer to the constructors for $A$, the code for an
inductive-inductive set is given by two functors
\[
F : (A : \Set) \to (B : A \to \Set) \to \Set
\]
\[
G : (A : \Set) \to (B : A \to \Set) \to F(A, B) \to \Set,
\]
i.e. an endofunctor $FG$ on $\Fam(\Set)$. The formation and
introduction rules we expect now says that there is $(A, B) :
\Fam(\Set)$ and $(c, d) : FG(A, B) \to_{\Fam(\Set)} (A, B)$, i.e.
\[
A : \Set \qquad \qquad B : A \to \Set
\]
\[
c : F(A, B) \to A \qquad \qquad d : (x : F(A, B)) \to G(A, B, x) \to B(c(x)).
\]

Here are the types of the eliminators I would expect (whatever
$\square_F$, $\square_G$, $\dmap_F$, $\dmap_G$ are):
\begin{multline*}
\elim_F :  (P : A \to \Set) \to (Q : (x : A) \to B(x) \to P(x) \to \Set) \to \\
 (\overline{c} : (x : F(A, B)) \to \square_F(P, Q, x) \to P(c(x))) \to \\
 (\overline{d} : (x : F(A, B)) \to (y : G(A, B, x)) \to (\overline{x} : \square_F(P, Q, x)) \\ \to \square_G(P, Q, x, y, \overline{x}) \to Q(c(x), d(x, y), \overline{c}(x, \overline{x}))) \to \\
 (x : A) \to P(x)
\end{multline*}
\begin{multline*}
\elim_G :  (P : A \to \Set) \to (Q : (x : A) \to B(x) \to P(x) \to \Set) \to \\
 (\overline{c} : (x : F(A, B)) \to \square_F(P, Q, x) \to P(c(x))) \to \\
 (\overline{d} : (x : F(A, B)) \to (y : G(A, B, x)) \to (\overline{x} : \square_F(P, Q, x)) \\ \to \square_G(P, Q, x, y, \overline{x}) \to Q(c(x), d(x, y), \overline{c}(x, \overline{x}))) \to \\
 (x : A) \to (y : B(x)) \to Q(x, y, \elim_F(P, Q, \overline{c}, \overline{d}, x))
\end{multline*}
with computation rules
\[
\elim_F(P, Q, \overline{c}, \overline{d}, c(x)) = \overline{c}(x, \dmap_F(f, g, x))
\]
\[
\elim_G(P, Q, \overline{c}, \overline{d}, c(x), d(x, y)) = \overline{d}(x, y, \dmap_F(f, g, x), \dmap_G(f, g, x, y))
\]
where $f = \elim_F(P, Q, \overline{c}, \overline{d})$, $g = \elim_G(P, Q, \overline{c}, \overline{d})$.



\begin{problem}
  The interpretation of ordinary $\elim_T$ in $\Fam(\Set)$ is $\elim_F$ and $\elim_G$.
\end{problem}
\begin{proof}
Let us compile a small list of translations (of course, one should prove that this list is correct):

\begin{itemize}
\item $\sem{A : \Set}$ should be $A : \Set$, $B : A \to \Set$ -- the new basic objects are families.
\item $\sem{x : A}$ should be $x : A$ and $y : B(x)$.
\item $\sem{f : A \to A'}$ should be $f : A \to A'$, $g : (x : A) \to B(x) \to B'(f(x))$ -- a family morphism.
\item $\sem{P : A \to \Set}$ should be $P : A \to \Set$, $Q : (a : A) \to B(a) \to P(a) \to \Set$ since we want $\sem{P} : \sem{A} \sem{\to} \sem{\Set} = (A, B) \to_{\Fam(\Set)} \Fam(\Set)$.
\item $\sem{x : A \vdash p : P(x)}$ should be $x : A \vdash p : P(x)$ (or should we allow $x : A, y : B(x) \vdash p : P(x)$ here?) and $x : A, y : B(x) \vdash q : Q(x, y, p)$.
\item $\sem{f : (x : A) \to P(x)}$ should be $f : (x : A) \to P(x)$ and $g : (x : A) \to \mbox{(y : B(x))} \to Q(x, y, f(x))$.
\item $\sem{\Sigma\ A\ P}$ should be $\Sigma\ A\ P : \Set$, $(\lambda\ \langle a, p\rangle\ .\ \Sigma\ b\!:\!B(a).Q(a,b, p)))$ -- this is a a set and a family over it.
\item $\sem{\widehat{f}}$ should, given $f : (x : A) \to P(x)$ and $g : (x : A) \to (y : B(x)) \to Q(x, y, f(x))$ [i.e.\ $\sem{f : (x : A) \to P(x)}$]  be a function $\widehat{f} : A \to \Sigma\ A\ P$, $\widehat{f}(a) = \langle a, f(a)\rangle$, and a function $\widehat{g} : (a : A) \to B(a) \to  \Sigma\ b\!:\!B(a).Q(a,b, f(a))$, $\widehat{g}(a, b) = \langle b, g(a, b)\rangle$.
\end{itemize}
We do not need all of these yet, but they should be useful in the future.

Now, we can start to deconstruct $\elim_F$ and $\elim_G$. We already
noticed that $F$ and $G$ form an endofunctor $FG$ on $\Fam(\Set)$. We
also see that the arguments $P : A \to \Set$ and $Q : (a : A) \to B(a)
\to P(a) \to \Set$ are the interpretation of some $P : A \to \Set$,
i.e.\ $\sem{P} = (P, Q)$.

Moving on, we notice that $\square_F(P, Q) : F(A, B) \to \Set$ and
$\square_G(P, Q) : \\ (x : F(A, B)) \to G(A, B, x) \to \square_F(P, Q,
x) \to \Set$ is a family over $(F(A,B), G(A,B))$, i.e. the interpretation of some $\square_{FG} : FG(AB) \to \Set$.

With some currying, we see that $\overline{c} : (x : F(A, B)) \to
\square_F(P, Q, x) \to P(c(x))$ and $\overline{d} : (x : F(A, B)) \to
(y : G(A, B, x)) \to (\overline{x} : \square_F(P, Q, x)) \to
\square_G(P, Q, x, y, \overline{x}) \to Q(c(x), d(x, y),
\overline{c}(x, \overline{x}))$ just form a family morphism from $xy :
\Sigma\ (F(A, B), G(A, B))\ (\square_F(P, Q), \square_G(P, Q))$ to
$(P, Q)((c, d)(\sem{\pi_0}(xy)))$, i.e. they are the interpretation of some $g : (x : \Sigma\ FG(AB)\ (\square_{FG}\ P)) \to P(\inn_{FG}(\pi_0(x)))$.


Finally, we see that $(\elim_F(P, Q,
\overline{c}, \overline{d}), \elim_G(P, Q, \overline{c},
\overline{d})$ is a dependent morphism from $(xy : (A, B))$ to $(P,
Q)(xy)$.

All in all, $(\elim_F, \elim_G)$ is thus the translation of 
\begin{multline*}
\elim_{FG} : (P : A \to \Set) \to \big(g : (x : \Sigma\ (FG(A))\ \square_{FG}(P)) \to P(\inn_{FG}(\pi_0(x)))\big) \\ \to (x : A) \to P(x)
\end{multline*}
which is the ususal eliminator. The computation rules also seem to be
exactly the interpretatino of the usual computation rules.
\end{proof}


\end{document}

