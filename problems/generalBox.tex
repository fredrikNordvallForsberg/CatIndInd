\documentclass{article}

\usepackage{amssymb}
\usepackage{amsthm}
\usepackage{amsmath}
\usepackage{amstext}

\usepackage[all]{xypic}
\usepackage{stmaryrd}
\usepackage{enumerate}
\usepackage{mathtools}

\input header.inc

\title{Generalising $\square_F$ and $\dmap_F$ for non-endofunctors}


\author{Fredrik Nordvall Forsberg}

\begin{document}
\maketitle

\section{The standard situation}

In the ordinary situation, we have an endofunctor $F : \Set \to \Set$
representing an inductive definition, i.e.\ we have a set $\muF$ and a
constructor $c : F(\muF) \to \muF$. The eliminator has type
\begin{multline*}
\elim_F : (P : \muF \to \Set) \to (g : (x : F(\muF)) \to \square_F(\muF, P,x)
   \to P(c(x))) \\ \to (x : \muF) \to P(x)
\end{multline*}
and computation rule
\[
\elim_F(P, g, c(x)) = g(x, \dmap_F(\muF, P, \elim_F(P, g), x)).
\]
Here, 
\[
\square_F : (A : \Set) \to (P : A \to \Set) \to F(A) \to \Set
\]
gives the induction hypothesis (it is called `IH' in the
induction-recursion papers) and
\begin{multline*}
  \dmap_F : (A : \Set) \to (P : A \to \Set) \to ((x : A) \to P(x)) \\
  \to (x : F(A)) \to \square_F(A, P, x)
\end{multline*}
takes care of the recursive calls (in the IR papers, it is called
mapIH). Both $\square_F$ and $\dmap_F$ can be explicitly defined:
\begin{align*}
\square_F(A, P, x) &= \{ y : F(\Sigma\ A\ P)\ |\ F(\pi_0)(y) = x \} \\
\dmap_F(A, P, g, x) &= F(\widehat{g})(x)
\end{align*}
where $\widehat{g} : A \to \Sigma\ A \ P$ is defined by
$\widehat{g}(z) = \langle z , g(z)\rangle$. Note that $\pi_0 \circ
\widehat{g} = \id$, so that
\[
F(\pi_0)(F(\widehat{g})(x)) = F(\pi_0 \circ \widehat{g})(x))
                            = F(\id)(x)
                            = \id(x)
                            = x
\]
which shows that indeed $\dmap_F(A, P, g, x) : \square_F(A, P, x)$.
One can show that this definition of $\square_F$ is naturally
isomorphic to the definition in the IR papers (see Problem 2 in
\texttt{problems.pdf}).

However, we prefer to just use some abstract properties of $\square_F$
and $\dmap_F$, namely
\begin{equation}
  \label{eq:square}
  \tag{$\ast_{\square}$}
F (\Sigma\ A\ B) \cong \Sigma\ F(A)\ \square_F(A, B)
\end{equation}
and this isomorphism $\varphi$ satisfies
\begin{equation}
  \label{eq:square2}
  \tag{$\ast\ast_{\square}$}
\xymatrix
{
F(\Sigma\ A\ B) \ar^-{\varphi}[r] \ar_{F(\pi_0)}[d] & \Sigma\ F(A)\ \square_F(A, B) \ar^{\pi_0}[dl] \\
F(A)  & 
}
\end{equation}
i.e.\ $\pi_0 \circ \varphi = F(\pi_0)$. For $\dmap_F$, we have for all
$f : (x : A) \to B(x)$
\begin{equation}
  \label{eq:dmap}
  \tag{$\ast_{\dmap}$}
\xymatrix{
F(A) \ar^-{F(\widehat{f})}[r] \ar_{\widehat{\dmap_F(f)}}[d]& F(\Sigma\ A\ B) \ar^{\varphi}[dl] \\
\Sigma\ F(A)\ \square_F(A, B) & 
}
\end{equation}
i.e.\ $\widehat{\dmap_F(f)} = \varphi \circ F(\widehat{f})$.

\section{Generalising $\square_F$ and $\dmap_F$ for non-endofunctors}

Let us now try to generalise things, first by moving away from the
category $\Set$ to an arbitrary category $\C$ (with some
structure, of course), and then by replacing the endofunctor $F :
\C \to \C$ with just a functor $F : \C \to
\D$.

\subsection{The category $\semFam(\C)$ of interpretations
  of families in $\C$}

Let us for the moment assume that the category $\C$ is a model
of type theory. I want to define a category $\semFam(\C)$
of ``interpretations of families in $\C$''. It has as objects
an interpretation of `$A : \Set, B : A \to \Set$' in $\C$, and
a morphism from $\sem{A, B}$ to $\sem{A', B'}$ is an interpretation of
`$f : A \to A', g : (x : A) \to B(x) \to B'(f(x))$'.

Assuming that the interpretation of `$A : \Set$' in $\C$ is
$\sem{A} : \C$, there should also be a forgetful functor $U :
\semFam(\C) \to \C$ defined by $U(\sem{A},
\sem{B})$ = $\sem{A}$ and $U(\sem{f}, \sem{g})$ = $\sem{f}$.


\begin{example}[$\C = \Set$]
  For $\C = \Set$, we recover the usual category $\Fam(\Set)$
  of families of sets, i.e.\ we have objects pairs $(A, B)$ where $A :
  \Set$ and $B : A \to \Set$, and a morphism $(A, B) \to (A', B')$ is
  a pair $f : A \to A'$, $g : (x : A) \to B(x) \to B'(f(x))$.

  The forgetful functor $U$ is the usual forgetful functor $U :
  \Fam(\Set) \to \Set$; $U(A, B) = A$, $U(f, g) = f$.
\end{example}

\begin{example}[$\C = \Fam(\Set)$]
  Unless I am mistaken, the objects in $\semFam(\Fam(\Set))$ are
  tuples $((A_0, A_1), (B_0, B_1))$ where
  \begin{align*}
    A_0 &: \Set, \\
    A_1 &: A_0 \to \Set, \\
    B_0 &: A_0 \to \Set, \\
    B_1 &: (x : A_0) \to A_1(x) \to  B_0(x) \to \Set.
  \end{align*}
 A morphism from $((A_0, A_1), (B_0, B_1))$ to
  $((A'_0, A'_1), (B'_0, B'_1))$ consists of 
  \begin{align*}
    f_0 &: A_0 \to A'_0, \\
    f_1 &: (x : A_0) \to A_1(x) \to A'_1(f_0(x)), \\
    g_0 &: (x : A_0) \to B_0(x) \to B'_0(f_0(x)) \\
    g_1 &: (x : A_0) \to (y : A_1(x)) \to (z : B_0(x)) \to B_1(x, y, z) \\
    & \qquad \qquad\qquad \qquad\qquad \qquad\qquad \to B'_1(f_0(x), f_1(x, y), g_0(x, z)).
  \end{align*}
$U((A_0, A_1), (B_0, B_1)) = (A_0, A_1)$, $U((f_0, f_1), (g_0, g_1)) = (f_0, f_1)$.
\end{example}

\begin{question}
  Is there a better way to present $\semFam(\C)$?
\end{question}

\subsection{$\Sigma$ as a functor from $\semFam(\C)$ to $\C$}

Now, I notice that the sigma constructor takes a family $(A, B)$ of
sets and gives me a set $\Sigma\ A\ B$ back. It is also functorial,
since if I have a morphism $(f, g) : (A, B) \to (A', B')$, then I can
construct a morphism $[f, g] : \Sigma\ A\ B \to \Sigma\ A'\ B'$ by defining
\[
[f, g] \langle a, b\rangle = \langle f(a), g(a, b)\rangle.
\]
The projection $\pi_0$ takes me from $\Sigma\ A\ B$ to $A = U(A,B)$, and $\pi_0
\circ [f, g] = f \circ \pi_0$. It is thus a natural transformation
$\pi_0 : \Sigma \xrightarrow{\cdot} U$.


In the general case, I would thus like to generalise $\Sigma$ to a functor
\[
\Sigma_{\C} : \semFam(\C) \to \C
\]
together with a natural transformation $\pi_0 : \Sigma_{\C} \xrightarrow{\cdot} U$.

\begin{question}
  Do we need $\pi_1 : (x : \Sigma_{\C}(A, B)) \to B(\pi_0(x))$
  as well?
\end{question}

\begin{question}
  I probably would like $\Sigma_{\C}$ to be the interpretation
  of $\Sigma$ in $\C$. For morphisms, this would mean that I
  want $\Sigma_{\C}(f,g)$ to be the interpretation of $[f,g]$?
\end{question}

\subsection{$\square$ is a functor $\square' : \Set^{\Set} \to \Fam(\Set)^{\Fam(\Set)}$}

Looking at the type 
\[
\square_F : (A : \Set) \to (P : A \to \Set) \to F(A) \to \Set
\]
of $\square_F$ again, we see that we can write this as
\[
\square' : \Set^{\Set} \to \Fam(\Set)^{\Fam(\Set)}
\]
if we define $\square'(F) = \lambda (A, B) .\ (F(A), \square_F(A,
B))$. For $\square'(F)$ to be a functor $\Fam(\Set)$ to $\Fam(\Set)$,
it must also act on morphisms. Suppose $(f, g) : (A, B) \to (A',
B')$. We must find a morphism from $\square'(F)(A, B)$ to
$\square'(F)(A', B')$, i.e.\ from
\[
(F(A),  \{ y : F(\Sigma\ A\ B)\ |\ F(\pi_0)(y) = x \})
\]
to
\[
(F(A'),  \{ y : F(\Sigma\ A'\ B')\ |\ F(\pi_0)(y) = x \}).
\]
But $f : A \to A'$, so $m = F(f) : F(A) \to F(A')$ can be our first
component. Now we need to define
\begin{multline*}
  n : (x : F(A)) \to \{ y : F(\Sigma\ A\ B)\ |\ F(\pi_0)(y) = x \} \\
  \to \{ z : F(\Sigma\ A'\ B')\ |\ F(\pi_0)(z) = F(f)(x) \}
\end{multline*}
Let $n(x) = F([f, g])$. We need to prove that $F(\pi_0)(F([f, g])(y))
= F(f)(x)$ for $y : F(\Sigma\ A\ B)$ such that $F(\pi_0)(y) = x$. But
$\pi_0 \circ [f, g] = f \circ \pi_0$, so
\[
F(\pi_0)(F([f, g])(y)) = F(\pi_0 \circ [f, g])(y) = F(f \circ
\pi_0)(y) = F(f)(F(\pi_0)(y)) = F(f)(x).
\]
Thus we can take $\square'(F)(f,g) = (n, m) = (F(f), \lambda x .\
F([f,g]))$. This shows that $\square'(F) : \Fam(\Set)^{\Fam(\Set)}$ if
$F : \Set \to \Set$ is a functor.

However, we want more! We want $\square'$ to be a functor $\square' :
\Set^{\Set} \to \Fam(\Set)^{\Fam(\Set)}$. We have defined the object
part of $\square'$, what is the morphism part? Given a natural
transformation $\eta : F \xrightarrow{\cdot} G$, we have to construct
a natural transformation $\square'(\eta) : \square'(F)
\xrightarrow{\cdot} \square'(G)$. The component of $\square'(\eta)$ at
$(A, B)$ consists of a function $f : F(A) \to G(A)$ and a function $g
: (x : F(A)) \to \square_F(A, B, x) \to \square_G(A, B, f(x))$. For
$f$, we can choose $f = \eta_A$. Choose $g(x) = \eta_{\Sigma\ A\
  B}$. Certainly $\eta_{\Sigma A B} : F(\Sigma\ A\ B) \to G(\Sigma\ A\
B)$, but do we have $G(\pi_0)(\eta_{\Sigma A B}(y)) = \eta_a(x)$,
given that $F(\pi_0)(y) = x$? Yes, because substituting $x =
F(\pi_0)(y)$, we end up with the equation $G(\pi_0)(\eta_{\Sigma A
  B}(y)) = \eta_a(F(\pi_0)(y))$, which is exactly the naturality
condition!
\[
\xymatrix
{
F(\Sigma\ A\ B) \ar^-{\eta_{\Sigma A B}}[r] \ar_{F(\pi_0)}[d] & G(\Sigma\ A\ B) \ar^{G(\pi_0)}[d] \\
F(A) \ar_-{\eta_{A}}[r] & G(A) 
}
\]
Hence we can define $\square'(\eta)_{(A, B)} = (\eta_A,
\lambda x .\ \eta_{\Sigma A B})$. The naturality of $\square'(\eta)$
follows directly from the naturality of $\eta$.

Now that $\Sigma$ and $\square$ can be seen as functors, we can
replace the isomorphism $\varphi$ in (\ref{eq:square}) with a natural
isomorphism $\eta : F \circ \Sigma \xrightarrow{\cdot} \Sigma\ \circ
\square$. One can easily check that the isomorphism given in Problem 1
in fact is natural:
\[
\xymatrix
{
F(\Sigma\ A\ B) \ar^-{\eta_{(A, B)}}[r] \ar_{F([f, g])}[d] & \Sigma\ F(A)\ \square_F(A,B) \ar^{[F(f), [F([f,g])]]}[d] \\
F(\Sigma\ A'\ B')  \ar_-{\eta_{(A',B')}}[r] & \Sigma\ F(A')\ \square_F(A',B')
}
\]

\subsection{A generalised $\square_F$ for $F$ not an endofunctor}

Assume that $\C$ and $\D$ are categories such that
$\semFam({\C})$ and $\semFam({\D})$ make
sense. (Then $\Sigma_{\C}$ and $\Sigma_{\D}$ should
make sense as well.) We can then generalise $\square_F$ to a functor
\[
\square : \D^{\C} \to \semFam(\C) \to \semFam(\C)
\]
such that there is a natural isomorphism
\begin{equation}
  \label{eq:square}
  \tag{$\ast_{\square}$}
\eta : F \circ \Sigma_{\C} \xrightarrow{\cdot} \Sigma_{\D}\ \circ \square_F
\end{equation}
satisfying
\begin{equation}
  \label{eq:square2}
  \tag{$\ast\ast_{\square}$}
\xymatrix
{
F(\Sigma_{\C}\ A\ B) \ar^-{\eta_{(A, B)}}[r] \ar_{F(\pi_0)}[d] & \Sigma_{\D}\ F(A)\ \square_F(A, B) \ar^{\pi_0}[dl] \\
F(A)  & 
}
\end{equation}
i.e.\ $\pi_0 \circ \eta_{(A, B)} = F(\pi_0)$.

\subsection{The Inverse image}
\label{inverseImage}

If $\C$ also has an inverse image operation which maps $f : X
\to Y$ to $\invIm{f} : \semFam(\C)$ with $U(\invIm{f}) = Y$, we can
define $\square_F(X) = (F(U(X)), \invIm{(F(\pi_{0_X}))})$.

We have to check that (\ref{eq:square}) and (\ref{eq:square2})
hold. \ldots more conditions on $\invIm{\cdot}$ needed (corresponding to being
left adjoint to reindexing) \ldots
% For (\ref{eq:square}), define $\eta : F \circ
%\Sigma_{\C} \xrightarrow{\cdot} \Sigma_{\D}\ \circ
%\square_F$ by
%\[
%\eta_{X} = 
%\]

\section{$\Sigma$ in the relevant categories}
Let us write $\Pow_{\C}(X)$ for the interpretation of $X \to \Set$ in
$\C$.

\begin{definition}
  A category $\C$ \emph{has sigma types} if $\semFam(\C)$ and a
  forgetful functor $U : \semFam(\C) \to \C$ exists, together with a
  functor $\Sigma_{\C} : \semFam(\C) \to \C$ and a natural
  transformation $\pi_0 : \Sigma_{\C} \to U$.
\end{definition}

\begin{definition}
  Let $\C$, $\D$ be categories having sigma types.  We say that
  \emph{$\C$ and $\D$ have boxes} if there is a functor $\square :
  \D^{\C} \to \semFam(\D)^\semFam(C)$ satisfying (\ref{eq:square}) and
  (\ref{eq:square2}).
\end{definition}

Section \ref{inverseImage} starts exploring the idea that if $\C$ and
$\D$ have sigma types, and $\D$ in addition an inverse image
operation, then $\C$ and $\D$ have boxes.

\subsection{$\Sigma$ and $\square$ in $\Set$}

This has been covered already. 

\subsection{$\Sigma$ and $\square$ in $\Fam(\Set)$}
\label{boxFamSet}

\begin{multline*}
\Sigma_{\Fam(\Set)} : (A : \Set)(B : A \to \Set) \to \\
     (P : A \to \Set)(Q: (x : A) \to B(a) \to P(a) \to \Set) \to \Fam(\Set)
\end{multline*}
\[
\Sigma_{\Fam(\Set)}(A, B, P, Q) = (\Sigma\ A\ P, \lambda \langle a, p\rangle . \Sigma b\!:\!B(a).\ Q(a, b, p))
\]

\[
(\pi_0, \overline{\pi_0})_{(A, B), (P, Q)} : \Sigma_{\Fam(\Set)}(A, B, P, Q) \to (A, B)
\]
\[
(\pi_0, \overline{\pi_0})_{(A, B), (P, Q)}(\langle a, p\rangle, \langle b, q\rangle) = (a, b)
\]


Let $F : \Fam(\Set) \to \Set$. Then
\[
\square_F : \semFam(\Fam(\Set)) \to \Fam(\Set)
\]
\begin{align*}
\square_F((A, B), (P, Q), x)
 &= \{ y : F(\Sigma_{\Fam(\Set)}\ (A, B)\ (P, Q)) | F(\pi_0, \overline{\pi_0})(y) = x\} \\
 &= \{ y : F(\Sigma\ A\ P, \lambda \langle a, p\rangle . \Sigma b\!:\!B(a).\ Q(a, b, p)) | F(\pi_0, \overline{\pi_0})(y) = x\}
\end{align*}

\subsection{Inverse image in $\Fam(\Set)$}

Given $(f, g) : (A, B) \to (A', B')$, we can define $(f, g)^{-1} :
(A', B') \to \Fam(\Set)$ by
\[
(f, g)^{-1} = (f^{-1}, g^{-1})
\]
where $f^{-1} : A' \to \Set$, $f^{-1}(x) = \{ y : A\ |\ f(y) = x\}$, and 
\[
g^{-1} : (x : A') \to B'(x) \to f^{-1}(x) \to \Set,
\] 
$g^{-1}(x, b, y) = \{ z : B(y)\ |\ g(y, z) = b \}$. The equality $g(y,
z) = b$ typechecks since $f(y) = x$, hence $g(y, z) : B'(f(y)) = B'(x)$.

\subsection{$\Sigma$ and $\square$ in $\Bialg(F, G)$ for $F, G : \C \to \D$}

Assume that $\C$ and $\D$ have boxes. Then $\Bialg(F, G)$ have sigma
types, and $\semFam(\Bialg(F, G)) = \Bialg(\square_F, \square_G)$.

\begin{remark}
  If one notices that $\square$ is the morphism part of the (weak)
  functor $\semFam :$ Cat$_{\text{models of TT}} \to $ Cat, this
  becomes
  \[
  \semFam(\Bialg(F, G)) = \Bialg(\semFam(F), \semFam(G)),
  \]
  which looks rather natural.
\end{remark}



\subsubsection{$\Sigma$ in $\Bialg(F, G)$}

Let $\varphi_F : F (\Sigma(A, P)) \to \Sigma(\square_F(A, P))$ and
similarly $\varphi_G$ be the natural isomorphisms for the boxes in
$\C$, $\D$. Given $(A, P) : \semFam(\C)$ and $(c, d) : \square_F(A, P)
\to \square_G(A, P)$, i.e. an object in $\semFam(\Bialg(F, G))$, we
construct
\[
\Sigma_{\Bialg(F, G)}\ (A, c)\ (P, d) = (\Sigma_{\C}\ A\ P, \varphi^{-1}_G \circ [c, d] \circ \varphi_F)
\]
where $[c, d] = \Sigma_\C(c, d) : \Sigma_\C \square_F(A, P) \to
\Sigma_\C \square_G(A, P)$ is the morphism part of $\Sigma_\C$ applied
to $(c, d)$. For morphisms, given $(f, g) : ((A, P), (c, d))
\to_{\semFam(\Bialg(F, G))} ((A', P'), (c', d'))$, i.e.\ $(f, g) : (A,
P) \to_{\semFam(\C)} (A', P')$ such that
\begin{equation}
\label{eq:331}
\xymatrix{
\square_F(A, P) \ar^-{(c, d)}[r] \ar_-{\square_F(f, g)}[d] & \square_G(A, P) \ar^-{\square_G(f, g)}[d] \\
\square_F(A', P') \ar^-{(c', d')}[r] & \square_G(A', P') \\
}
\end{equation}
commutes, we define $\Sigma_{\Bialg(F, G)}(f, g) = \Sigma_{\C}(f, g)$.
We have to check that this really is a morphism in
$\Bialg(F, G)$, i.e.\ that the following commutes:

\[
\xymatrix{
F(\Sigma_{\C}\ A\ P) \ar^-{\varphi^{-1}_G \circ [c, d] \circ \varphi_F}[rr] \ar_-{F[f, g]}[d] & & G(\Sigma_{\C}\ A\ P) \ar^-{G[f,g]}[d] \\
F(\Sigma_{\C}\ A'\ P') \ar^-{\varphi^{-1}_G \circ [c', d'] \circ \varphi_F}[rr] & & G(\Sigma_{\C}\ A'\ P')
}
\]
By the naturality of $\varphi$, we have $\Sigma\square_F(f, g) \circ
\varphi_F = \varphi_F \circ F\Sigma(f, g)$ or equivalently $F\Sigma(f,
g) = \varphi^{-1} \circ \Sigma\square_F(f, g) \circ \varphi_F$. This
shows that the left and rightmost squares in the following diagram
commutes. The outermost square commutes by (\ref{eq:331}), and the top
and bottom squares by definition, so that the square in the middle
commutes by diagram chasing, and we are done:
\[
\xymatrix{
\Sigma_{\C}\square_F(A,P) \ar^-{[c, d]}[rr]  \ar_-{\Sigma\square_F(f, g)}@<-5ex>@/_2pc/[ddd] & & \Sigma_{\C}\square_G(A,P) \ar^-{\varphi^{-1}_G}@/^/[d] \ar^-{\Sigma\square_G(f, g)}@<5ex>@/^2pc/[ddd] \\
F(\Sigma_{\C}\ A\ P) \ar^-{\varphi_F}[u] \ar^-{\varphi^{-1}_G \circ [c, d] \circ \varphi_F}[rr] \ar_-{F[f, g]}[d] & & G(\Sigma_{\C}\ A\ P) \ar^-{G[f,g]}[d] \ar^-{\varphi_G}@/^/[u] \\
F(\Sigma_{\C}\ A'\ P') \ar_-{\varphi_F}@/_/[d] \ar^-{\varphi^{-1}_G \circ [c', d'] \circ \varphi_F}[rr] & & G(\Sigma_{\C}\ A'\ P') \\
\Sigma_{\C}\square_F(A',P') \ar_-{\varphi^{-1}_F}@/_/[u] \ar^-{[c', d']}[rr]  & & \Sigma_{\C}\square_G(A',P') \ar^-{\varphi^{-1}_G}[u] 
}
\]

\subsubsection{$\pi_0$ in $\Bialg(F, G)$}

For $\pi_0 : \Sigma_{\Bialg(F, G)}\ (A, c)\ (P, d) \to (A, c)$, we of course define $(\pi_0)_{\Bialg(F, G)} = (\pi_0)_{\C}$. We have to check that the diagram
\[
\xymatrix{
F(\Sigma_{\C}\ A\ P) \ar^-{\varphi^{-1}_G \circ [c, d] \circ \varphi_F}[rr] \ar_-{F(\pi_0)}[d] & & G(\Sigma_{\C}\ A\ P) \ar^-{G(\pi_0)}[d] \\
F(A) \ar^-{c}[rr] & & G(A)
}
\]
commutes. But by (\ref{eq:square2}), $G(\pi_0) = \pi_0 \circ \varphi_G$ so that
\begin{align*}
  G(\pi_0) \circ \varphi^{-1}_G \circ [c, d] \circ \varphi_F
%  &= \pi_0 \circ \varphi_G \circ \varphi^{-1}_G \circ [c, d] \circ \varphi_F \\
  = \pi_0 \circ [c, d] \circ \varphi_F %\\
  = c \circ \pi_0  \circ \varphi_F %\\
  = c \circ F(\pi_0).
\end{align*}

\subsubsection{$\square_{\widehat{G}}$ for $\widehat{G} : \semFam(\Bialg(F, U)) \to \Fam(\Set)$}

First, notice that $\square_U(A, B, P, Q, x) \cong P(x)$ for the forgetful functor $U$:
\begin{align*}
\square_U(A, B, P, Q, x)
&= \{ z : U(\Sigma_{\Fam(\Set)}\ (A, B)\ (P, Q)\ |\ U(\pi_0, \overline{\pi_0})(z) = x\} \\
&= \{ z : \Sigma\ A\ P\ |\ \pi_0(z) = x\} \\
&= \{ \langle x, y\rangle : \Sigma\ A\ P\ \} \cong P(x)
\end{align*}
where the isomorphism sends $\langle x, y\rangle : \Sigma\ A\ P$ to $y
: P(x)$ and the inverse sends $y : P(x)$ to $\langle x,
y\rangle$. Thus, we can replace the morphism $d : (x : F(A, B)) \to
\square_F(A, B, P, Q, x) \to \square_U(A, B, P, Q, c(x))$ with a
morphism $d' : (x : F(A, B)) \to \square_F(A, B, P, Q, x) \to P(c(x))$
in the definition of $\semFam(\Bialg(F, U))$. This also simplifies the
definition of the morphism of $\Sigma_{\Bialg(F, U)}$ to be $[c, d']
\circ \varphi_F$ instead of $\varphi^{-1}_U [c, d] \circ \varphi_F$.

We now define
\[
\square_{\widehat{G}} : \semFam(\Bialg(F, U)) \to \semFam(\Fam(\Set)).
\]
by
\[
  \square_{\widehat{G}}((A, B), (P, Q), (c, d)) =
  (\widehat{G}(A, B, c), \invIm{(\widehat{G}(\pi_0, \overline{\pi_0}))}).
\]

Explicitly, we have 

\begin{multline*}
    \square_{\widehat{G}}((A, B), (P, Q), (c, d)) = \\
  (F(A, B), G(A, B, c), \square_F(A, B, P, Q), \square_G(A, B, P, Q, c, d))
\end{multline*}
where $\square_F$ is exactly the $\square_F$ from Section
\ref{boxFamSet}, i.e.
\[
\square_F(A, B, P, Q, x) =
\{ y : F(\Sigma_{\Fam(\Set)}\ (A, B)\ (P, Q))\ |\ F(\pi_0, \overline{\pi_0})(y) = x\},
\]
and $\square_G(A, B, P, Q, c, d) : (x : F(A, B) \to G(A, B, c, x) \to \square_F(A, B, P,
Q, x) \to \Set$ is defined by
\begin{multline*}
\square_G(A, B, P, Q, c, d, x, b, y) = \\
  \{ z : G(\Sigma_{\Fam(\Set)}\ (A, B)\ (P, Q), [c, d] \circ \varphi_F, y)\ |\ G(\pi_0, \overline{\pi_0}, y, z) = b \}
\end{multline*}


\section{Example}

Let us try to calculate $\square_F$ and $\square_G$ for the contexts and types.
\begin{align*}
  F(A, B) &= 1 + \Sigma\ A\ B \\
  G(A, B, c, x) &= 1 + \Sigma a : A, b : B(a), B(c (\text{inr}(\langle a, b\rangle))).\ c(x) = a.
\end{align*}

For $\square_F$, we get
\[
\square_F(A, B, P, Q, x) = \{ y : 1 + \Sigma \langle a, p\rangle\!:\!(\Sigma\ A\ P)\ .\ \Sigma\!b\!: B(a)\ .\ Q(a, b, p)\ |\ (1 + [\pi_0, \overline{\pi_0}])(y) = x \}
\]
so that
\begin{align*}
  \square_F(A, B, P, Q, \inl(\oneelt)) &\cong \one \\
  \square_F(A, B, P, Q, \inr(\langle a, b\rangle)) &\cong (p : P(a)) \times Q(a, b, p)
\end{align*}

For $\square_G$, we have in all its glory
\begin{multline*}
  \square_G(A, B, P, Q, c, \overline{c}, x, y, \overline{x}) = \\
  \{ z : 1 + \Sigma \langle a, p\rangle\!:\!(\Sigma\ A\ P)\ . \langle
  b, q\rangle\!:\! \Sigma\!b\!: B(a)\ .\ Q(a, b, p)\ . 
  \Sigma b'\!:\!B(c(\inr(\langle a, b\rangle)))\ . \\ Q(c(\inr(\langle a,
  b\rangle)), b', \overline{c}(\inr(\langle a, b\rangle),
  \inr\langle\langle a, b\rangle, \langle a, b\rangle\rangle)).  
  [c, \overline{c}](\varphi_F(\overline{x}) = (a, p) \\|\ (1 + [\pi_0, [\pi_0, \overline{\pi_0}]])(\overline{x}, z) = y \}
\end{multline*}
which, ignoring indicies, should be
\[
  \square_G(A, B, P, Q, x, \inl(\oneelt), \overline{x}) \cong \one
\]\vskip -1cm
\begin{multline*}
  \square_F(A, B, P, Q, x, \inr(\langle a, b, b'\rangle), \overline{x}) \cong \\
(p : P(a)) \times (q : Q(a, b, p)) \times Q(c(\inr(\langle a, b\rangle)), b', \overline{c}(\inr(\langle a, b\rangle), \inr(\langle p, q\rangle))).
\end{multline*}



\end{document}

