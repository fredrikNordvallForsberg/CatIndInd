\documentclass{article}

\usepackage{amssymb}
\usepackage{amsthm}
\usepackage{amsmath}
\usepackage{amstext}

\usepackage{xypic}
\usepackage{stmaryrd}
\usepackage{enumerate}
\usepackage{mathtools}

\input header.inc

\title{Generalising $\square_F$ and $\dmap_F$ for non-endofunctors}


\author{Fredrik Nordvall Forsberg}

\begin{document}
\maketitle

\section{The standard situation}

In the ordinary situation, we have an endofunctor $F : \Set \to \Set$
representing an inductive definition, i.e.\ we have a set $\muF$ and a
constructor $c : F(\muF) \to \muF$. The eliminator has type
\begin{multline*}
\elim_F : (P : \muF \to \Set) \to (g : (x : F(\muF)) \to \square_F(\muF, P,x)
   \to P(c(x))) \\ \to (x : \muF) \to P(x)
\end{multline*}
and computation rule
\[
\elim_F(P, g, c(x)) = g(x, \dmap_F(\muF, P, \elim_F(P, g), x)).
\]
Here, 
\[
\square_F : (A : \Set) \to (P : A \to \Set) \to F(A) \to \Set
\]
gives the induction hypothesis (it is called `IH' in the
induction-recursion papers) and
\begin{multline*}
  \dmap_F : (A : \Set) \to (P : A \to \Set) \to ((x : A) \to P(x)) \\
  \to (x : F(A)) \to \square_F(A, P, x)
\end{multline*}
takes care of the recursive calls (in the IR papers, it is called
mapIH). Both $\square_F$ and $\dmap_F$ can be explicitly defined:
\begin{align*}
\square_F(A, P, x) &= \{ y : F(\Sigma\ A\ P)\ |\ F(\pi_0)(y) = x \} \\
\dmap_F(A, P, g, x) &= F(\widehat{g})(x)
\end{align*}
where $\widehat{g} : A \to \Sigma\ A \ P$ is defined by
$\widehat{g}(z) = \langle z , g(z)\rangle$. Note that $\pi_0 \circ
\widehat{g} = \id$, so that
\[
F(\pi_0)(F(\widehat{g})(x)) = F(\pi_0 \circ \widehat{g})(x))
                            = F(\id)(x)
                            = \id(x)
                            = x
\]
which shows that indeed $\dmap_F(A, P, g, x) : \square_F(A, P, x)$.
One can show that this definition of $\square_F$ is naturally
isomorphic to the definition in the IR papers (see Problem 2 in
\texttt{problems.pdf}).

However, we prefer to just use some abstract properties of $\square_F$
and $\dmap_F$, namely
\begin{equation}
  \label{eq:square}
  \tag{$\ast_{\square}$}
F (\Sigma\ A\ B) \cong \Sigma\ F(A)\ \square_F(A, B)
\end{equation}
and this isomorphism $\varphi$ satisfies
\begin{equation}
  \label{eq:square2}
  \tag{$\ast\ast_{\square}$}
\xymatrix
{
F(\Sigma\ A\ B) \ar^-{\varphi}[r] \ar_{F(\pi_0)}[d] & \Sigma\ F(A)\ \square_F(A, B) \ar^{\pi_0}[dl] \\
F(A)  & 
}
\end{equation}
i.e.\ $\pi_0 \circ \varphi = F(\pi_0)$. For $\dmap_F$, we have for all
$f : (x : A) \to B(x)$
\begin{equation}
  \label{eq:dmap}
  \tag{$\ast_{\dmap}$}
\xymatrix{
F(A) \ar^-{F(\widehat{f})}[r] \ar_{\widehat{\dmap_F(f)}}[d]& F(\Sigma\ A\ B) \ar^{\varphi}[dl] \\
\Sigma\ F(A)\ \square_F(A, B) & 
}
\end{equation}
i.e.\ $\widehat{\dmap_F(f)} = \varphi \circ F(\widehat{f})$.

\section{Generalising $\square_F$ and $\dmap_F$ for non-endofunctors}

Let us now try to generalise things, first by moving away from the
category $\Set$ to an arbitrary category $\mathbb{C}$ (with some
structure, of course), and then by replacing the endofunctor $F :
\mathbb{C} \to \mathbb{C}$ with just a functor $F : \mathbb{C} \to
\mathbb{D}$.

\subsection{The category $\sem{\Fam}(\mathbb{C})$ of interpretations
  of families in $\mathbb{C}$}

Let us for the moment assume that the category $\mathbb{C}$ is a model
of type theory. I want to define a category $\sem{\Fam}(\mathbb{C})$
of ``interpretations of families in $\mathbb{C}$''. It has as objects
an interpretation of `$A : \Set, B : A \to \Set$' in $\mathbb{C}$, and
a morphism from $\sem{A, B}$ to $\sem{A', B'}$ is an interpretation of
`$f : A \to A', g : (x : A) \to B(x) \to B'(f(x))$'.

\begin{example}[$\mathbb{C} = \Set$]
  For $\mathbb{C} = \Set$, we recover the usual category $\Fam(\Set)$
  of families of sets, i.e.\ we have objects pairs $(A, B)$ where $A :
  \Set$ and $B : A \to \Set$, and a morphism $(A, B) \to (A', B')$ is
  a pair $f : A \to A'$, $g : (x : A) \to B(x) \to B'(f(x))$.
\end{example}

\begin{example}[$\mathbb{C} = \Fam(\Set)$]
  Unless I am mistaken, the objects in $\sem{\Fam}(\Fam(\Set))$ are
  tuples $((A_0, A_1), (B_0, B_1))$ where
  \begin{align*}
    A_0 &: \Set, \\
    A_1 &: A_0 \to \Set, \\
    B_0 &: A_0 \to \Set, \\
    B_1 &: (x : A_0) \to A_1(x) \to  B_0(x) \to \Set.
  \end{align*}
 A morphism from $((A_0, A_1), (B_0, B_1))$ to
  $((A'_0, A'_1), (B'_0, B'_1))$ consists of 
  \begin{align*}
    f_0 &: A_0 \to A'_0, \\
    f_1 &: (x : A_0) \to A_1(x) \to A'_1(f_0(x)), \\
    g_0 &: (x : A_0) \to B_0(x) \to B'_0(f_0(x)) \\
    g_1 &: (x : A_0) \to (y : A_1(x)) \to (z : B_0(x)) \to B_1(x, y, z) \\
    & \qquad \qquad\qquad \qquad\qquad \qquad\qquad \to B'_1(f_0(x), f_1(x, y), g_0(x, z)).
  \end{align*}
\end{example}

\begin{question}
  Is there a better way to present $\sem{\Fam}(\mathbb{C})$?
\end{question}

\subsection{$\Sigma$ as a functor from $\sem{\Fam}(\mathbb{C})$ to $\mathbb{C}$}

Now, I notice that the sigma constructor takes a family $(A, B)$ of
sets and gives me a set $\Sigma\ A\ B$ back. It is also functorial,
since if I have a morphism $(f, g) : (A, B) \to (A', B')$, then I can
construct a morphism $[f, g] : \Sigma\ A\ B \to \Sigma\ A'\ B'$ by defining
\[
[f, g] \langle a, b\rangle = \langle f(a), g(a, b)\rangle.
\]

In the general case, I would thus like to generalise $\Sigma$ to a functor
\[
\Sigma_{\mathbb{C}} : \sem{\Fam}(\mathbb{C}) \to \mathbb{C}
\]
together with some kind of (interpretation of) projection morphisms \\
$\pi_0 : \Sigma_{\mathbb{C}}(A, B) \to A$, $\pi_1 : (x :
\Sigma_{\mathbb{C}}(A, B)) \to B(\pi_0(x))$.

\begin{question}
  I probably would like $\Sigma_{\mathbb{C}}$ to be the interpretation
  of $\Sigma$ in $\mathbb{C}$. For morphisms, this would mean that I
  want $\Sigma_{\mathbb{C}}(f,g)$ to be the interpretation of $[f,g]$?
\end{question}

\subsection{$\square$ is a functor $\square' : \Set^{\Set} \to \Fam(\Set)^{\Fam(\Set)}$}

Looking at the type 
\[
\square_F : (A : \Set) \to (P : A \to \Set) \to F(A) \to \Set
\]
of $\square_F$ again, we see that we can write this as
\[
\square' : \Set^{\Set} \to \Fam(\Set)^{\Fam(\Set)}
\]
if we define $\square'(F) = \lambda (A, B) .\ (F(A), \square_F(A,
B))$. For $\square'(F)$ to be a functor $\Fam(\Set)$ to $\Fam(\Set)$,
it must also act on morphisms. Suppose $(f, g) : (A, B) \to (A',
B')$. We must find a morphism from $\square'(F)(A, B)$ to
$\square'(F)(A', B')$, i.e.\ from
\[
(F(A),  \{ y : F(\Sigma\ A\ B)\ |\ F(\pi_0)(y) = x \})
\]
to
\[
(F(A'),  \{ y : F(\Sigma\ A'\ B')\ |\ F(\pi_0)(y) = x \}).
\]
But $f : A \to A'$, so $m = F(f) : F(A) \to F(A')$ can be our first
component. Now we need to define
\begin{multline*}
  n : (x : F(A)) \to \{ y : F(\Sigma\ A\ B)\ |\ F(\pi_0)(y) = x \} \\
  \to \{ z : F(\Sigma\ A'\ B')\ |\ F(\pi_0)(z) = F(f)(x) \}
\end{multline*}
Let $n(x) = F([f, g])$. We need to prove that $F(\pi_0)(F([f, g])(y))
= F(f)(x)$ for $y : F(\Sigma\ A\ B)$ such that $F(\pi_0)(y) = x$. But
$\pi_0 \circ [f, g] = f \circ \pi_0$, so
\[
F(\pi_0)(F([f, g])(y)) = F(\pi_0 \circ [f, g])(y) = F(f \circ
\pi_0)(y) = F(f)(F(\pi_0)(y)) = F(f)(x).
\]
Thus we can take $\square'(F)(f,g) = (n, m) = (F(f), \lambda x .\
F([f,g]))$. This shows that $\square'(F) : \Fam(\Set)^{\Fam(\Set)}$ if
$F : \Set \to \Set$ is a functor.

However, we want more! We want $\square'$ to be a functor $\square' :
\Set^{\Set} \to \Fam(\Set)^{\Fam(\Set)}$. We have defined the object
part of $\square'$, what is the morphism part? Given a natural
transformation $\eta : F \xrightarrow{\cdot} G$, we have to construct
a natural transformation $\square'(\eta) : \square'(F)
\xrightarrow{\cdot} \square'(G)$. The component of $\square'(\eta)$ at
$(A, B)$ consists of a function $f : F(A) \to G(A)$ and a function $g
: (x : F(A)) \to \square_F(A, B, x) \to \square_G(A, B, f(x))$. For
$f$, we can choose $f = \eta_A$. Choose $g(x) = \eta_{\Sigma\ A\
  B}$. Certainly $\eta_{\Sigma A B} : F(\Sigma\ A\ B) \to G(\Sigma\ A\
B)$, but do we have $G(\pi_0)(\eta_{\Sigma A B}(y)) = \eta_a(x)$,
given that $F(\pi_0)(y) = x$? Yes, because substituting $x =
F(\pi_0)(y)$, we end up with the equation $G(\pi_0)(\eta_{\Sigma A
  B}(y)) = \eta_a(F(\pi_0)(y))$, which is exactly the naturality
condition!
\[
\xymatrix
{
F(\Sigma\ A\ B) \ar^-{\eta_{\Sigma A B}}[r] \ar_{F(\pi_0)}[d] & G(\Sigma\ A\ B) \ar^{G(\pi_0)}[d] \\
F(A) \ar_-{\eta_{A}}[r] & G(A) 
}
\]
Hence we can define $\square'(\eta)_{(A, B)} = (\eta_A,
\lambda x .\ \eta_{\Sigma A B})$. The naturality of $\square'(\eta)$
follows directly from the naturality of $\eta$.

\subsection{A generalised $\square_F$ for $F$ not an endofunctor}

Assume that $\mathbb{C}$ and $\mathbb{D}$ are categories such that
$\sem{\Fam}({\mathbb{C}})$ and $\sem{\Fam}({\mathbb{D}})$ make
sense. (Then $\Sigma_{\mathbb{C}}$ and $\Sigma_{\mathbb{D}}$ should
make sense as well.) We can then generalise $\square_F$ to a functor
\[
\square : \mathbb{D}^{\mathbb{C}} \to \sem{\Fam}(\mathbb{C}) \to \sem{\Fam}(\mathbb{C})
\]
such that there is a natural isomorphism
\begin{equation}
  \label{eq:square}
  \tag{$\ast_{\square}$}
\eta : F \circ \Sigma \xrightarrow{\cdot} \Sigma\ \circ \square
\end{equation}
satisfying
\begin{equation}
  \label{eq:square2}
  \tag{$\ast\ast_{\square}$}
\xymatrix
{
F(\Sigma\ A\ B) \ar^-{\eta_{(A, B)}}[r] \ar_{F(\pi_0)}[d] & \Sigma\ F(A)\ \square_F(A, B) \ar^{\pi_0}[dl] \\
F(A)  & 
}
\end{equation}
i.e.\ $\pi_0 \circ \varphi = F(\pi_0)$.

\ldots

\section{Comparision with Fibrational Induction Rules for Initial Algebras}



\end{document}

