\documentclass{article}

\usepackage{amssymb}
\usepackage{amsthm}
\usepackage{amsmath}
\usepackage{amstext}

\usepackage[all]{xypic}
\usepackage{stmaryrd}
\usepackage{enumerate}
\usepackage{mathtools}

\input header.inc

\newcommand{\isoGR}{\ensuremath{\phi_{G\rightarrow}}}
\newcommand{\isoGL}{\ensuremath{\phi_{G\leftarrow}}}
\newcommand{\isoFL}{\ensuremath{\phi_{F}}} %{\phi_{F\leftarrow}}}

\title{Extending some categories to categories with families}


\author{Fredrik Nordvall Forsberg}

\begin{document}
\maketitle

\section{Categories with families}

\begin{definition}
  A \emph{category with families} is given by
  \begin{itemize}
  \item A category $\C$ with a terminal object $[]$,
  \item A functor $F = (\Ty, \Tm) : \C^{\text{op}} \to
    \Fam(\Set)$. For the morphism part, we introduce the notation
    $\_\{\cdot\}$ for both types and terms, i.e.\ if $f : \Delta \to
    \Gamma$ then $\_\{f\} : \Ty(\Gamma) \to \Ty(\Delta)$ and for every
    $\sigma \in \Ty(\Delta)$ we have $\_\{f\} : \Tm(\Delta, \sigma)
    \to \Tm(\Gamma, \sigma\{f\})$.
  \item For each object $\Gamma$ in $\C$ and $\sigma \in \Ty(\Gamma)$
    an object $\Gamma.\sigma$ together with a morphism $\p{\sigma} :
    \Gamma.\sigma \to \Gamma$ (the \emph{first projection}) and a term
    $\vv{\sigma} \in \Tm(\Gamma.\sigma, \sigma\{\p{\sigma}\})$ (the
    \emph{second projection}) with the following universal property:
    for each $f : \Delta \to \Gamma$ and $M \in \Tm(\Delta, \sigma\{f\})$
    there exists a unique morphism $\theta = \langle f,
    M\rangle_{\sigma} : \Delta \to \Gamma.\sigma$ such that
    $\p{\sigma} \circ \theta = f$ and
    $\vv{\sigma}\{\theta\} = M$.
  \end{itemize}
\end{definition}

\section{$\Set$}
\label{cwfSet}

Directly from Dybjer \cite{dybjer1996internalTT} , Hofmann
\cite{hofmann1997syntaxsemantics}, Buisse and Dybjer \cite{buisseDybjer2008cwflccc}, \ldots

Choose $\C = \Set$ (with $[] = \one$ any singleton), and define
\begin{align*}
  \Ty(\Gamma) &= \{ \sigma\ |\ \sigma : \Gamma \to \Set\} \\
  \Tm(\Gamma, \sigma) &= \prod_{\gamma \in \Gamma}\sigma(\gamma)
\end{align*}
(this should really be $\sigma : \Gamma \to U$ for some universe $(U,
T)$  for size considerations, and accordingly $\Tm(\Gamma, \sigma) = \prod_{\gamma \in
  \Gamma}T(\sigma(\gamma))$).  For $f : \Delta \to \Gamma$, $\sigma :
\Ty(\Gamma)$, $h : \Tm(\Gamma, \sigma)$, define
\begin{align*}
  \sigma\{f\} &: \Ty(\Delta) = \{ \sigma\ |\ \sigma : \Delta \to \Set\} \\
  \sigma\{f\} &= \sigma \circ f \\
  h\{f\} &: \Tm(\Delta, \sigma\{f\}) = \prod_{\delta \in \Delta}\sigma(f(\delta)) \\
  h\{f\} &= h \circ f
\end{align*}
For the context comprehension, define
\begin{align*}
  \Gamma.\sigma &= \sum_{\gamma \in \Gamma}\sigma(\gamma) \\
  \p{\sigma} &: \sum_{\gamma \in \Gamma}\sigma(\gamma) \to \Gamma \\
  \p{\sigma}&(\langle \gamma, s\rangle ) = \gamma \\
  \vv{\sigma} &\in \Tm(\Gamma.\sigma, \sigma\{\p{\sigma}\}) = \prod_{\langle \gamma, s \rangle \in \Gamma.\sigma} \sigma(\gamma) \\
  \vv{\sigma}&(\langle \gamma, s\rangle ) = s
\end{align*}
Given $f : \Delta \to \Gamma$ and $M \in \Tm(\Delta, \sigma\{f\}) =
\prod_{\delta \in \Delta}\sigma(f(\delta))$, we define
\[
\theta = \langle f, M\rangle_{\sigma} : \Delta \to \underbrace{\Gamma.\sigma}_{\displaystyle\sum_{\gamma \in \Gamma}\sigma(\gamma)}
\]
by $\theta(\delta) = \langle f(\delta), M(\delta)\rangle$. We then have
$\p{\sigma} \circ \theta = f$ and $\vv{\sigma}\{\theta\} = M$, and any
other function satisfying these equations must be extensionally equal
to $\theta$, hence $\theta$ is unique.


\section{$\Fam(\Set)$}
\label{cwfFamSet}

$\Fam(\Set)$ can also be extended to a category with families. We
start with $\C = \Fam(\Set)$ (and $[] = (\one, \lambda x.\one)$), and define
\begin{align*}
  \Ty(X, Y) &= \{ (A, B)\ |\ A : X \to \Set, B : (x : X) \to Y(x) \to A(x) \to \Set\} \\
  \Tm((X, Y), (A, B)) &=  \{ (h, k)\ |\ h : \prod_{x \in X}A(x), k : \prod_{x \in X, y \in Y(x)}B(x, y, h(x))\}
\end{align*}
%\text{``}\prod_{\gamma \in (X, Y)}(A, B)(\gamma)\text{''}
(similar size considerations apply as for $\Set$).
  For $(f, g) : (X, Y) \to (X', Y')$, $(A, B) :
\Ty(X', Y')$, $(h, k) : \Tm((X', Y'), (A, B))$, define
\begin{align*}
  (A, B)\{f, g\} &: \Ty(X, Y) = \{ (A, B)\ |\ A : X \to \Set, B : (x : X) \to Y(x) \to A(x) \to \Set\} \\
  (A, B)\{f, g\} &= (A, B) \circ (f, g) = (A \circ f, \lambda x, y\ .\ B(f(x), g(x, y)) \\
  (h, k)\{f, g\} &: \Tm(\Delta, \sigma\{f\}) \\
  (h, k)\{f, g\} &= (h, k) \circ (f, g) = (h \circ f, \lambda x, y\ .\ k(f(x), g(x, y)))
\end{align*}
For the context comprehension, define
\begin{align*}
  (X, Y).(A, B) &= (\sum_{x \in X}A(x), \lambda \langle x, a\rangle\ .\sum_{y \in Y(x)}B(x, y, a) ) \\
%  \p{A, B} &: (X, Y).(A, B) \to (X, Y) \\
  \p{A, B} &= (\fst, \lambda x . \fst) \\
%  \vv{A, B} &\in \Tm((X, Y).(A, B), (A, B)\{\p{A, B}\})
  \vv{A, B}&= (\snd, \lambda x . \snd)
\end{align*}
Given $(f, g) : (X', Y') \to (X, Y)$ and $(h, k) \in \Tm((X', Y'), (A, B)\{f, g\})$, we define
\[
(\theta, \psi) = \langle (f, g), (h, k)\rangle_{(A, B)} : (X', Y') \to (X, Y).(A, B)
\]
by $\theta(x) = \langle f(x), h(x)\rangle$, $\psi(x, y) = \langle g(x,
y), k(x, y)\rangle$. It is clear that $\p{\sigma} \circ \theta = f$
and $\vv{\sigma}\{\theta\} = M$ and that these conditions force
$(\theta, \psi)$ to be unique.


\section{$\Bialg(F, G)$ for $F, G : \C \to \D$}
\label{cwfDialg}

\begin{lemma}
  $\Bialg(F, G)$ has a terminal object if $\C$ and $\D$ does, and $G$
  preserves terminal objects (i.e.\ $G(\one_\C) \cong \one_\D$).
\end{lemma}
\begin{proof}
  Define $\one_{\Bialg(F, G)} \coloneqq (\one_\C, !_{F(\one_\C)})$
  where $!_{F(\one_\C)}$ is the unique map $F(\one_\C) \to
  \one_D$. For any object $(X, f)$, the unique morphism $(X, f) \to
  (\one_\C, !_{F(\one_\C)})$ is given by the unique arrow $!_{X}$ from
  $X$ to $\one_C$ in $\C$, and the diagram
\[
\xymatrix{
FX \ar^-{f}[r] \ar_-{F(!_{X})}[d] & GX \ar^-{G(!_{X})}[d] \\
F(\one_C) \ar_-{!_{F(\one_C)}}[r] & G(\one_\C) = \one_\D
}
\]
commutes since both paths are arrows into $\one_\D$, hence equal.
\end{proof}

\subsection{Some CwF preliminaries}

Clairambault \cite[4.1]{clairambault2006cwf} defines a category
$\Type_\C(\Gamma)$ of types in context $\Gamma$ from the base category
$\C$. The morphisms between $A, B \in \Ty_\C(\Gamma)$ are defined to
be the terms $f \in \Tm_\C(\Gamma.A, B\{\p{A}\})$, with identity given
by $\vv{A}$. We will be mostly interested in the composition of two
terms $f \in \Tm_\C(\Gamma.A, B\{\p{A}\})$ and $g \in \Tm_\C(\Gamma.B,
C\{\p{B}\})$, which is defined to be
\[
g \bullet f \coloneqq g\{\langle \p{A}, f\rangle_{B}\}.
\]

The following proposition says that comprehension is a functor from
``families in $\C$'' to $\C$, which is quite convenient.
\begin{lemma}
\label{compfunctor}
  Given $g : \Gamma' \to \Gamma$ and $M \in \Tm(\Gamma'.\sigma',
  \sigma\{g \circ \p{\sigma'}\})$, one can construct $\psi :
  \Gamma'.\sigma' \to \Gamma.\sigma$.
\end{lemma}
\begin{proof}
Take $\psi \coloneqq \langle g \circ \p{\sigma'}, M\rangle_{\sigma}$.
\end{proof}

\begin{lemma}
  Let $f : \Delta \to \Gamma$, $M \in \Tm(\Delta, \sigma\{f\})$, $h :
  \Theta \to \Delta$. Then $\langle f, M\rangle_\sigma \circ h = \langle f \circ h, M\{h\}\rangle_{\sigma\{f\}}$.
\end{lemma}
\begin{proof}
  $\langle f, M\rangle_\sigma \circ h$ satisfies the universal
  property for $f \circ h$ and $M\{h\}$.
\end{proof}

\subsection{The construction}

Assume that $\C$ and $\D$ are CwFs (with boxes, or can this be defined for all CwFs?).

 Assume further that for $f : \Delta \to \Gamma$, we have
 \[
 \square_G(\Gamma, \sigma)\{G(f)\} = \square_G(\Delta, \sigma\{f\})
% \square_G(\Gamma, \sigma)\{G(f)\} \cong \square_G(\Delta, \sigma\{f\})
 \]
% in $\Type_{\Bialg(F, G)}(G(\Delta))$, i.e.\ there are terms 
% \begin{align*}
% \isoGR &\in \Tm(G(\Delta).\square_G(\Gamma, \sigma)\{G(f)\}, \square_G(\Delta, \sigma\{f\})\{\p{\square_G(\Gamma, \sigma)\{G(f)\}}\}) \\
% \isoGL &\in \Tm(G(\Delta).\square_G(\Delta, \sigma\{f\}), \square_G(\Gamma, \sigma)\{G(f) \circ \p{\square_G(\Delta, \sigma\{f\})}\})
% \end{align*}
% such that $\isoGR \bullet \isoGL = \id = \isoGL \bullet \isoGR$.
\begin{remark}
  Demanding equality on the nose instead of isomorphism simplifies
  matters -- we are spared transporting terms hidden inside
  substitutions along the isomorphisms. I guess it should be possible
  in principle though.

  However, with the ususal definition of $\square_G$, one (almost)
  never has equality. (In $\Set$, for example, the left hand side is
  $\{ y : G(\Sigma\ \Gamma\ \sigma)\ |\ldots\}$ and the right hand
  side $\{ y : G(\Sigma\ \Delta\ (\sigma \circ f))\ |\ldots\}$.)  If
  $G = U$ is a forgetful functor, though, then the usual definition of
  $\square_U(\Gamma, \sigma)$ is isomorphic to a $X(\Gamma, \sigma)$
  such that $X(\Gamma, \sigma)\{U(f)\} = X(\Delta, \sigma\{f\})$. I
  see no harm in replacing $\square_U(\Gamma, \sigma)$ with $X(\Gamma,
  \sigma)$ for $U$, so that we get an actual equality? The properties
  we need $\square_U$ to have are of course preserved by isomorphism
  anyway.
\end{remark}
%
For $F$, we only require the existence of
\[
\isoFL(f) \in \Tm(F(\Delta).\square_F(\Delta, \sigma\{f\}),
\square_F(\Gamma, \sigma)\{F(f) \circ \p{\square_F(\Delta,
  \sigma\{f\})}\})
\]
which should be functorial in $f$, i.e. $\isoFL(\id) = \vv{\sigma}$
and 
\[
\isoFL(f \circ g) = \isoFL(f)\{\langle F(g) \circ \ps, \isoFL(g)\rangle\}.
\]

\begin{remark}
  $\isoFL(f)$ and $\isoFL(g)$ are not composable in
  $\Type(F(\Gamma))$, as their types depend on $f$ and $g$, but the
  ``composition'' above should be composition in some more liberal
  category (where $\vv{\sigma}$ still is the identity)? It is in any
  case exactly what we need, and holds e.g. in $\Set$ (I
  have not checked $\Fam(\Set)$, but would be very surprised if it did
  not hold).
\end{remark}



\subsubsection{Types}

Define
\[
\Ty_{\Bialg(F, G)}(\Gamma, h) = \{ (\sigma, M)\ |\ \sigma \in \Ty_{\C}(\Gamma),
 M \in \Tm_\D(F(\Gamma).\square_F(\Gamma, \sigma),\square_G(\Gamma, \sigma)\{h \circ \ps\})\}
\]
For substitutions, assume $f : (\Delta, h') \to (\Gamma, h)$, i.e.\ $f
: \Delta \to \Gamma$ and $G(f) \circ h' = h \circ F(f)$. Define for $(\sigma, M) \in \Ty_{\Bialg(F, G)}(\Gamma, h)$
\[
(\sigma, M)\{f\} = (\sigma\{f\}, M\{\langle F(f) \circ \ps, \isoFL(f)\rangle\}) \in \Ty_{\Bialg(F, G)}(\Delta, h')
\]
%
We should check that this makes sense. Since $\sigma \in
\Ty_{\C}(\Gamma)$, we have $\sigma\{f\} \in \Ty_{\C}(\Delta)$. We now
need a term in $\Tm_\D(F(\Delta).\square_F(\Delta,
\sigma\{f\}),\square_G(\Delta, \sigma\{f\})\{h \circ \ps\})$.  Since
$F(f) : F(\Delta) \to F(\Gamma)$ and
\[
\isoFL \in \Tm(F(\Delta).\square_F(\Delta, \sigma\{f\}),
\square_F(\Gamma, \sigma)\{F(f) \circ \ps\}),
\]
Lemma \ref{compfunctor} applies and we get $g \coloneqq \langle F(f) \circ \ps,
\isoFL\rangle : F(\Delta).\square_F(\Delta, \sigma\{f\}) \to
F(\Gamma).\square_F(\Gamma, \sigma)$, so that
\[
M\{g\} \in
\Tm_\D(F(\Delta).\square_F(\Delta, \sigma\{f\}),\square_G(\Gamma,
\sigma)\{h \circ \ps \circ g\})
\]
and since
\[
h \circ \ps \circ g = h \circ \ps \circ \langle F(f) \circ \ps,
\isoFL\rangle = h \circ F(f) \circ \ps = G(f) \circ h' \circ \ps
\]
and $\square_G(\Gamma, \sigma)\{G(f)\} = \square_G(\Delta,
\sigma\{f\})$, we in fact have
\[
M\{g\} \in \Tm_\D(F(\Delta).\square_F(\Delta,
\sigma\{f\}),\square_G(\Delta, \sigma\{f\})\{h' \circ \ps\})
\]
as needed.
Functoriality follows from functoriality of $\isoFL(f)$ and functoriality one level down:
\begin{align*}
(\sigma, M)\{\id\} &= (\sigma\{\id\}, M\{\langle F(\id) \circ \ps, \isoFL(\id)\rangle\}) \\ &=
  (\sigma, M\{\langle p, \vv{\sigma}\rangle\}) = (\sigma, M\{\id\}) = (\sigma, M)
\end{align*}
\begin{align*}
  (\sigma, M)\{f\}\{g\} 
  &= (\sigma\{f\}\{g\}, M\{\langle F(f) \circ \ps, \isoFL(f)\rangle\}\{\langle F(g) \circ \ps, \isoFL(g)\rangle\}) \\
  &= (\sigma\{f \circ g\}, M\{\langle F(f) \circ \ps, \isoFL(f)\rangle \circ \langle F(g) \circ \ps, \isoFL(g)\rangle\}) \\
  &= (\sigma\{f \circ g\}, M\{\langle F(f) \circ \ps \circ \langle F(g) \circ \ps, \isoFL(g)\rangle, \isoFL(f)\{\langle F(g) \circ \ps, \isoFL(g)\rangle\}) \\
  &= (\sigma\{f \circ g\}, M\{\langle F(f) \circ F(g) \circ \ps, \isoFL(f \circ g)) \\
  &= (\sigma\{f \circ g\}, M\{\langle F(f \circ g) \circ \ps, \isoFL(f \circ g)) \\
  &= (\sigma, M)\{f \circ g\}
\end{align*}


\subsubsection{Terms}


\subsubsection{Context comprehension}


\bibliographystyle{alpha}
\bibliography{../../../references/biblio}

\end{document}
