\documentclass{article}

\usepackage{amssymb}
\usepackage{amsthm}
\usepackage{amsmath}
\usepackage{amstext}

\usepackage[all]{xypic}
\usepackage{stmaryrd}
\usepackage{enumerate}
\usepackage{mathtools}

\input header.inc

\newcommand{\isoGR}{\ensuremath{\phi_{G\rightarrow}}}
\newcommand{\isoGL}{\ensuremath{\phi_{G\leftarrow}}}
\newcommand{\isoFL}{\ensuremath{\phi_{F}}} %{\phi_{F\leftarrow}}}

\title{Extending some categories to categories with families}


\author{Fredrik Nordvall Forsberg}

\begin{document}
\maketitle

\section{Categories with families}

\begin{definition}
  A \emph{category with families} is given by
  \begin{itemize}
  \item A category $\C$ with a terminal object $[]$,
  \item A functor $F = (\Ty, \Tm) : \C^{\text{op}} \to
    \Fam(\Set)$. For the morphism part, we introduce the notation
    $\_\{\cdot\}$ for both types and terms, i.e.\ if $f : \Delta \to
    \Gamma$ then $\_\{f\} : \Ty(\Gamma) \to \Ty(\Delta)$ and for every
    $\sigma \in \Ty(\Delta)$ we have $\_\{f\} : \Tm(\Delta, \sigma)
    \to \Tm(\Gamma, \sigma\{f\})$.
  \item For each object $\Gamma$ in $\C$ and $\sigma \in \Ty(\Gamma)$
    an object $\Gamma.\sigma$ together with a morphism $\p{\sigma} :
    \Gamma.\sigma \to \Gamma$ (the \emph{first projection}) and a term
    $\vv{\sigma} \in \Tm(\Gamma.\sigma, \sigma\{\p{\sigma}\})$ (the
    \emph{second projection}) with the following universal property:
    for each $f : \Delta \to \Gamma$ and $M \in \Tm(\Delta, \sigma\{f\})$
    there exists a unique morphism $\theta = \langle f,
    M\rangle_{\sigma} : \Delta \to \Gamma.\sigma$ such that
    $\p{\sigma} \circ \theta = f$ and
    $\vv{\sigma}\{\theta\} = M$.
  \end{itemize}
\end{definition}

\section{$\Set$}
\label{cwfSet}

Directly from Dybjer \cite{dybjer1996internalTT} , Hofmann
\cite{hofmann1997syntaxsemantics}, Buisse and Dybjer \cite{buisseDybjer2008cwflccc}, \ldots

Choose $\C = \Set$ (with $[] = \one$ any singleton), and define
\begin{align*}
  \Ty(\Gamma) &= \{ \sigma\ |\ \sigma : \Gamma \to \Set\} \\
  \Tm(\Gamma, \sigma) &= \prod_{\gamma \in \Gamma}\sigma(\gamma)
\end{align*}
(this should really be $\sigma : \Gamma \to U$ for some universe $(U,
T)$  for size considerations, and accordingly $\Tm(\Gamma, \sigma) = \prod_{\gamma \in
  \Gamma}T(\sigma(\gamma))$).  For $f : \Delta \to \Gamma$, $\sigma :
\Ty(\Gamma)$, $h : \Tm(\Gamma, \sigma)$, define
\begin{align*}
  \sigma\{f\} &: \Ty(\Delta) = \{ \sigma\ |\ \sigma : \Delta \to \Set\} \\
  \sigma\{f\} &= \sigma \circ f \\
  h\{f\} &: \Tm(\Delta, \sigma\{f\}) = \prod_{\delta \in \Delta}\sigma(f(\delta)) \\
  h\{f\} &= h \circ f
\end{align*}
For the context comprehension, define
\begin{align*}
  \Gamma.\sigma &= \sum_{\gamma \in \Gamma}\sigma(\gamma) \\
  \p{\sigma} &: \sum_{\gamma \in \Gamma}\sigma(\gamma) \to \Gamma \\
  \p{\sigma}&(\langle \gamma, s\rangle ) = \gamma \\
  \vv{\sigma} &\in \Tm(\Gamma.\sigma, \sigma\{\p{\sigma}\}) = \prod_{\langle \gamma, s \rangle \in \Gamma.\sigma} \sigma(\gamma) \\
  \vv{\sigma}&(\langle \gamma, s\rangle ) = s
\end{align*}
Given $f : \Delta \to \Gamma$ and $M \in \Tm(\Delta, \sigma\{f\}) =
\prod_{\delta \in \Delta}\sigma(f(\delta))$, we define
\[
\theta = \langle f, M\rangle_{\sigma} : \Delta \to \underbrace{\Gamma.\sigma}_{\displaystyle\sum_{\gamma \in \Gamma}\sigma(\gamma)}
\]
by $\theta(\delta) = \langle f(\delta), M(\delta)\rangle$. We then have
$\p{\sigma} \circ \theta = f$ and $\vv{\sigma}\{\theta\} = M$, and any
other function satisfying these equations must be extensionally equal
to $\theta$, hence $\theta$ is unique.


\section{$\Fam(\Set)$}
\label{cwfFamSet}

$\Fam(\Set)$ can also be extended to a category with families. We
start with $\C = \Fam(\Set)$ (and $[] = (\one, \lambda x.\one)$), and define
\begin{align*}
  \Ty(X, Y) &= \{ (A, B)\ |\ A : X \to \Set, B : (x : X) \to Y(x) \to A(x) \to \Set\} \\
  \Tm((X, Y), (A, B)) &=  \{ (h, k)\ |\ h : \prod_{x \in X}A(x), k : \prod_{x \in X, y \in Y(x)}B(x, y, h(x))\}
\end{align*}
%\text{``}\prod_{\gamma \in (X, Y)}(A, B)(\gamma)\text{''}
(similar size considerations apply as for $\Set$).
  For $(f, g) : (X, Y) \to (X', Y')$, $(A, B) :
\Ty(X', Y')$, $(h, k) : \Tm((X', Y'), (A, B))$, define
\begin{align*}
  (A, B)\{f, g\} &: \Ty(X, Y) = \{ (A, B)\ |\ A : X \to \Set, B : (x : X) \to Y(x) \to A(x) \to \Set\} \\
  (A, B)\{f, g\} &= (A, B) \circ (f, g) = (A \circ f, \lambda x, y\ .\ B(f(x), g(x, y)) \\
  (h, k)\{f, g\} &: \Tm(\Delta, \sigma\{f\}) \\
  (h, k)\{f, g\} &= (h, k) \circ (f, g) = (h \circ f, \lambda x, y\ .\ k(f(x), g(x, y)))
\end{align*}
For the context comprehension, define
\begin{align*}
  (X, Y).(A, B) &= (\sum_{x \in X}A(x), \lambda \langle x, a\rangle\ .\sum_{y \in Y(x)}B(x, y, a) ) \\
%  \p{A, B} &: (X, Y).(A, B) \to (X, Y) \\
  \p{A, B} &= (\fst, \lambda x . \fst) \\
%  \vv{A, B} &\in \Tm((X, Y).(A, B), (A, B)\{\p{A, B}\})
  \vv{A, B}&= (\snd, \lambda x . \snd)
\end{align*}
Given $(f, g) : (X', Y') \to (X, Y)$ and $(h, k) \in \Tm((X', Y'), (A, B)\{f, g\})$, we define
\[
(\theta, \psi) = \langle (f, g), (h, k)\rangle_{(A, B)} : (X', Y') \to (X, Y).(A, B)
\]
by $\theta(x) = \langle f(x), h(x)\rangle$, $\psi(x, y) = \langle g(x,
y), k(x, y)\rangle$. It is clear that $\p{\sigma} \circ \theta = f$
and $\vv{\sigma}\{\theta\} = M$ and that these conditions force
$(\theta, \psi)$ to be unique.


\section{$\Bialg(F, G)$ for $F, G : \C \to \D$}
\label{cwfDialg}

\begin{lemma}
  $\Bialg(F, G)$ has a terminal object if $\C$ and $\D$ does, and $G$
  preserves terminal objects (i.e.\ $G(\one_\C) \cong \one_\D$).
\end{lemma}
\begin{proof}
  Define $\one_{\Bialg(F, G)} \coloneqq (\one_\C, !_{F(\one_\C)})$
  where $!_{F(\one_\C)}$ is the unique map $F(\one_\C) \to
  \one_D$. For any object $(X, f)$, the unique morphism $(X, f) \to
  (\one_\C, !_{F(\one_\C)})$ is given by the unique arrow $!_{X}$ from
  $X$ to $\one_C$ in $\C$, and the diagram
\[
\xymatrix{
FX \ar^-{f}[r] \ar_-{F(!_{X})}[d] & GX \ar^-{G(!_{X})}[d] \\
F(\one_C) \ar_-{!_{F(\one_C)}}[r] & G(\one_\C) = \one_\D
}
\]
commutes since both paths are arrows into $\one_\D$, hence equal.
\end{proof}

\subsection{Some CwF preliminaries}

Clairambault \cite[4.1]{clairambault2006cwf} defines a category
$\Type_\C(\Gamma)$ of types in context $\Gamma$ from the base category
$\C$. The morphisms between $A, B \in \Ty_\C(\Gamma)$ are defined to
be the terms $f \in \Tm_\C(\Gamma.A, B\{\p{A}\})$, with identity given
by $\vv{A}$. We will be mostly interested in the composition of two
terms $f \in \Tm_\C(\Gamma.A, B\{\p{A}\})$ and $g \in \Tm_\C(\Gamma.B,
C\{\p{B}\})$, which is defined to be
\[
g \bullet f \coloneqq g\{\langle \p{A}, f\rangle_{B}\}.
\]

The following proposition says that comprehension is a functor from
``families in $\C$'' to $\C$, which is quite convenient.
\begin{lemma}
\label{compfunctor}
  Given $g : \Gamma' \to \Gamma$ and $M \in \Tm(\Gamma'.\sigma',
  \sigma\{g \circ \p{\sigma'}\})$, one can construct $\psi :
  \Gamma'.\sigma' \to \Gamma.\sigma$.
\end{lemma}
\begin{proof}
Take $\psi \coloneqq \langle g \circ \p{\sigma'}, M\rangle_{\sigma}$.
\end{proof}

\begin{lemma}
  Let $f : \Delta \to \Gamma$, $M \in \Tm(\Delta, \sigma\{f\})$, $h :
  \Theta \to \Delta$. Then $\langle f, M\rangle_\sigma \circ h = \langle f \circ h, M\{h\}\rangle_{\sigma\{f\}}$.
\end{lemma}
\begin{proof}
  $\langle f, M\rangle_\sigma \circ h$ satisfies the universal
  property for $f \circ h$ and $M\{h\}$.
\end{proof}

\begin{lemma}
  For every $M \in \Tm(\Gamma, \sigma)$, there is $\overline{M} :
  \Gamma \to \Gamma.\sigma$ such that $\p{\sigma} \circ \overline{M} =
  \id$ and $\vv{\sigma}\{\overline{M}\} = M$.
\end{lemma}
\begin{proof}
  There is no choice but to define $\overline{M} \coloneqq \langle \id, M\rangle_{\sigma}$.
\end{proof}

\subsection{Some box preliminaries}

In order to extend $\Bialg(F, G)$ to a CwF (with $F, G : \C \to \D$),
we assume that $\C$ and $\D$ are CwFs, and that $\square_F$ and
$\square_G$ exist and satisfy certain requirements. We collect these here.

\begin{definition}
  Assume that $\C$ and $\D$ are CwFs. They \emph{has boxes}, if for
  each $F : \C \to \D$, $\Gamma \in \C$ and $\sigma \in \Ty(\Gamma)$,
  there exists $\square_F(\Gamma, \sigma) \in \Ty(F(\Gamma))$ and an
  isomorphism $\varphi_F : F(\Gamma.\sigma) \to F(\Gamma).\square_F(\Gamma, \sigma)$
  such that $\ps \circ \varphi = F(\ps)$.

  We also require a morphism part of $\square_F$, namely that
  for $f : \Delta \to \Gamma$, $M \in \Tm(\Delta, \sigma\{f\})$ we
  have $\square_F(f, M) \in \Tm(F(\Delta), \square_F(\Gamma,
  \sigma)\{F(f)\})$ with ``naturality condition'' $\varphi_F \circ
  F(\langle f , M\rangle) = \langle F(f), \square_F(f, M)\rangle$.
\end{definition}


Assume that $\C$ and $\D$ are CwFs with boxes. We assume that for $f :
\Delta \to \Gamma$, we have
 \begin{equation} \label{square_G_eq}
 \square_G(\Gamma, \sigma)\{G(f)\} = \square_G(\Delta, \sigma\{f\})
% \square_G(\Gamma, \sigma)\{G(f)\} \cong \square_G(\Delta, \sigma\{f\})
 \end{equation}
% in $\Type_{\Bialg(F, G)}(G(\Delta))$, i.e.\ there are terms 
% \begin{align*}
% \isoGR &\in \Tm(G(\Delta).\square_G(\Gamma, \sigma)\{G(f)\}, \square_G(\Delta, \sigma\{f\})\{\p{\square_G(\Gamma, \sigma)\{G(f)\}}\}) \\
% \isoGL &\in \Tm(G(\Delta).\square_G(\Delta, \sigma\{f\}), \square_G(\Gamma, \sigma)\{G(f) \circ \p{\square_G(\Delta, \sigma\{f\})}\})
% \end{align*}
% such that $\isoGR \bullet \isoGL = \id = \isoGL \bullet \isoGR$.
 and similarly $\square_G(\id_{\Gamma}, N)\{G(f)\} =
 \square_G(\id_{\Delta}, N\{f\})$ for all $N \in \Tm(\Gamma, \sigma)$
 and $f : \Delta \to \Gamma$. Here, $\square_G(\id_{\Gamma},
 N)\{G(f)\} \in \Tm(G(\Delta), \square_G(\Gamma, \sigma)\{G(f)\})$ and
 $\square_G(\id_{\Delta}, N\{f\}) \in \Tm(G(\Delta), \square_G(\Delta,
 \sigma\{f\}))$, so both sides of the equation have the same type by
 (\ref{square_G_eq}).

\begin{remark}
  Demanding equality on the nose instead of isomorphism simplifies
  matters -- we are spared transporting terms hidden inside
  substitutions along the isomorphisms. I guess it should be possible
  in principle though.

  However, with the usual definition of $\square_G$, one (almost)
  never has equality. (In $\Set$, for example, the left hand side is
  $\{ y : G(\Sigma\ \Gamma\ \sigma)\ |\ldots\}$ and the right hand
  side $\{ y : G(\Sigma\ \Delta\ (\sigma \circ f))\ |\ldots\}$.)  If
  $G = U$ is a forgetful functor, though, then the usual definition of
  $\square_U(\Gamma, \sigma)$ is isomorphic to a $X(\Gamma, \sigma)$
  such that $X(\Gamma, \sigma)\{U(f)\} = X(\Delta, \sigma\{f\})$. I
  see no harm in replacing $\square_U(\Gamma, \sigma)$ with $X(\Gamma,
  \sigma)$ for $U$, so that we get an actual equality? The properties
  we need $\square_U$ to have are of course preserved by isomorphism
  anyway.
\end{remark}
%
For $F$, we only require the existence of
\[
\isoFL(f) \in \Tm(F(\Delta).\square_F(\Delta, \sigma\{f\}),
\square_F(\Gamma, \sigma)\{F(f) \circ \p{\square_F(\Delta,
  \sigma\{f\})}\})
\]
which should be functorial in $f$, i.e. $\isoFL(\id) = \vv{\sigma}$
and 
\[
\isoFL(f \circ g) = \isoFL(f)\{\langle F(g) \circ \ps, \isoFL(g)\rangle\}.
\]

\begin{remark}
  $\isoFL(f)$ and $\isoFL(g)$ are not composable in
  $\Type(F(\Gamma))$, as their types depend on $f$ and $g$, but the
  ``composition'' above should be composition in some more liberal
  category (where $\vv{\sigma}$ still is the identity)? It is in any
  case exactly what we need, and holds e.g. in $\Set$ (I
  have not checked $\Fam(\Set)$, but would be very surprised if it did
  not hold).
\end{remark}

We assume that $\square_F$, $\isoFL(f)$ and substitution relate in the
following ways: for every $f : \Delta \to \Gamma$, $M \in \Tm(\Gamma,
\sigma\{f\})$ and $g : \Theta \to \Delta$, we have
\begin{itemize}
\item $\square_F(f, M)\{F(g)\} = \square_F(f \circ g, M\{g\})$  \quad(``$\square_F(f \circ g) = \square_F(f) \circ \square_F(g)$''),
\item $\square_F(\ps, \vv{\sigma}) = \vv{\square_F(\Gamma, \sigma)}\{\varphi_F\}$ \quad\qquad\qquad(``$\square_F(\id) = \id$''),
\item $\isoFL(f)\{\overline{\square_F(\id, M)}\} = \square_F(f, M)$.
\end{itemize}

\subsection{The construction}

\subsubsection{Types}

Define
\[
\Ty_{\Bialg(F, G)}(\Gamma, h) = \{ (\sigma, M)\ |\ \sigma \in \Ty_{\C}(\Gamma),
 M \in \Tm_\D(F(\Gamma).\square_F(\Gamma, \sigma),\square_G(\Gamma, \sigma)\{h \circ \ps\})\}
\]
For substitutions, assume $f : (\Delta, h') \to (\Gamma, h)$, i.e.\ $f
: \Delta \to \Gamma$ and $G(f) \circ h' = h \circ F(f)$. Define for $(\sigma, M) \in \Ty_{\Bialg(F, G)}(\Gamma, h)$
\[
(\sigma, M)\{f\} = (\sigma\{f\}, M\{\langle F(f) \circ \ps, \isoFL(f)\rangle\}) \in \Ty_{\Bialg(F, G)}(\Delta, h')
\]
%
We should check that this makes sense. Since $\sigma \in
\Ty_{\C}(\Gamma)$, we have $\sigma\{f\} \in \Ty_{\C}(\Delta)$. We now
need a term in $\Tm_\D(F(\Delta).\square_F(\Delta,
\sigma\{f\}),\square_G(\Delta, \sigma\{f\})\{h \circ \ps\})$.  Since
$F(f) : F(\Delta) \to F(\Gamma)$ and
\[
\isoFL \in \Tm(F(\Delta).\square_F(\Delta, \sigma\{f\}),
\square_F(\Gamma, \sigma)\{F(f) \circ \ps\}),
\]
Lemma \ref{compfunctor} applies and we get $g \coloneqq \langle F(f) \circ \ps,
\isoFL\rangle : F(\Delta).\square_F(\Delta, \sigma\{f\}) \to
F(\Gamma).\square_F(\Gamma, \sigma)$, so that
\[
M\{g\} \in
\Tm_\D(F(\Delta).\square_F(\Delta, \sigma\{f\}),\square_G(\Gamma,
\sigma)\{h \circ \ps \circ g\})
\]
and since
\[
h \circ \ps \circ g = h \circ \ps \circ \langle F(f) \circ \ps,
\isoFL\rangle = h \circ F(f) \circ \ps = G(f) \circ h' \circ \ps
\]
and $\square_G(\Gamma, \sigma)\{G(f)\} = \square_G(\Delta,
\sigma\{f\})$, we in fact have
\[
M\{g\} \in \Tm_\D(F(\Delta).\square_F(\Delta,
\sigma\{f\}),\square_G(\Delta, \sigma\{f\})\{h' \circ \ps\})
\]
as needed.
Functoriality follows from functoriality of $\isoFL(f)$ and functoriality one level down:
\begin{align*}
(\sigma, M)\{\id\} &= (\sigma\{\id\}, M\{\langle F(\id) \circ \ps, \isoFL(\id)\rangle\}) \\ &=
  (\sigma, M\{\langle p, \vv{\sigma}\rangle\}) = (\sigma, M\{\id\}) = (\sigma, M)
\end{align*}
\begin{align*}
  (\sigma, M)\{f\}\{g\} 
  &= (\sigma\{f\}\{g\}, M\{\langle F(f) \circ \ps, \isoFL(f)\rangle\}\{\langle F(g) \circ \ps, \isoFL(g)\rangle\}) \\
  &= (\sigma\{f \circ g\}, M\{\langle F(f) \circ \ps, \isoFL(f)\rangle \circ \langle F(g) \circ \ps, \isoFL(g)\rangle\}) \\
  &= (\sigma\{f \circ g\}, M\{\langle F(f) \circ \ps \circ \langle F(g) \circ \ps, \isoFL(g)\rangle, \isoFL(f)\{\langle F(g) \circ \ps, \isoFL(g)\rangle\}) \\
  &= (\sigma\{f \circ g\}, M\{\langle F(f) \circ F(g) \circ \ps, \isoFL(f \circ g)) \\
  &= (\sigma\{f \circ g\}, M\{\langle F(f \circ g) \circ \ps, \isoFL(f \circ g)) \\
  &= (\sigma, M)\{f \circ g\}
\end{align*}

\subsubsection{Terms}

Define
\[
\Tm((\Gamma, h), (\sigma, M)) = \{ N \in \Tm_{\C}(\Gamma, \sigma) \ |\
   \square_G(\id_\Gamma, N)\{h\} = M\{\overline{\square_F(\id_{\Gamma}, N)}\}\}
                                 %M\{\langle \id_{F(\Gamma)}, \square_F(\id_{\Gamma}, N)\rangle\}\}
\]

If $f : (\Delta, h') \to (\Gamma, h)$, we define $N\{f\}$ for $N \in
\Tm((\Gamma, h), (\sigma, M))$ to be $N\{f\}$ inherited from $\C$. We
have to check that $N\{f\} \in \Tm((\Delta, h'), (\sigma, M)\{f\})$
$=$ $\Tm((\Delta, h'), (\sigma\{f\}, M\{\langle F(f) \circ \ps,
\isoFL(f)\rangle\}))$, i.e.\ that
\[
\square_G(\id_\Delta, N\{f\})\{h'\} = M\{\langle F(f) \circ \ps,
\isoFL(f)\rangle\}\{\overline{\square_F(\id_{\Delta}, N\{f\})}\}
\]
given that $\square_G(\id_\Gamma, N)\{h\} =
M\{\overline{\square_F(\id_{\Gamma}, N)}\}$. 
We calculate
%The right hand side can
%be simplified to $M\{\langle F(f), \isoFL (f)\{\langle\id_{F(\Delta)},
%\square_F(\id_{\Delta}, N\{f\})\rangle\}$. The left hand side can now
%be rewritten
\begin{align*}
  \square_G(\id_\Delta, N\{f\})\{h'\}
&= \square_G(\id_\Gamma, N)\{G(f)\}\{h'\} \\
&= \square_G(\id_\Gamma, N)\{h\}\{F(f)\} \\
&= M\{\langle\id_{F(\Gamma)}, \square_F(\id_{\Gamma}, N)\rangle\}\{F(f)\} \\
&= M\{\langle F(f), \square_F(\id_{\Gamma}, N)\{F(f)\}\rangle\} \\
&= M\{\langle F(f), \square_F(f, N\{f\})\rangle\} \\
&= M\{\langle F(f), \isoFL(f)\{\overline{\square_F(\id, N\{f\})}\rangle\} \\
&= M\{\langle F(f) \circ \ps, \isoFL(f)\rangle\}\{\overline{\square_F(\id, N\{f\})}\} .
\end{align*}

\subsubsection{Context comprehension}

Given $(\Gamma, h) \in \Bialg(F, G)$ and $(\sigma, M) \in \Ty(\Gamma,
h)$, we define
\begin{align*}
(\Gamma, h).(\sigma, M) &\coloneqq
  (\Gamma.\sigma, \varphi_G^{-1} \circ \langle h \circ \ps, M \rangle \circ \varphi_F) \\
\p{(\sigma, M)} &\coloneqq \p{\sigma} \\
\vv{(\sigma, M)} &\coloneqq \vv{\sigma} \\
\langle f , N\rangle_{(\sigma, M)} &\coloneqq \langle f , N\rangle_{\sigma}
\end{align*}
We have to check that (i) $\p{(\sigma, M)} : (\Gamma, h).(\sigma, M)
\to (\Gamma, h)$, (ii) $\vv{(\sigma, M)} \in \Tm((\Gamma, h).(\sigma,
M), (\sigma, M)\{\p{(\sigma, M)}\})$ and that (iii) $\langle f ,
N\rangle_{\sigma} : (\Delta, h') \to (\Gamma, h).(\sigma, M)$ when $f
: (\Delta, h') \to (\Gamma, h)$ and $N \in \Tm((\Gamma, h).(\sigma, M), (\sigma, M)\{p\})$.

For (i), we need that $G(\ps) \circ \varphi_G^{-1} \circ \langle h
\circ \ps, M\rangle \circ \varphi_F = h \circ F(\ps)$. But $G(\ps) = \ps \circ \varphi_G$, so
\begin{align*}
   G(\ps) \circ \varphi_G^{-1} \circ \langle h \circ \ps, M\rangle \circ \varphi_F
&= \ps \circ \varphi_G \circ \varphi_G^{-1} \circ \langle h \circ \ps, M\rangle \circ \varphi_F \\
&= \ps \circ \langle h \circ \ps, M\rangle \circ \varphi_F \\
&= h \circ \ps \circ \varphi_F \\
&= h \circ F(\ps).
\end{align*}
Hence $\p{(\sigma, M)} : (\Gamma, h).(\sigma, M) \to (\Gamma, h)$.
For (ii), we need to show that $\vv{\sigma} \in
\Tm((\Gamma.\sigma, \varphi_G^{-1} \circ \langle h \circ \ps, M
\rangle \circ \varphi_F), (\sigma\{\ps\}, M\{\langle F(\ps) \circ \ps,
\isoFL(\ps)\})$, i.e.\ that
$\square_G(\id_\Gamma, \vv{\sigma})\{\varphi_G^{-1} \circ \langle h \circ \ps, M
\rangle \circ \varphi_F\} = M\{\langle F(\ps) \circ \ps,
\isoFL(\ps)\rangle\}\{\overline{\square_F(\id_{\Gamma}, \vv{\sigma})}\}$. First, note that
$\square_G(\id_\Gamma, \vv{\sigma}) = \square_G(\p{\sigma}, \vv{\sigma})$ because
\begin{align*}
\square_G(\id, \vv{}) &= \square_G(\ps \circ \overline{\vv{}}, \vv{}\{\overline{\vv{}}\})
                      = \square_G(\ps, \vv{})\{G(\overline{\vv{}})\} \\
                      &= \square_G(\ps, \vv{}{\{\overline{\vv{}}\}})
                      = \square_G(\ps, \vv{})
\end{align*}
Now
\begin{align*}
  \square_G(\id_\Gamma, \vv{\sigma})\{\varphi_G^{-1} \circ \langle h \circ \ps, M
\rangle \circ \varphi_F\}
 &= \square_G(\ps, \vv{\sigma})\{\varphi_G^{-1} \circ \langle h \circ \ps, M
\rangle \circ \varphi_F\} \\
 &= \vv{\sigma}\{\varphi_G \circ \varphi_G^{-1} \circ \langle h \circ \ps, M
\rangle \circ \varphi_F\} \\
 &= \vv{\sigma}\{\langle h \circ \ps, M \rangle \circ \varphi_F\} \\
 &= M\{\varphi_F\}
\end{align*}
but also
\begin{align*}
  M\{\langle F(\ps) \circ \ps,
\isoFL(\ps)\rangle\}\{\overline{\square_F(\id_{\Gamma}, \vv{\sigma})}\}
 &= M\{\langle F(\ps),\isoFL(\ps)\{\overline{\square_F(\id_{\Gamma}, \vv{\sigma})}\}\rangle\} \\
 &= M\{\langle F(\ps),\square_F(\ps, \vv{\sigma})\rangle\} \\
 &= M\{\varphi_F \circ F(\langle \ps, \vv{\sigma}\rangle)\} \\
 &= M\{\varphi_F \circ F(\id)\} \\
 &= M\{\varphi_F\} .
\end{align*}

Finally, we have to check that $\langle f, N \rangle$ really is a
morphism. (Uniqueness of $\langle f, N\rangle$ is of course inherited
from $\C$.)  We are given $f : (\Delta, h') \to (\Gamma, h)$ and $N
\in \Tm((\Gamma, h).(\sigma, M), (\sigma, M)\{p\})$, that is we have
$G(f) \circ h' = h \circ F(f)$ and
\[
\square_G(\id_{\Delta}, N)\{h'\} = M\{\langle F(f) \circ \ps, \isoFL(f)\rangle\}\{\overline{\square_F(\id_{\Delta}, N)}\}
\]
which simplifies to
\[
\square_G(\id_{\Delta}, N)\{h'\} = M\{\langle F(f), \square_F(f, N)\rangle\}
\]
We also need that $\square_G(\id_{\Delta}, N) = \square_G(f, N)$ (they
have the same type thanks to our assumption $\square_G(\Gamma,
\sigma)\{G(f)\} = \square_G(\Delta, \sigma\{f\})$) -- can this be
proven from the facts we have or should it be added to our global list
of assumptions? Thus, we have
\[
\square_G(f, N)\{h'\} = \square_G(\id, N)\{h'\} = M\{\langle F(f), \square_F(f, N)\rangle\}
\]

We have to show that
\[
G(\langle f, N\rangle) \circ h' = \varphi_G^{-1} \circ \langle h \circ \ps, M\rangle \circ \varphi_F \circ F(\langle f, N\rangle)
\]
or equivalently
\[
\varphi_G \circ G(\langle f, N\rangle) \circ h' = \langle h \circ \ps, M\rangle \circ \varphi_F \circ F(\langle f, N\rangle)
\]
We have
\begin{align*}
  \varphi_G \circ G(\langle f, N\rangle) \circ h'
 &= \langle G(f), \square_G(f, N)\rangle \circ h' \\
 &= \langle G(f) \circ h', \square_G(f, N)\{h'\}\rangle \\
 &= \langle h \circ F(f), M\{\langle F(f), \square_F(f, N)\rangle\}\rangle \\
\end{align*}
and also
\begin{align*}
  \langle h \circ \ps, M\rangle \circ \varphi_F \circ F(\langle f, N\rangle)
 &= \langle h \circ \ps, M\rangle \circ \langle F(f), \square_F(f, N)\rangle \\
 &= \langle h \circ F(f), M\{\langle F(f), \square_F(f, N)\rangle\}\rangle \\
\end{align*}
and we are done.

\begin{question}
  Can we recover the inverse image type (and hence $\square_F$) from
  $\C$ as well?
\end{question}

\section{The equivalence of elim and init in the CwF formulation}

Since our constructor now has type $\inn : F(A) \to G(A)$, we need to
change the type of $\elim$ accordingly. This means that the type and
specification of $\dmap_F$ has to be changed as well; in particular,
it depends on both $F$ and $G$, and should more accurately be called $\dmap_{F,G}$.

\begin{definition}
For every $g \in \Tm(G(\Gamma), \square_G(\Gamma, \sigma))$, we demand the existence of
$\dmap_{F,G}(\Gamma, \sigma, g) \in \Tm(F(\Gamma), \square_F(\Gamma, \sigma))$
such that if $f \in \Tm(\Gamma, \sigma)$ then
\[
\varphi_F \circ F(\overline{f}) = \overline{\dmap_F(\Gamma, \sigma,
  \vv{}\{\varphi_G \circ G(\overline{f})\})}.
\]
\end{definition}

Note that if $G = \ID$, then $\square_G(A, P) = P$ and $\varphi_G =
\id$, and the above equation collapses to $\varphi_F \circ
F(\overline{f}) = \overline{\dmap_F(\Gamma, \sigma, f)}$, since $\vv{}\{\overline{f}\} = f$.


Let $F, G : \C \to \D$ with $\C$ and $\D$ CwFs. The elimination
principle for $A \in \C$, $\inn : F(A) \to G(A)$ says that if $P \in
\Ty(A)$ and $g \in \Tm(F(A).\square_F(A, P), \square_G(A, P)\{\inn
\circ \ps\})$ then there exist $\elim(P, g) \in \Tm(G(A), \square_G(A,
P))$. The computation rule says
\[
\elim(P, g)\{\inn\} = g\{\overline{\dmap(A, P, \elim(P, g))}\}.
\]

For $G = \ID$ the identity functor, we can choose $\square_{\ID}(A, P)
= P$, and this reduces to $\elim(P, g) \in \Tm(A, P)$ if $g \in
\Tm(F(A).\square_F(A, P), P\{\inn \circ \ps\})$. In $\Set$, this means
that $\elim(P, g) : (x : A) \to P(x)$ if $g : (x : F(A)) \to
\square_F(A, P, x) \to P(\inn(x))$.


\subsection{Init $\implies$ elim}

\begin{theorem}
  Let $F, G : \C \to \D$ with $\C$ and $\D$ CwFs. If $(A, \inn)$ is
  initial in $\Bialg(F, G)$ then the elimination principle holds for
  $(A, \inn)$.
\end{theorem}
\begin{proof}
  Let $P \in \Ty(A)$ and $g \in \Tm(F(A).\square_F(A, P), \square_G(A,
  P)\{\inn \circ \ps\})$ be given. Then $h \coloneqq \varphi_G^{-1}
  \circ \langle \inn \circ p, g\rangle \circ \varphi_F : F(A.P) \to
  G(A.P)$, so by initiality, we get $\fold(h) : A \to A.P$ such that
  $h \circ F(\fold(h)) = G(\fold(h)) \circ \inn$. Hence the following
  diagram commutes:
\[
\xymatrix@-1pc{
F(A) \ar^-{\inn}[rrr] \ar_-{F(\fold(h))}[d] & & & G(A) \ar^-{G(\fold(h))}[d] \\
F(A.P) \ar_{\varphi_F}[dr] \ar_-{F(\ps)}[dd] & & & G(A.P) \ar@<-1ex>_-{\varphi_G}[dl] \ar^-{G(\ps)}[dd]\\
 & F(A).\square_F(A,P) \ar^{\langle \inn \circ p, g\rangle}[r] \ar^{\ps}[dl]& G(A).\square_G(A,P) \ar@<-1ex>_-{\varphi_G^{-1}}[ur] \ar^{\ps}[dr] & \\
F(A) \ar_-{\inn}[rrr] & & & G(A)
}
\]
This means that $\ps \circ \fold(h)$ is a morphism in $\Bialg(F, G)$,
so by initiality, we must have $\ps \circ \fold(h) = \id$. We now
define $\elim(P, g) \coloneqq \vv{}\{\varphi_G \circ G(\fold(h))\}$.
We then have
\begin{align*}
  \elim(p, G) &\in \Tm(G(A), \square_G(A, P)\{ \ps \circ \varphi_G \circ U(\fold(h))\}) \\
              &=  \Tm(G(A), \square_G(A, P)\{ U(\ps) \circ U(\fold(h))\}) \\
              &=  \Tm(G(A), \square_G(A, P)\{ U(\ps \circ \fold(h))\}) \\
              &=  \Tm(G(A), \square_G(A, P)\{ U(\id)\}) \\
              &=  \Tm(G(A), \square_G(A, P))
\end{align*}
as required.

We must check that the computation rule $\elim(P, g)\{\inn\} =
g\{\overline{\dmap(A, P, \elim(P, g))}\}$ holds. Note first that since
$\ps \circ \fold(h) = \id$, we have
\[
\fold(h) = \langle \ps \circ \fold(h), \vv{}\{\fold(h)\}\rangle
         = \langle \id , \vv{}\{\fold(h)\}\rangle
         = \overline{\vv{}\{\fold(h)\}}
\]
Using this, we have
\begin{align*}
  \elim(P, g)\{\inn\} &=  \vv{}\{\varphi_G \circ G(\fold(h)) \circ \inn\} \\
                      &=  \vv{}\{\varphi_G \circ \varphi_G^{-1} \circ \langle \inn \circ \ps, g\rangle \circ \varphi_F \circ F(\fold(h))\} \\
                      &=  g\{\varphi_F \circ F(\fold(h))\} \\
                      &=  g\{\varphi_F \circ F(\overline{\vv{}\{\fold(h)\}})\} \\
                      &=  g\{\overline{\dmap(A, P, \vv{}\{\varphi_G \circ G(\overline{\vv{}\{\fold(h)\}})\})}\} \\
                      &=  g\{\overline{\dmap(A, P, \vv{}\{\varphi_G \circ G(\fold(h))\})}\} \\
                      &=  g\{\overline{\dmap(A, P, \elim(P, g))}\}
\end{align*}
as required.
\end{proof}

Let $\E$ be the equaliser category which we intend to interpret
inductive-inductive definitions in. It inherits a CwF structure from
$\Bialg(\widehat{G}, U)$.

\begin{lemma}
  $\E$ preserves $\Gamma.\sigma$, i.e.\ if $\Gamma \in \E$ then $\Gamma.\sigma \in \E$.
\end{lemma}

\begin{corollary}
    Let $F, G : \C \to \D$ with $\C$ and $\D$ CwFs. If $(A, \inn)$ is
  initial in $\E$ then the elimination principle holds for
  $(A, \inn)$.
\end{corollary}

\subsection{Elim $\implies$ weak init}


\bibliographystyle{alpha}
\bibliography{../../../references/biblio}

\end{document}
