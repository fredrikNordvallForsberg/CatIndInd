\documentclass{article}

\usepackage{amssymb}
\usepackage{amsthm}
\usepackage{amsmath}
\usepackage{amstext}

\usepackage{mathabx}

\usepackage[all]{xy}
\usepackage{stmaryrd}
\usepackage{enumerate}
\usepackage{mathtools}

\input header.inc


\newcommand{\isoGL}{\ensuremath{\phi_{G}}}
\newcommand{\isoFL}{\ensuremath{\phi_{F}}}

\title{All is well if $G$ preserves pullbacks}


\author{Fredrik Nordvall Forsberg}

\begin{document}
\maketitle

\begin{abstract}
  Using too many diagrams, which makes the text a lot longer than it
  could be, we derive the special properties needed of $G$ from the
  assumption that $G$ preserves pullbacks.
\end{abstract}

\section{General assumptions}

For every $\Gamma \in \C$ and $\sigma \in \Ty(\Gamma)$, there exists a
unique $\BOX_F(\Gamma, \sigma) \in \Ty(F(\Gamma))$ and an isomorphism
$\varphi_F : F(\Gamma.\sigma) \to F(\Gamma).\BOX_F(\Gamma, \sigma)$
such that $\ps \circ \varphi = F(\ps)$. [Up to iso etc.]

We have a morphism part: for $f : \Delta \to \Gamma$, $M \in
\Tm(\Delta, \sigma\{f\})$ there exists $\BOX_F(f, M) \in \Tm(F(\Delta),
\BOX_F(\Gamma, \sigma)\{F(f)\})$.

Finally, we have for every
$f : \Delta \to \Gamma$, $\sigma \in
\Ty(\Gamma)$ a term
\[\isoFL(f) \in \Tm(F(\Delta).\BOX_F(\Delta,
\sigma\{f\}), \BOX_F(\Gamma, \sigma)\{F(f) \circ \ps\}).\]

We use the following global assumptions, which we expect to hold for
every $F : \C \to \D$:

\begin{itemize}
\item $\varphi_F \circ F(\langle f , M\rangle) = \langle F(f), \BOX_F(f, M)\rangle$.
%\item $\isoFL(\id) = \vvs$.
\item $\isoFL(f \circ g) = \isoFL(f)\{\langle F(g) \circ \ps, \isoFL(g)\rangle\}$.
%\item $\BOX_F(f, M)\{F(g)\} = \BOX_F(f \circ g, M\{g\})$
%\item $\BOX_F(\ps, \vv{\sigma}) = \vv{\BOX_F(\Gamma, \sigma)}\{\varphi_F\}$
\item $\isoFL(f)\{\overline{\BOX_F(\id, M)}\} = \BOX_F(f, M)$.
\end{itemize}

The only assumption on $G$ is that it preserves pullbacks.

\section{$\BOX_G(\Gamma, \sigma)\{G(f)\} = \BOX_G(\Delta, \sigma\{f\})$}
\begin{theorem}
  if $G$ preserves pullbacks, then $\BOX_G(\Gamma, \sigma)\{G(f)\} = \BOX_G(\Delta, \sigma\{f\})$.
\end{theorem}
\begin{proof}
We show that there is an isomorphism $\gamma : G(\Delta.\sigma\{f\})
\to G(\Delta).\BOX_G(\Gamma, \sigma)\{G(f)\}$ satisfying $\ps \circ
\gamma = G(\ps)$. By the uniqueness of $\BOX_G(\Delta, \sigma\{f\})$,
we must thus have $\BOX_G(\Gamma, \sigma)\{G(f)\} = \BOX_G(\Delta,
\sigma\{f\})$.

The following diagram is always a pullback in every CwF:
\[
\xymatrix{
\Delta.\sigma\{f\} \ar^-{\langle f \circ \ps, \vvs\rangle}[r] \ar_-{\ps}[d] & \Gamma.\sigma \ar^-{\ps}[d]\\
\Delta \ar^-{f}[r] & \Gamma\\
}
\]
Since $G$ preserves pullbacks, also
\[
\xymatrix{
G(\Delta.\sigma\{f\}) \ar^-{G(\langle f \circ \ps, \vvs\rangle)}[r] \ar_-{G(\ps)}[d] & G(\Gamma.\sigma) \ar^-{G(\ps)}[d]\\
G(\Delta) \ar^-{G(f)}[r] & G(\Gamma)\\
}
\]
is a pullback square. But now note that
\[
G(\ps) \circ \varphi_G \circ \langle G(f) \circ \ps , \vvs\rangle = \ps \circ \langle G(f) \circ \ps , \vvs\rangle = G(f) \circ \ps
\]
so that we get a mediating arrow $h$ in the following pullback diagram:
\[
\xymatrix{
G(\Delta).\BOX_G(\Gamma, \sigma)\{G(f)\} \ar@/_1pc/^-{\ps}[ddr] \ar@/^/^-{\langle G(f) \circ \ps , \vvs\rangle}[rr] \ar@{-->}^{h}[dr] & & G(\Gamma).\BOX_G(\Gamma, \sigma) \ar@/^/^{\varphi_G^{-1}}[d]\\
& G(\Delta.\sigma\{f\}) \ar^-{G(\langle f \circ \ps, \vvs\rangle)}[r] \ar_-{G(\ps)}[d] & G(\Gamma.\sigma) \ar^-{G(\ps)}[d]\\
& G(\Delta) \ar^-{G(f)}[r] & G(\Gamma)\\
}
\]
This $h$ will be our $\gamma^{-1}$. We can construct $\gamma$
explicitly by $\gamma \coloneqq \langle \ps , \isoGL(f)\rangle \circ
\varphi_G$. By the uniqueness of the mediating arrow in the following
diagram, we get $h \circ \gamma = \id$ for free, at least after we
have checked that the outer triangles commute.
\[
\xymatrix{
G(\Delta.\sigma\{f\}) \ar@{-->}^-{\varphi_G}[r] \ar@/_2pc/_-{G(\ps)}[dddrr] 
\ar@/^5pc/^-{G(\langle f \circ \ps, \vvs\rangle)}[ddrrr] 
& G(\Delta).\BOX_G(\Gamma, \sigma\{f\}) \ar@{-->}^-{\langle \ps , \isoGL(f)\rangle}[d] & \\
& G(\Delta).\BOX_G(\Gamma, \sigma)\{G(f)\} \ar@/_1pc/^-{\ps}[ddr] \ar@/^2.5pc/^<<<<<<<{\varphi_G^{-1} \circ  \langle G(f) \circ \ps , \vvs\rangle}[drr] \ar@{-->}^{h}[dr] & &\\
& & G(\Delta.\sigma\{f\}) \ar^-{G(\langle f \circ \ps, \vvs\rangle)}[r] \ar_-{G(\ps)}[d] & G(\Gamma.\sigma) \ar^-{G(\ps)}[d]\\
& & G(\Delta) \ar^-{G(f)}[r] & G(\Gamma)\\
}
\]
The bottom one is straightforward; it can be split up into two commuting triangles:
\[
\xymatrix{
G(\Delta.\sigma\{f\}) \ar@{-->}^-{\varphi_G}[r] \ar_-{G(\ps)}[dr] 
& G(\Delta).\BOX_G(\Gamma, \sigma\{f\}) \ar@{-->}^-{\langle \ps , \isoGL(f)\rangle}[r] \ar^-{\ps}[d] & G(\Delta).\BOX_G(\Gamma, \sigma)\{G(f)\} \ar^-{\ps}[dl] \\
& G(\Delta) &
}
\]
For the top triangle, we need the following identity:

\begin{lemma}
  $\isoGL(f)\{\varphi_G\} = \BOX_G(f \circ p, \vvs)$
\end{lemma}
\begin{proof}
First, write $\varphi_G$ in a complicated way:
\begin{align*}
  \varphi_G &= \varphi_G \circ \id \\
            &= \varphi_G \circ G(\langle \ps, \vvs\rangle) \\
            &= \langle G(\ps), \BOX_G(\ps, \vvs)\rangle \\
            &= \langle G(\ps), \isoGL(\ps)\{\overline{\BOX_G(\id, \vvs)}\}\rangle \\
            &= \langle G(\ps) \circ \ps, \isoGL(\ps)\rangle \circ \overline{\BOX_G(\id, \vvs)}
\end{align*}
Then note that $\isoGL(f)\{\langle G(\ps) \circ \ps,
\isoGL(\ps)\rangle\} = \isoGL(f \circ \ps)$. Thus, we have
\begin{align*}
  \isoGL(f)\{\varphi_G\} &= \isoGL(f)\{\langle G(\ps) \circ \ps, \isoGL(\ps)\rangle \circ \overline{\BOX_G(\id, \vvs)}\} \\
  &= \isoGL(f \circ \ps)\{\overline{\BOX_G(\id, \vvs)}\} \\
  &= \BOX_G(f \circ \ps, \vvs).
\end{align*}
\end{proof}

Now, we can just calculate: we want
\[
\varphi_G^{-1} \circ \langle G(f) \circ \ps, \isoGL(f)\rangle \circ \varphi_G = G(\langle f \circ \ps, \vvs\rangle)
\]
or equivalently
\[
\langle G(f) \circ \ps, \isoGL(f)\rangle \circ \varphi_G = \varphi_G \circ G(\langle f \circ \ps, \vvs\rangle).
\]
We have
\begin{align*}
  \langle G(f) \circ \ps, \isoGL(f)\rangle \circ \varphi_G &= \langle \ps \circ \varphi_G, \isoGL(f)\{\varphi_G\}\rangle \\
&= \langle G(f) \circ \ps \circ \varphi_G, \isoGL(f)\{\varphi_G\}\rangle \\
&= \langle G(f) \circ G(\ps), \isoGL(f)\{\varphi_G\}\rangle \\
&= \langle G(f) \circ G(\ps), \BOX_G(f \circ \ps, \vvs)\rangle \\
&= \langle G(f \circ \ps), \BOX_G(f \circ \ps, \vvs)\rangle \\
&=  \varphi_G \circ G(\langle f \circ \ps, \vvs\rangle).
\end{align*}

Hence, we have $h \circ \gamma = \id$. To prove $\gamma \circ h =
\id$, consider the commuting square 
\[
\xymatrix{
G(\Delta).\BOX_G(\Gamma, \sigma)\{G(f)\} \ar@/^/^-{\langle G(f) \circ \ps , \vvs\rangle}[rr] \ar@{-->}^{h}[dr] & & G(\Gamma).\BOX_G(\Gamma, \sigma) \ar@/^/^{\varphi_G^{-1}}[d]\\
& G(\Delta.\sigma\{f\}) \ar^-{G(\langle f \circ \ps, \vvs\rangle)}[r]  & G(\Gamma.\sigma) 
}
\]
Since
\begin{align*}
G(\langle f \circ \ps, \vvs\rangle) \circ h
 &= \varphi^{-1} \circ \langle G(f \circ \ps), \BOX_G(f \circ \ps, \vvs)\rangle \circ h\\
 &= \varphi^{-1} \circ \langle G(f) \circ G(\ps) \circ h, \BOX_G(f \circ \ps, \vvs)\{h\}\rangle \\
 &= \varphi^{-1} \circ \langle G(f) \circ \ps, \BOX_G(f \circ \ps, \vvs)\{h\}\rangle,
\end{align*}
the square tells us that $\BOX_G(f \circ \ps, \vvs)\{h\} = \vvs$. Now we can show that $\gamma \circ h = \id$:
\begin{align*}
  \gamma \circ h
 &= \langle \ps , \isoGL(f)\rangle \circ \varphi_G \circ h \\
 &= \langle \ps \circ \varphi_G \circ h , \isoGL(f)\{\varphi_G \circ h\}\rangle \\
 &= \langle G(\ps) \circ h , \BOX_G(f \circ \ps, \vvs)\{h\}\rangle \\
 &= \langle \ps , \vvs\rangle = \id
\end{align*}
Thus we have found our isomorphism and have $\gamma^{-1} =
h$. Furthermore, we see that $G(\ps) \circ \gamma^{-1} = \ps$ or
equivalently $\ps \circ \gamma = G(\ps)$, so by the uniqueness of
$\BOX_G$, we must have $\BOX_G(\Gamma, \sigma)\{G(f)\} =
\BOX_G(\Delta, \sigma\{f\})$.
\end{proof}



%\bibliographystyle{alpha}
%\bibliography{../../../references/biblio}

\end{document}
