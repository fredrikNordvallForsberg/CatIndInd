\documentclass[orivec,envcountsame, ,envcountsect]{llncs}

\usepackage{amssymb}
\usepackage{amsmath}

\usepackage{mathabx} %for \bigplus %TODO: makes \langle \rangle look stupid

\usepackage{mathtools}
\usepackage{multirow}

\usepackage{enumerate}

\usepackage{hyperref}

\usepackage{microtype}

\usepackage[matrix, arrow, curve]{xy}

\usepackage{proof} %natural deduction rules

\usepackage{graphics} %resixebox

\pagestyle{plain} %temporary page numbers



%%%%%%%%%%%%%%%%%%%%%%%%%%%%%%%%%%%%%%%%%%%%
% Bits and pieces
%%%%%%%%%%%%%%%%%%%%%%%%%%%%%%%%%%%%%%%%%%%%

\newcommand{\Nat}{\ensuremath{\mathbb{N}}}
\newcommand{\suc}{\ensuremath{\text{suc}}}

\newcommand{\id}{\ensuremath{\text{id}}}

\newcommand{\refl}{\ensuremath{\text{refl}}}

\newcommand{\one}{\ensuremath{\mathbf{1}}}
\newcommand{\oneelt}{\ensuremath{\star}}
\newcommand{\two}{\ensuremath{\mathbf{2}}}

\newcommand{\inn}{\ensuremath{\text{in}}}

\newcommand{\inr}{\ensuremath{\text{inr}}}
\newcommand{\inl}{\ensuremath{\text{inl}}}
\newcommand{\coprodmorph}[2]{\ensuremath{[#1, #2]}} %todo: better name

\newcommand{\etal}{\ensuremath{\text{et. al.}}}

%%%%%%%%%%%%%%%%%%%%%%%%%%%%%%%%%%%%%%%%%%%%
% Notation from CSL article
%%%%%%%%%%%%%%%%%%%%%%%%%%%%%%%%%%%%%%%%%%%%

\newcommand{\CSLArgA}{\ensuremath{\ArgA'}}
\newcommand{\CSLArgB}{\ensuremath{\ArgB'}}
\newcommand{\CSLIndex}{\ensuremath{\text{Index}'_{\text{B}}}}


%%%%%%%%%%%%%%%%%%%%%%%%%%%%%%%%%%%%%%%%%%%%
% Different categories
%%%%%%%%%%%%%%%%%%%%%%%%%%%%%%%%%%%%%%%%%%%%
\newcommand{\C}{\ensuremath{\mathbb{C}}}
\newcommand{\D}{\ensuremath{\mathbb{D}}}

\newcommand{\Eind}[2]{\ensuremath{\mathbb{E}_{#1, #2}}}
\newcommand{\E}{\ensuremath{\mathbb{E}}}

\newcommand{\Set}{\ensuremath{\text{Set}}}
\newcommand{\FamSet}{\ensuremath{\text{Fam}(\Set)}}
\newcommand{\Dialg}[2]{\ensuremath{\text{Dialg}(#1, #2)}}

\newcommand{\FAM}{\ensuremath{\text{FAM}}}
%%%%%%%%%%%%%%%%%%%%%%%%%%%%%%%%%%%%%%%%%%%%
% Different functors
%%%%%%%%%%%%%%%%%%%%%%%%%%%%%%%%%%%%%%%%%%%%
\newcommand{\ArgA}{\ensuremath{\text{Arg}_{\text{A}}}}
\newcommand{\ArgB}{\ensuremath{\text{Arg}_{\text{B}}}}

\newcommand{\ArgC}{\ensuremath{\text{Arg}_{\text{C}}}}

\newcommand{\ArggammaA}{\ensuremath{\text{Arg}_{\gamma_\text{A}}}}
\newcommand{\ArggammaB}{\ensuremath{\text{Arg}_{\gamma_\text{B}}}}

\newcommand{\UFamSet}{\ensuremath{U}} %{\ensuremath{U_{\FamSet}}} 
\newcommand{\UDialgFG}{\ensuremath{V}} %{\ensuremath{U_{\Dialg{\ArgA}{\UFamSet}}}}
\newcommand{\UDialgArgAU}{\ensuremath{V}} %{\ensuremath{U_{\Dialg{\ArgA}{\UFamSet}}}}
\newcommand{\UDialgArgBV}{\ensuremath{W}} %{\ensuremath{U_{\Dialg{(\ArgA, \ArgB)}{\UDialgArgAU}}}}

%%%%%%%%%%%%%%%%%%%%%%%%%%%%%%%%%%%%%%%%%%%%
% Contexts and types
%%%%%%%%%%%%%%%%%%%%%%%%%%%%%%%%%%%%%%%%%%%
\newcommand{\Ctxt}{\ensuremath{\text{Ctxt}}}
\newcommand{\Type}{\ensuremath{\text{Type}}}

\newcommand{\emptyCtxt}{\ensuremath{\varepsilon}}
\newcommand{\consCtxtbare}{\ensuremath{\triangleright}}
\newcommand{\consCtxt}[2]{\ensuremath{#1 \consCtxtbare #2}}


\newcommand{\baseTybare}{\ensuremath{\iota}}
\newcommand{\baseTy}[1]{\ensuremath{\baseTybare_{#1}}}
\newcommand{\piTybare}{\ensuremath{\Pi}}
\newcommand{\piTy}[3]{\ensuremath{\piTybare_{#1}(#2, #3)}}

\newcommand{\ArgCtxt}{\ensuremath{\text{Arg}_{\Ctxt}}}
\newcommand{\ArgType}{\ensuremath{\text{Arg}_{\Type}}}

%stolen from http://tex.stackexchange.com/questions/4194/how-to-typeset-haskell-operator-and-friends
\newcommand{\concatCtxt}{\ensuremath{\mathbin{+\mkern-6mu+}}}
\newcommand{\wk}{\ensuremath{\text{wk}}}
\newcommand{\wkapp}[3]{\ensuremath{\wk_{#1}(#3, #2)}}


%%%%%%%%%%%%%%%%%%%%%%%%%%%%%%%%%%%%%%%%%%%

%%%%%%%%%%%%%%%%%%%%%%%%%%%%%%%%%%%%%%%%%%%%
% Sorted lists
%%%%%%%%%%%%%%%%%%%%%%%%%%%%%%%%%%%%%%%%%%%
\newcommand{\SortedList}{\ensuremath{\text{SortedList}}}
\newcommand{\lessList}{\ensuremath{\leq_{\text{L}}}}

\newcommand{\nilList}{\ensuremath{\text{nil}}}
\newcommand{\consListbare}{\ensuremath{\text{cons}}}
\newcommand{\consList}[3]{\ensuremath{\consListbare(#1, #2, #3)}}

\newcommand{\nilLess}[1]{\ensuremath{\text{triv}_{#1}}}
\newcommand{\consLess}[6]{\ensuremath{\ll #5, #6\gg_{#4, #1, #2, #3}}}
\newcommand{\consLessbare}{\ensuremath{\ll \cdot \gg}}

\newcommand{\ArgSList}{\ensuremath{\text{Arg}_{\text{SList}}}}
\newcommand{\ArglessList}{\ensuremath{\text{Arg}_{\lessList}}}



%%%%%%%%%%%%%%%%%%%%%%%%%%%%%%%%%%%%%%%%%%%%
% Theorem environments
%%%%%%%%%%%%%%%%%%%%%%%%%%%%%%%%%%%%%%%%%%%
\spnewtheorem{fact}{Fact}{\itshape}{\rmfamily}

%%%%%%%%%%%%%%%%%%%%%%%%%%%%%%%%%%%%%%%%%%%%
% Eliminators
%%%%%%%%%%%%%%%%%%%%%%%%%%%%%%%%%%%%%%%%%%%
\newcommand{\elim}{\ensuremath{\text{elim}}}
\newcommand{\Prop}{\ensuremath{\text{Prop}}}
\newcommand{\BOX}{\ensuremath{\Box}}
\newcommand{\dmap}{\ensuremath{\text{dmap}}}

%%%%%%%%%%%%%%%%%%%%%%%%%%%%%%%%%%%%%%%%%%%%
% Sigmas / comprehension
%%%%%%%%%%%%%%%%%%%%%%%%%%%%%%%%%%%%%%%%%%%
\newcommand{\SigmaFAM}{\ensuremath{\Sigma_{\FamSet}}}
\newcommand{\SigmaDialg}{\ensuremath{\Sigma_{\text{Dialg}}}}

\newcommand{\sigmamorph}[2]{\ensuremath{[#1, #2]}}
\newcommand{\CwFmorph}[2]{\ensuremath{\langle#1, #2\rangle}}
\newcommand{\CwFsect}[1]{\ensuremath{\widehat{#1}}}


%%% Local Variables: 
%%% mode: latex
%%% TeX-master: "calco"
%%% End: 



\title{A categorical semantics for inductive-inductive definitions}

\author{Thorsten Altenkirch\inst{1}\thanks{Supported by EPSRC grant EP/G033374/1.}\fnmsep
                                   \thanks{Supported by EPSRC grant EP/G034109/1.}
\and Peter Morris\inst{1}$^{\star\star}$
\and Fredrik Nordvall Forsberg\inst{2}$^{\star}$
\and Anton Setzer\inst{2}$^{\star}$}

\institute{School of Computer Science, University of Nottingham
      \and Department of Computer Science, Swansea University}


\begin{document}
\maketitle

\begin{abstract}
  Induction-induction is a principle for defining datatypes in
  Martin-L\"of Type Theory. An inductive-inductive definition consists
  of a set $A$, together with an $A$-indexed family $B : A \to \Set$,
  where both $A$ and $B$ are inductively defined in such a way that
  the constructors for $A$ can refer to $B$ and vice versa. In
  addition, the constructors for $B$ can refer to the constructors for
  $A$.
%
  We extend the usual initial algebra semantics for ordinary
  inductive datatypes to the inductive-inductive setting by
  considering dialgebras instead of ordinary algebras. This gives a
  new and compact formalisation of inductive-inductive definitions,
  which we prove is equivalent to the usual formulation with
  elimination and computation rules.
\end{abstract}

\section{Introduction}

Induction is an important principle of definition and reasoning,
especially so in constructive mathematics and computer science, where
the concept of inductively defined set and datatype coincide. There
are two well-established approaches to model the semantics of such
datatypes: in Martin-L\"of Type Theory
\cite{martinlof1984bibliopolis}, each set $A$ comes equipped with an
eliminator which at the same time represents reasoning by induction
over $A$ and the definition of recursive functions out of $A$.  A more
categorical approach~\cite{goguenThatcherWagnerWright1977initialAlgebra}
models datatypes as initial $T$-algebras for a suitable endofunctor
$T$.

At first, it would seem that the eliminator approach is stronger, as
it allows us to define dependent functions $(x : A) \to P(x)$, in
contrast with the non-dependent arrows $A \to B$ given by the
initiality of the algebra. However, Hermida and Jacobs
\cite{hermidaJacobs1998initInduction} showed that an eliminator can be
defined for every initial $T$-algebra, where $T$ is a polynomial
functor.  Ghani \etal\ \cite{ghaniJohannFumex2010fibind} then extended
this to arbitrary endofunctors.  This covers many forms of induction
and datatype definitions such as indexed inductive definitions
\cite{dybjer1994indfam} and induction-recursion
\cite{dybjersetzer1999finax} (Dybjer and
Setzer~\cite{dybjersetzer2003inalg} also give a direct proof for
induction-recursion).

There are, however, other meaningful forms of datatypes which are not
covered by these results. One such example are inductive-inductive
definitions~\cite{nordvallforsbergSetzer2010indind}, where a set $A$
and a function $B : A \to \Set$ are simultaneously inductively defined
(compare with induction-recursion, where $A$ is defined inductively
and $B$ recursively). In addition, the constructors for $B$ can refer
to the constructors for $A$.

In earlier work \cite{nordvallforsbergSetzer2010indind}, a subset of
the authors gave an eliminator-based axiomatisation of a type theory
with inductive-inductive definitions and showed it to be consistent
(i.e.\ not every type is inhabitated). In this article, we describe a
generalised initial algebra semantics for induction-induction, and
prove that it is equivalent to the original axiomatisation.

One could imagine that that inductive-inductive definitions could be
described by functors mapping families of sets to families of sets
(similar to the situation for induction-recursion
\cite{dybjersetzer2003inalg}), but this fails to take into account
that the constructors for $B$ should be able to refer to the
constructors for $A$. Thus, we will see that the constructor for $B$
can be described by an operation
\[
\ArgBpp : (A : \Set)(B : A \to \Set)(c : \ArgA(A, B) \to A) \to \ArgA(A, B) \to \Set
\]
where $c : \ArgA(A, B) \to A$ refers to the already defined
constructor for $A$. However, $(\ArgA, \ArgBpp)$ is then no longer an
endofunctor and we move to the more general setting of dialgebras
\cite{hagino1987phdthesis,pollZwanenburg2001dialgebra} to describe
algebras of such functors.

The equivalence between initiality and having an eliminator still
carries over to this new setting. This is because our development
actually is an instance of a generic principle for categories that are
models of Type Theory (i.e.\ Categories with Families
\cite{dybjer1996internalTT,hofmann1997syntaxsemantics}). Due to space
constraints, we have however instantiated the theory to our current
setting and are only treating the case of inductive-inductive
definitions.


\subsection{Examples of Inductive-Inductive Definitions}
\label{sec:ex-indind}

Danielsson \cite{danielsson2007formalisation} and Chapman
\cite{chapman2009eatitself} define the syntax of dependent type theory
in the theory itself by inductively defining contexts, types in a
given context and terms of a given type. Let us concentrate on
contexts and types for simplicity. There should be an empty
context $\emptyCtxt$, and if we have any context $\Gamma$ and a valid
type $\sigma$ in that context, then we should be able to extend the
context with a fresh variable of that type. We end up with the
following inductive definition of the set of contexts:
\[
\infer{\emptyCtxt : \Ctxt}{} \qquad
\infer{\consCtxt{\Gamma}{\sigma} : \Ctxt}{\Gamma : \Ctxt & \sigma : \Type(\Gamma)}
\]

For types, let us have a base type $\baseTy{}$ (valid in any context)
and a dependent function type: if $\sigma$ is a type in context $\Gamma$,
and $\tau$ is a type in $\Gamma$ extended with a fresh variable of type
$\sigma$ (the variable from the domain), then $\Pi(\sigma, \tau)$ is a
type in the original context. This leads us to the following inductive
definition of $\Type : \Ctxt \to \Set$:
\[
\infer{\baseTy{\Gamma} : \Type(\Gamma)}{\Gamma : \Ctxt} \qquad
\infer{\piTy{\Gamma}{\sigma}{\tau} : \Type(\Gamma)}{\Gamma : \Ctxt
                    & \sigma : \Type(\Gamma)
                    & \tau : \Type(\consCtxt{\Gamma}{\sigma})}
\]

Note that the definition of $\Ctxt$ refers to $\Type$, so both sets
have to be defined simultaneously. Another peculiarity is how the
introduction rule for $\Pi$ explicitly focuses on a specific
constructor in the index of the type of $B$.

For an example with more of a programming flavour, consider defining a
datatype consisting of sorted lists (of natural numbers, say). With
induction-induction, we can simultaneously define the set
$\SortedList$ of sorted lists and the predicate $\lessList : (\Nat
\times \SortedList) \to \Set$ with $n \lessList \ell$ true if $n$ is
less than or equal to every element of $\ell$.

The empty list is certainly sorted, and if we have a proof $p$ that
$n$ is less than or equal to every element of the list $\ell$, we can
put $n$ in front of $\ell$ to get a new sorted list
$\consList{n}{\ell}{p}$. Translated into introduction rules, this becomes:
\[
\infer{\nilList : \SortedList}{} \qquad
\infer{\consList{n}{\ell}{p} : \SortedList}{n : \Nat \quad & \ell : \SortedList \quad & p : n \lessList \ell}
\]
For $\lessList$, we have that every $m : \Nat$ is trivially smaller
than every element of the empty list, and if $m \leq n$ and
inductively $m \lessList \ell$, then $m \lessList \consList{n}{\ell}{p}$:
\[
\infer{\nilLess{m} : m \lessList \nilList}
%{m : \Nat}
{} \qquad
\infer{\consLess{n}{\ell}{p}{m}{q}{p_{m, \ell}}\ : m \lessList \consList{n}{\ell}{p}}
%      {m, n : \Nat \quad & \ell : \SortedList \quad & p : n \lessList \ell \quad &
       {q : m \leq n \quad & p_{m, \ell} : m \lessList \ell}
\]

Of course, there are many alternative ways to define such a datatype
using ordinary induction, but the inductive-inductive one seems
natural and might be more convenient for some purposes. It is
certainly more pleasant to work with in the proof assistant/
programming language Agda \cite{norell2007thesis} which allows
inductive-inductive definitions using the \texttt{mutual} keyword. One
aim of our investigation into inductive-inductive definitions is to
justify their existence in Agda.

\subsection{Preliminaries and notation}

We work in an extensional type theory \cite{NPS1990book} with the
following ingredients:
\begin{description}
\item[$\Set$] We use $\Set$ to denote our universe of small types, and
  we write $B : A \to \Set$ for an $A$-indexed family of sets.
\item[$\Pi$-types] Given $A : \Set$ and $B : A \to \Set$, then $\big((x :
  A) \to B(x)\big) : \Set$.  Elements of $(x : A) \to B(x)$ are
  functions $f$ that map $a : A$ to $f(a) : B(a)$.
\item[$\Sigma$-types] Given $A : \Set$ and $B : A \to \Set$, then
  $\Sigma x\!:\!A.\, B(x) : \Set$. Elements of $\Sigma x\!:\!A.\,
  B(x)$ are dependent pairs $\langle a, b\rangle$ where $a : A$ and $b : B(a)$.
  We write $\pi_0 : \Sigma x\!:\!A.\, B(x) \to A$ and $\pi_1 : (y :
  \Sigma x\!:\!A.\, B(x)) \to B(\pi_0(y))$ for the projections. We write $\{\ a : A\ |\
  B(a)\}$ for $\Sigma x\!:\!A.\, B(x)$ if $B
  : A \to \Set$ is propositional, i.e.\ there is at most one
  inhabitant in $B(a)$ for every $a : A$.
\item[$+$] Given $A, B : \Set$, we denote their coproduct $A + B$ with
  coprojections $\inl : A \to A + B$ and $\inr : B \to A + B$. We
  use $\coprodmorph{f}{g}$ for cotupling.
\item[Equality and unit types] Given $a, b : A$ we write $a = b :
  \Set$ for the equality type, inhabitated by $\refl$ if and only if
  $a = b$. In contrast, the unit type $\one$
always has a unique element  $\oneelt : \one$.
\end{description}

%todo: [f, g] : \Sigma A B \to \Sigma A' B' not introduced

\section{Inductive-Inductive Definitions as Dialgebras}
\label{sec:indind-as-dialg}

In this section, our goal is to describe each inductive-inductively
defined set as the initial object in a category constructed from a
description of the set. Just as for ordinary induction and initial
algebras, this description will be a functor of sorts, but because of
the more complicated structure involved, this will no longer be an
endofunctor.
The interesting complication is the fact that the
constructor for the second set $B$ can refer to the constructor for
the first set $A$ (as for example the argument $\tau :
\Type(\consCtxt{\Gamma}{\sigma})$ referring to
$\consCtxt{\cdot}{\cdot}$ in the introduction rule for the
$\Pi$-type). 
Thus we will model the constructor for $B$ as (the second component
of) a morphism $(c, d) : \ArgB(A, B, c) \to (A, B)$ where $c :
\ArgA(A, B) \to A$ is the constructor for $A$. Here, $(c, d)$ is a
morphism in the category of families of sets:

\begin{definition}
  The category $\FamSet$ of families of sets has as objects pairs $(A,
  B)$, where $A$ is a set and $B : A \to \Set$ is an $A$-indexed family
  of sets. A morphism from $(A, B)$ to $(A', B')$ is a pair $(f, g)$
  where $f : A \to A'$ and ${g : (x : A) \to B(x) \to B'(f(x))}$.
\end{definition}
Note that there is a forgetful functor $\UFamSet : \FamSet \to \Set$
sending $(A, B)$ to $A$ and $(f, g)$ to $f$.
Now, $c : \ArgA(A, B) \to A$ is not an $\ArgA$-algebra,
since $\ArgA : \FamSet \to \Set$ is not an endofunctor. However, we have $c :
\ArgA(A, B) \to \UFamSet(A, B)$. This means that $c$ is a $(\ArgA,
\UFamSet)$-dialgebra, as introduced by Hagino~\cite{hagino1987phdthesis}:

\begin{definition}
  Let $F, G : \C \to \D$ be functors. The category $\Dialg{F}{G}$ has
  as objects pairs $(A, f)$ where $A \in \C$ and $f : F(A) \to
  G(A)$. A morphism from $(A, f)$ to $(A', f')$ is a morphism $h : A
  \to A'$ in $\C$ such that $G(h) \circ f = f' \circ F(h)$.
\end{definition}
Note that there always is a forgetful functor $\UDialgFG : \Dialg{F}{G} \to \C$
defined by $\UDialgFG(A, f) = A$.

Putting things together, we will model the constructor for $A$ as a
morphism $c : \ArgA(A, B) \to A$ in $\Set$ and the constructor for $B$
as the second component of a morphism $(c, d) : \ArgB(A, B, c) \to (A,
B)$ in $\FamSet$. Thus, we see that the data needed to describe $(A,
B)$ as inductively generated with constructors $c, d$ are
the functors $\ArgA$ and $\ArgB$. However, we must also make sure that
the first component of $\ArgB$ coincides with $\ArgA$, i.e.\ that
$\UFamSet \circ \ArgB = \ArgA \circ \UDialgArgAU$.
\begin{definition} \label{def:indind-functors}
  An inductive-inductive definition is given by two functors
  \begin{align*}
    \ArgA &: \FamSet \to \Set \\
    \ArgB &: \Dialg{\ArgA}{\UFamSet} \to \FamSet
  \end{align*}
  such that $\UFamSet \circ \ArgB = \ArgA \circ \UDialgArgAU$.
\end{definition}
%
Since the first functor is determined by the second, we often write
such a pair as $\ArgB = (\ArgA, \ArgBpp)$ where
\[
 \ArgBpp : (A : \Set)(B : A \to \Set)(c : \ArgA(A, B) \to A) \to \ArgA(A, B) \to \Set \enspace .
 \]

% \begin{lemma}
% Equivalently, $\ArgB$ can be replaced by an operation
% \[
% \ArgBpp : (A : \Set)(B : A \to \Set)(c : \ArgA(A, B) \to A) \to \ArgA(A, B) \to \Set
% \]
% that is functorial in the sense that if if we have a
% $\Dialg{\ArgA}{\UFamSet}$-morphism $(f, g) : (A, B, c) \to (A', B',
% c')$ then this can be lifted to an action on morphisms
%  \[\ArgBpp(f, g) : (x : \ArgA(A, B)) \to \ArgBpp(A, B,
%    c, x) \to \ArgBpp(A', B', c', \ArgA(f, g)(x))
% \]
% such that $\ArgBpp(\id)(x, y) = y$ and
% \[
% \ArgBpp((f, g) \circ (f', g'))(x, y) =\ArgBpp(f, g)(\ArgA(f', g')(x), \ArgBpp(f',g')(x, y)) \enspace .
% \]
% \end{lemma}
%\begin{proof}
%  We can define $\ArgB(A, B, c) = (\ArgA(A, B), \ArgBpp(A, B, c))$ in both directions. \qed
%\end{proof}


\begin{example}[Contexts and types]
\label{ex:ctxtTy-functor}
The inductive-inductive definition of $\Ctxt : \Set$ and $\Type :
\Ctxt \to \Set$ from the introduction is given by
\begin{align*}
  \ArgCtxt(A, B) &= \one + \Sigma\, \Gamma\!:\!A .\, B(\Gamma)\\
  \ArgType(A, B, c, x) &= \one + (\Sigma\,  \sigma\!:\! B(c(x)).\, \tau\!:\!B(c(\inr(c(x), \sigma))) ) \enspace .
\end{align*}
For $\ArgCtxt$, the left summand $\one$ corresponds to the constructor
$\emptyCtxt$ taking no arguments, and the right summand $\Sigma\,
\Gamma\!:\!A .\, B(\Gamma)$ corresponds to $\consCtxt{}{}$'s two
arguments $\Gamma : \Ctxt$ and $\sigma : \Type(\Gamma)$. Similar
considerations apply to $\ArgType$.

\end{example}

\begin{example}[Sorted lists]
\label{ex:SList-functor}
The sorted list example does not fit into our framework, since
$\lessList : (\Nat \times \SortedList) \to \Set$ is indexed by $\Nat
\times \SortedList$ and not simply $\SortedList$. It is however
straightforward to generalise the construction to include this example
as well: instead of considering ordinary families, consider
``${\Nat\!\times\!A }$-indexed'' families $(A, B)$ where $A$ is a set
and $B : (\Nat \times A) \to \Set$.  The inductive-inductive
definition of $\SortedList : \Set$ and $\lessList : (\Nat \times
\SortedList) \to \Set$ is then given by
\begin{align*}
  \ArgSList(A, B) &= \one + (\Sigma\, n\!:\!\Nat .\,\Sigma\,\ell\!:\!A.\, B(n,\ell))\\
  \ArglessList(A, B, c, m, \inl(\oneelt)) &= \one \\
  \ArglessList(A, B, c, m, \inr(\langle n, \ell, p\rangle)) &= \Sigma\,m \leq n.\, B(m, \ell) \enspace .
\end{align*}
For ease of presentation, we will only consider ordinary families of
sets.

\end{example}


\subsection{A Category for Inductive-Inductive Definitions}

Given $\ArgB = (\ArgA, \ArgBpp)$ representing an inductive-inductive
definition, we will now construct a category $\Eind{\ArgB}$
whose initial object (if it exists) is the intended interpretation of
the inductive-inductive definition. Figure \ref{fig:map-of-functors}
summarises the functors and categories involved ($\UFamSet$,
$\UDialgArgAU$ and $\UDialgArgBV$ are all forgetful functors).

\begin{figure}
  \centering
  \resizebox{1\textwidth}{!}{%
    $\xymatrix{
      \Set & \ar@/_1pc/_-{\ArgA}[l] \ar@/^1pc/^-{\UFamSet}[l]  \FamSet
      & \Dialg{\ArgA}{\UFamSet} \ar@/_1pc/_-{\ArgB}[l] \ar@/^1pc/^-{\UDialgArgAU}[l]
      & \Dialg{\ArgB}{\UDialgArgAU} \ar@/_1pc/_-{(\UDialgArgAU, \UFamSet)}[l] \ar@/^1pc/^-{\UDialgArgBV}[l]
      &\ \Eind{\ArgB} \ar@{_{(}->}@<0.5ex>[l]
    }$
  }
  \caption{The functors and categories involved.}
  \label{fig:map-of-functors}
\end{figure}

One might think that the category we are looking for is
$\Dialg{\ArgB}{\UDialgArgAU}$, where $\UDialgArgAU :
\Dialg{\ArgA}{\UFamSet} \to \FamSet$ is the forgetful functor. $\Dialg{\ArgB}{\UDialgArgAU}$ has
objects $(A, B, c, (d_0, d_1))$, where $A : \Set$, $B : A \to \Set$,
$c : \ArgA(A, B) \to A$ and $(d_0, d_1) : \ArgB(A, B, c) \to (A, B)$.
The function $d_0 : \ArgA(A, B) \to A$ looks like the constructor for
$A$ that we want, but
\[
d_1 : (x : \ArgA(A, B)) \to \ArgBpp(A, B, c,x) \to B(d_0(x))
\]
does not look quite right -- we need $c$ and $d_0$ to be the same!

To this end, we will consider the equalizer of the forgetful functor
$\UDialgArgBV : \Dialg{\ArgB}{\UDialgArgAU} \to
\Dialg{\ArgA}{\UFamSet}$, $\UDialgArgBV(A, B, c, (d_0, d_1)) = (A, B, c)$, and the functor $(V, U)$ defined by 
\begin{align*}
  (\UDialgArgAU, \UFamSet)(A, B, c, (d_0, d_1))
     &\coloneqq (\UDialgArgAU(A, B, c), \UFamSet(d_0, d_1)) = (A, B, d_0) \\
  (\UDialgArgAU, \UFamSet)(f, g)
     &\coloneqq (f, g)
\end{align*}
Note that $\UFamSet(d_0, d_1) : \UFamSet(\ArgB(A, B, c)) \to
\UFamSet(\UDialgArgAU(A, B, c))$ but $\UFamSet \circ \ArgB = \ArgA
\circ \UDialgArgAU$, so that $\UFamSet(d_0, d_1) :
\ArgA(\UDialgArgAU(A, B, c)) \to \UFamSet(\UDialgArgAU(A, B, c))$. In
other words, $(\UDialgArgAU(A, B, c), \UFamSet(d_0, d_1))$ is an
object in $\Dialg{\ArgA}{\UFamSet}$, so $(\UDialgArgAU, \UFamSet)$
really is a functor from $\Dialg{\ArgB}{\UDialgArgAU}$ to
$\Dialg{\ArgA}{\UFamSet}$.

% Since $(f,g) \circ (d_0, d_1) = (d'_0, d'_1) \circ \ArgB(f, g)$ and
% $\UFamSet \circ \ArgB = \ArgA \circ \UDialgArgAU$ in
% particular means that $f \circ d_0 = d'_0 \circ \ArgA(f, g)$, every
% morphism $(f, g) : (A, B, c, d_0, d_1) \to (A', B', c', d'_0, d'_1)$
% in $\Dialg{\ArgB}{\UDialgArgAU}$ is also a morphism $(f, g) : (A, B,
% d_0) \to (A', B', d'_0)$ in $\Dialg{\ArgA}{\UFamSet}$.

\begin{definition}
  For $\ArgB = (\ArgA, \ArgBpp)$ representing an inductive-inductive
  definition, let $\Eind{\ArgB}$ be the underlying category of
  the equaliser of $(V, U)$ and the forgetful functor ${W :
    \Dialg{\ArgB}{\UDialgArgAU} \to \Dialg{\ArgA}{\UFamSet}}$.
\end{definition}
Explicitly, the category $\Eind{\ArgB}$ has
\begin{itemize}
\item Objects $(A, B, c, d)$, where $A : \Set$, $B : A \to \Set$, $c : \ArgA(A, B) \to A$, $d : (x : \ArgA(A, B)) \to \ArgBpp(A, B, c, x) \to B(c(x))$.
\item Morphisms from $(A, B, c, d)$ to $(A', B', c', d')$ are morphisms \\ $(f, g) :
  (A, B, c) \Rightarrow_{\Dialg{\ArgA}{\UFamSet}} (A', B', c')$ such that in addition
  \[
  g(c(x), d(x, y)) = d'(\ArgA(f,g)(x), \ArgBpp(f,g)(x, y)) \enspace .
  \]
% [This equation is only well-typed since $(f,g)$ is an
%  $\Dialg{\ArgA}{\UFamSet}$ morphism -- the LHS has type $B(f(c(x))$,
%  the RHS $B(c'(\ArgA(f,g)(x)))$, but $f(c(x)) = c'(\ArgA(f,g)(x)$ for
%  $\Dialg{\ArgA}{\UFamSet}$ morphisms.]
\end{itemize}


\begin{example}
  Consider the functors $\ArgCtxt$, $\ArgType$ from Example \ref{ex:ctxtTy-functor}:
\begin{align*}
  \ArgCtxt(A, B) &= \one + \Sigma\, \Gamma\!:\!A .\, B(\Gamma)\\
  \ArgType(A, B, c, x) &= \one + (\Sigma\,  \sigma\!:\! B(c(x)).\, \tau\!:\!B(c(\inr(c(x), \sigma))) ) \enspace .
\end{align*}
An object in $\Eind{(\ArgCtxt, \ArgType)}$ consists of $A : \Set$, $B :
A \to \Set$ and morphisms $c =
\coprodmorph{\emptyCtxt_{A,B}}{\consCtxtbare_{A,B}}$ and $d = \lambda \Gamma . \coprodmorph{\baseTybare_{A,B}(\Gamma)}{\piTybare_{A,B}(\Gamma)}$ which can be
split up into\footnote{Notice that $\baseTybare_{A,B} : (\Gamma :
  \ArgCtxt(A, B)) \to \ldots$ and not $\baseTybare_{A,B} : (\Gamma :
  A) \to \ldots$ as one would maybe expect. There is no difference for
  initial $A$, as we have $\ArgCtxt(A, B) \cong A$ by (a
  variant of) Lambek's Lemma.}
\begin{align*}
  \emptyCtxt_{A,B} &: \one \to A \enspace , \qquad\qquad \consCtxtbare_{A,B} : ((\Gamma : A) \times B(\Gamma)) \to A  \enspace , \\
\baseTybare_{A,B} &: (\Gamma : \ArgCtxt(A, B)) \to \one \to B(c(\Gamma)) \enspace , \\
\piTybare_{A,B} &: (\Gamma : \ArgCtxt(A, B)) \to ((\sigma : B(c(\Gamma))) \times (\tau : B(\consCtxtbare_{A,B}(c(\Gamma), \sigma)))) \to B(c(\Gamma)) \enspace.
\end{align*}
\end{example}

\begin{remark}
  The intended interpretation of the inductive-inductive definition
  given by $\ArgB = (\ArgA, \ArgBpp)$ is the initial object in
  $\Eind{\ArgB}$. Depending on the meta-theory, this might of
  course not exist. However, we will show that it does if and only if
  an eliminator for the inductive-inductive definition exists.
\end{remark}

\begin{remark} %todo: keep, remove or expand? [aka cliff, shag or marry?]
  From Figure \ref{fig:map-of-functors}, it should be clear how to
  generalise the current construction to the simultaneous definition
  of $A : \Set$, $B : A \to \Set$, $C : {(x : A)} \to B(x) \to \Set$,
  etc.: for a definition of $n$ sets, replace $\FamSet$ with the category $\FAM_n$ of families $(A_1, A_2, A_3, \ldots, A_n)$ and
  consider $\ArgA : \FAM_n \to \Set$, $\ArgB : \Dialg{\ArgA}{U} \to
  \FamSet$, $\ArgC : \Eind{\ArgB} \to \FAM_3$, \ldots taking an
  equalizer where necessary to make the constructors in different
  positions equal.
\end{remark}

\subsection{How to Exploit Initiality: An Example}
\label{sec:exploit-init}

Let us consider an example of how to use initiality to
derive a program dealing with the contexts and types from the introduction.
Suppose that we want to define a concatenation $\concatCtxt : \Ctxt
\to \Ctxt \to \Ctxt$ of contexts -- such an operation could be useful
to formulate more general
% (admissible) % since the def of Type : Ctxt -> Set is done
formation rules, such as:
\[
\infer{\sigma \times \tau : \Type(\Gamma \concatCtxt \Delta)}{\sigma : \Type(\Gamma) \quad & \tau : \Type(\Delta)}
\]
Such an operation should satisfy the equations
\[
\begin{tabular}{lcccl}
  $\Delta$ & $\concatCtxt$ & $\emptyCtxt$ & $=$ & $\Delta$ \\
  $\Delta$ & $\concatCtxt$ & $(\consCtxt{\Gamma}{\sigma})$ & $=$ & $\consCtxt{(\Delta \concatCtxt \Gamma)}{$(\wkapp{\Gamma}{\Delta}{\sigma})$}$ \enspace ,
\end{tabular}
\]
where $\wk : (\Gamma : \Ctxt) \to (\sigma : \Type(\Gamma)) \to (\Delta
: \Ctxt) \to \Type(\Delta \concatCtxt \Gamma)$ is a weakening
operation, i.e.\ if $\sigma : \Type(\Gamma)$, then
$\wkapp{\Gamma}{\Delta}{\sigma} : \Type(\Delta \concatCtxt \Gamma)$. A
moment's thought should convince us that we want $\wk$ to satisfy
\begin{align*}
  \wkapp{\Gamma}{\Delta}{\baseTy{\Gamma}} &= \baseTy{\Delta \concatCtxt \Gamma} \\
  \wkapp{\Gamma}{\Delta}{\piTy{\Gamma}{\sigma}{\tau}} &= \piTy{\Delta \concatCtxt \Gamma}{\wkapp{\Gamma}{\Delta}{\sigma}}{\wkapp{\consCtxt{\Gamma}{\sigma}}{\Delta}{\tau}} \enspace .
\end{align*}

Our hope is now to exploit the initiality of $(\Ctxt, \Type)$ to get
such operations. Recall from Example \ref{ex:ctxtTy-functor} that
$\Ctxt$, $\Type$ are the underlying sets for the inductive-inductive
definition given by the functors
\begin{align*}
  \ArgCtxt(A, B) &= \one + \Sigma\, \Gamma\!:\!A .\, B(\Gamma)\\
  \ArgType(A, B, c, x) &= \one + (\Sigma\,  \sigma\!:\! B(c(x)).\, \tau\!:\!B(c(\inr(c(x), \sigma))) ) \enspace .
\end{align*} 
From the types of $\concatCtxt : \Ctxt \to \Ctxt \to \Ctxt$ and $\wk :
(\Gamma : \Ctxt) \to (A : \Type(\Gamma)) \to (\Delta : \Ctxt) \to
\Type(\Delta \concatCtxt \Gamma)$, we see that if we can equip $(A,
B)$ where $A = \Ctxt \to \Ctxt$ and $B(f) = (\Delta : \Ctxt) \to
\Type(f(\Delta))$ with an $(\ArgCtxt, \ArgType)$ structure, initiality
will give us functions of the right type. Of course, we must choose
the right structure so that our equations will be satisfied:
\[
\begin{tabular}{lcl}
  \multicolumn{3}{l}{$\inn_A : \ArgCtxt(A, B) \to A$} \\
  $\inn_A(\inl(\oneelt))$ &$=$& $\lambda \Delta .\, \Delta$ \\
  $\inn_A(\inr(\langle f , g\rangle))$ &$=$& $\lambda \Delta .\, (\consCtxt{f(\Delta)}{g(\Delta)})$ \enspace ,\\
\\
  \multicolumn{3}{l}{$\inn_B : (x : \ArgCtxt(A, B)) \to \ArgType(A, B, \inn_A, x) \to B(\inn_A(x))$} \\
%  $\inn_B(x, \inl(\inn_A(x)) , refl))$ &$=$& $\lambda y . \baseTy{\inn_A(x,y)}$ \\
  $\inn_B(\Delta, \inl(\oneelt))$ &$=$& $\lambda \Gamma .\, \baseTy{\inn_A(\Delta)(\Gamma)}$ \\
  $\inn_B(\Delta, \inr(\langle g , h\rangle))$ &$=$& $\lambda \Gamma .\, \piTy{\inn_A(\Delta)(\Gamma)}{g(\Gamma)}{h(\Gamma)}$ \enspace .
\end{tabular}
\]
Since $(A, B, \inn_A, \inn_B)$ is an object in
$\Eind{\ArgB}$, initiality gives us a morphism $(\concatCtxt, \wk) :
(\Ctxt, \Type) \to (A, B)$ such that $(\concatCtxt, \wk) \circ
(\coprodmorph{\emptyCtxt}{\consCtxtbare},
\coprodmorph{\baseTybare}{\piTybare}) = (\inn_A, \inn_B) \circ (\ArgA,
\ArgBpp)(\concatCtxt, \wk)$.  In particular, this means that
\begin{align*}
\concatCtxt(\emptyCtxt) 
 &= \inn_A(\ArgA(\concatCtxt, \wk)(\inl(\oneelt)))
  = \inn_A(\inl(\oneelt))
  = \lambda \Delta.\, \Delta \\
\concatCtxt(\consCtxt{\Gamma}{\sigma})
 &= \inn_A(\ArgA(\inr(\langle\Gamma, \sigma\rangle)))
  = \inn_A(\langle \concatCtxt(\Gamma), \wk(\Gamma, \sigma)\rangle)) \\
  &= \lambda \Delta . \consCtxt{\concatCtxt(\Gamma, \Delta)}{\wk(\Gamma, \sigma, \Delta)} \enspace .
\end{align*}
Thus, we see that $\Delta \concatCtxt \emptyCtxt = \Delta$ and $\Delta
\concatCtxt (\consCtxt{\Gamma}{\sigma}) = \consCtxt{(\Delta
  \concatCtxt \Gamma)}{\wkapp{\Gamma}{\Delta}{\sigma}}$ as
required.\footnote{Actually, the order of the arguments is reversed,
  so we would have to define \\$\Delta \concatCtxt' \Gamma {\coloneqq}
  \concatCtxt(\Gamma, \Delta)$.} In the same way, the equations for
the weakening operation hold.


\subsection{Relationship to induction-induction as axiomatised in \cite{nordvallforsbergSetzer2010indind}} %TODO: Better title
\label{sec:functors-CSL-article}

In short, the earlier
axiomatisation \cite{nordvallforsbergSetzer2010indind} postulated the
existence of a universe $(\text{SP}'_{\text{A}},
\text{SP}'_{\text{B}})$ of codes for inductive-inductive sets,
together with decoding functions $\CSLArgA$, $\CSLArgB$ and
$\CSLIndex$. Intuitively, $\CSLArgA$ gives the domain of the
constructor $\text{intro}_{\text{A}}$ for $A$, $\CSLArgB$ the domain
for the constructor $\text{intro}_{\text{B}}$ for $B$ and
$\CSLIndex(x)$ the index of the type of $\text{intro}_{\text{B}}(x)$. More formally, they have types
\begin{align*}
\CSLArgA &: (\gamma_A : \text{SP}'_{\text{A}})(A : \Set)(B : A \to \Set) \to \Set \enspace ,\\
\CSLArgB &: (\gamma_A : \text{SP}'_{\text{A}})(\gamma_B : \text{SP}'_{\text{B}}(\gamma_A)) \\
       & \quad \to (A : \Set)(B_0 : A \to \Set)(B_1 : \CSLArgA(\gamma_A, A, B_0) \to \Set) \\
       & \qquad \to \ldots \to (B_n : \CSLArgA^n(\gamma_A, A, \vec{B}_{(n)}) \to \Set) \to \Set \enspace , \\
\CSLIndex&(\gamma_A, \gamma_B, A, B_0, \ldots, B_n) : \\ \noalign{\vskip -4mm}
              &  \qquad \CSLArgB(\gamma_A, \gamma_B, A, B_0, \ldots, B_n) \to \bigplus_{i = 0}^i \CSLArgA^n(\gamma_A, A, \vec{B}_{(i)}) \enspace ,
\end{align*} 
where $\vec{B}_{(i)} = (B_0, \ldots, B_{i - 1})$ and $\CSLArgA^i(\gamma_A, A, B_{(i)})$ is defined by
\begin{align*}
  \CSLArgA^0(\gamma_A, A, B_{(0)}) &\coloneqq A \\
  \CSLArgA^{n + 1}(\gamma_A, A, \vec{B}_{(n), B_{n + 1}}) &\coloneqq \CSLArgA(\gamma_A, \bigplus_{i = 0}^n \CSLArgA^i(\gamma_A, A, \vec{B}_{(i)}), \coprodmorph{B_0}{\coprodmorph{\ldots}{B_n}}) \enspace .
\end{align*}

 The axiomatisation
then states that we have introduction and elimination rules, i.e.\ that
for each code $\gamma = (\gamma_A, \gamma_B)$ there exists is a family
$A_\gamma : \Set$, $B_\gamma : A_\gamma \to \Set$ with constructors
$\text{intro}_{\text{A}} : \CSLArgA(\gamma_A, A_\gamma, B_\gamma) \to
A_\gamma$ and $\text{intro}_{\text{B}} : (x : \CSLArgB(\gamma,
A_\gamma, B_\gamma, B_1, \ldots, B_n)) \to B_\gamma(\text{index}(x))$,
and a suitable eliminator (see Section \ref{sec:eliminator}). Here,
$B_i = B \circ k_i$ and $\text{index}(x) = \coprodmorph{k_0,
  \ldots}{k_n}(\CSLIndex(\gamma, A, B_0, \ldots, B_n, x))$ where $k_0 = \id$ and $k_{i+1}
= \text{intro}_{\text{A}} \circ \CSLArgA^i(\coprodmorph{k_0,
  \ldots}{k_{i}}, \coprodmorph{\id', \ldots}{\id'})$.

The relationship between the codes from this axiomatisation and the formalisation in this article can now be summed up in the following proposition:
\begin{proposition}
  For each code $\gamma = (\gamma_A, \gamma_B)$, the operations $\ArggammaA : \FamSet \to \Set$ and $\Arggamma = (\ArggammaA, \ArggammaB) : \Dialg{\ArggammaA}{U} \to \FamSet$ given by
\begin{align*}
\ArggammaA(A, B) &\coloneqq \CSLArgA(\gamma_A, A, B) \enspace ,\\
\ArggammaB(A, B, c, x) &\coloneqq \{\, y\!:\!\CSLArgB(\gamma_A, \gamma_B, A, B_0, \ldots, B_n)
                                                \ |\  c(x) = \text{index}(y) \}
\end{align*}
are functorial. \qed %as in Definition \ref{def:indind-functors}. \qed
\end{proposition}
%
In Section \ref{sec:elim-init-equiv-CSL} (Theorem
\ref{thm:equiv-init-elim}), we show that that the original
introduction and elimination rules hold
if and only if $\Eind{\Arggamma}$ has an initial object.
  

\section{The Elimination Principle}
\label{sec:eliminator}

In this section, we introduce the elimination principle for
inductive-inductive definitions and show that it is equivalent to
the existence of an initial object.

\subsection{Warm-up: a Generic Eliminator for an Inductive Definition}
\label{sec:elim-ind-intro}

The traditional type-theoretical way of defining recursive functions
like the context concatenation $\concatCtxt$ in Section
\ref{sec:exploit-init} is to define them in terms of eliminators.
Roughly, the eliminator for an $F$-algebra $(A, c)$ is a term %of type
%\begin{align*}
%\elim_F :\ & (P : A \to \Set) \to \\
%         & (\stepind{c} : (x : F(A)) \to \BOX_F(P, x) \to P(c(x))) \to \\
%         & (x : A) \to P(x) \enspace .
%\end{align*}
\[
\infer{\elim_F(P, \stepind{c}) : (x : A) \to P(x)}
      {P : A \to \Set & \qquad
       \stepind{c} : (x : F(A)) \to \BOX_F(P, x) \to P(c(x))}
\]
with computation rule $\elim_F(P, \stepind{c}, c(x)) = \stepind{c}(x, \dmap_F(P,
\elim(P, \stepind{c}), x))$. Here, $\BOX_F(P) : F(A) \to \Set$ is the type of
inductive hypothesis for $P$; it consists of proofs that $P$ holds at
all $F$-substructures of $x$, and $\dmap_F(P) : (x : F(A) \to P(x)) \to (x :
F(A)) \to \BOX_F(P, x)$ takes care of recursive calls.

\begin{example}
\label{ex:natexample}
  Let $F(X) = \one + X$, i.e.\ $F$ is the functor whose initial
  algebra is $(\Nat, \coprodmorph{0}{\suc})$.  We then have
\[ %begin{align*}
  \BOX_{\one + X}(P, \inl(\oneelt)) \cong \one \qquad\quad
  \BOX_{\one + X}(P, \inr(n)) \cong P(n)
\] %end{align*}
so that the eliminator for $(\Nat, \coprodmorph{0}{\suc})$ becomes
% \begin{align*}
% \elim_{\one + X} :\ & (P : \Nat \to \Set) \to \\
%          & (f_0 : \one \to P(0)) \to \\
%          & (f_{\suc} : (n : \Nat) \to P(n) \to P(\suc(n))) \to \\
%          & (x : \Nat) \to P(x) \enspace .
% \end{align*}
\[
\infer{\elim_{\one + X}(P, \stepind{0}, \stepind{\suc}) : (x : \Nat) \to P(x)}
      {P : \Nat \to \Set & \qquad 
       \deduce{\stepind{\suc} : (n : \Nat) \to P(n) \to P(\suc(n))}{\stepind{0} : \one \to P(0)}}
\]
\end{example}

For polynomial functors $F$, $\BOX_F$ can be defined inductively over
the structure of $F$ as is given in e.g.\ Dybjer and Setzer
\cite{dybjersetzer2003inalg}. However, $\BOX_F$ and $\dmap_F$ can be
defined for any functor $F : \Set \to \Set$ by defining
\begin{align*}
  \BOX_F(P, x) &\coloneqq \{ y : F(\Sigma\,z\!:\!A .\,P(z)) | F(\pi_0)(y) = x\} \\
  \dmap_F(P, \stepind{c}, x) &\coloneqq F(\lambda y . \langle y, \stepind{c}(y)\rangle)(x) \enspace .
\end{align*}
We see that indeed $\BOX_{\one + X}(P, \inl(\oneelt)) \cong \one$ and
$\BOX_{\one + X}(P, \inr(n)) \cong P(n)$ as in Example \ref{ex:natexample}.


\subsection{The Generic Eliminator for an Inductive-Inductive Definition}
\label{sec:elim-indind}

Let us now generalise the preceding discussion from inductive
definitions (i.e.\ endo\-functors on $\Set$) to inductive-inductive
definitions (i.e.\ functors $\ArgB = (\ArgA, \ArgBpp)$ as in Definition
\ref{def:indind-functors}). Since we replace the carrier set $A$ with
a carrier family $(A, B)$, we should also replace the predicate $P : A
\to \Set$ with a ``predicate family'' $(P, Q)$ where $P : A \to \Set$
and $Q : (x : A) \to B(x) \to P(x) \to \Set$. This forces us to refine
the step function $\stepind{c} : (x : F(A)) \to \BOX_F(P, x) \to P(c(x))$
into two functions
\begin{align*}
  \stepind{c} &: (x : \ArgA(A, B)) \to \BOX_{\ArgA}(P, Q, x) \to P(c(x)) \enspace ,\\
  \stepind{d} &: (x : \ArgA(A, B)) \to (y : \ArgBpp(A, B, c, x)) \to
                   (\stepindarg{x} : \BOX_{\ArgA}(P, Q, x)) \\
              & \qquad\qquad\qquad \to \BOX_{\ArgBpp}(P, Q, c, \stepind{c}, x, y, \stepindarg{x}) \to Q(c(x), d(x, y), \stepind{c}(x, \stepindarg{x})) \enspace .
\end{align*}
%
As can already be seen in the types of $\stepind{c}$ and $\stepind{d}$ above, we replace
$\BOX_F$ with $\BOX_{\ArgA}$ and $\BOX_{\ArgBpp}$ of type
\begin{align*}
  \BOX_{\ArgA}(P, Q) :\ &\ArgA(A, B) \to \Set \enspace ,\\
%
  \BOX_{\ArgBpp}(P, Q) :\ %&(c : \ArgA(A, B) \to A) \to \\
                     &(\stepind{c} : (x : \ArgA(A, B)) \to \BOX_{\ArgA}(P, Q, x) \to P(c(x))) \to \\
                     &(x : \ArgA(A, B)) \to (y : \ArgBpp(A, B, c, x)) \to \\
                     & (\stepindarg{x} : \BOX_{\ArgA}(P, Q, x)) \to \Set
\end{align*} %todo: change order to x xbar y ? YES
%
and we replace $\dmap_F$ with $\dmap_{\ArgA}$, $\dmap_{\ArgBpp}$ of type
\begin{align*}
  \dmap_{\ArgA}(P, Q) :\ & (f : (x : A) \to P(x)) \to \\
                 & (g : (x : A) \to (y : B(x)) \to Q(x, y, f(x))) \to \\
                 & (x : \ArgA(A, B)) \to \BOX_{\ArgA}(P, Q, x) \\
%
  \dmap_{\ArgBpp}(P, Q) :\ &(\stepind{c} : (x : \ArgA(A, B)) \to \BOX_{\ArgA}(P, Q, x) \to P(c(x))) \to \\
                 & (f : (x : A) \to P(x)) \to \\
                 & (g : (x : A) \to (y : B(x)) \to Q(x, y, f(x))) \to \\
                 & (x : \ArgA(A, B)) \to (y : \ArgBpp(A, B, c, x)) \\ 
                 & \quad \to \BOX_{\ArgBpp}(P, Q, \stepind{c}, x, y, \dmap_{\ArgA}(P, Q, f, g, x)) \enspace .
\end{align*}

We can define $\BOX_{\ArgA}$, $\BOX_{\ArgBpp}$, $\dmap_{\ArgA}$ and $\dmap_{\ArgBpp}$ for arbitrary functors
representing inductive-inductive definitions. First, define:

\begin{definition}
    Let $(A, B) \in \FamSet$, $P : A \to \Set$, $Q : (x : A) \to B(x)
    \to P(x) \to \Set$.
  \begin{enumerate}[(i)]
  \item Define $\SigmaFAM (A, B)\ (P, Q) \in \FamSet$ by
 \[
\SigmaFAM (A, B)\ (P, Q) \coloneqq (\Sigma\ A\ P, \lambda
    \langle a, p\rangle . \Sigma b\!:\!B(a).\ Q(a, b, p))
 \]
  \item In addition, for $(f, g) : (A, B) \to (A', B')$ and
  \begin{align*}
h &: (x : A) \to P(f(x)) \\
k &: (x : A) \to (y : B(x)) \to Q(f(x), g(x, y), h(x)) \enspace ,
  \end{align*}
define $\CwFmorph{(f, g)}{(h, k)} : (A, B) \to \SigmaFAM (A', B')\ (P, Q)$ by
\[
\CwFmorph{(f, g)}{(h, k)} \coloneqq (\lambda x .\, \langle f(x), h(x)\rangle, \lambda x\,y .\, \langle g(x, y), k(x, y)\rangle)
\]
%\item For $(f, g) : (A, B) \to (A', B')$ and
%  \begin{align*}
%h &: (x : A) \to P(f(x)) \\
%k &: (x : A) \to (y : B(x)) \to Q(f(x), g(x, y), h(x)) \enspace ,
%  \end{align*}
%define $\sigmamorph{(f, g)}{(h, k)} : \SigmaFAM (A, B)\ (P, Q) \to \SigmaFAM (A', B')\ (P', Q')$
%by $\sigmamorph{(f, g)}{(h, k)} = \CwFmorph{(f, g) \circ (\pi_0, \pi'_0)}{(h, k)}$
\item For $h : (x : A) \to P(x)$ and $ k : (x : A) \to (y : B(x)) \to
  Q(x, y, h(x))$, define $\CwFsect{(h, k)} : (A, B) \to \SigmaFAM (A, B)\
  (P, Q)$ by $\CwFsect{(h, k)} \coloneqq \CwFmorph{\id}{(h, k)}$.
  \end{enumerate}
\end{definition}
We have $(\pi_0, \pi'_0) \coloneqq (\pi_0, \lambda x .\,
\pi_0) : \SigmaFAM (A, B)\ (P, Q) \to (A, B)$ with $(\pi_0, \pi'_0)
\circ \CwFsect{(h, k)} = \id$. This is enough to define $\BOX_{\ArgA}$ and $\dmap_{\ArgA}$:

\begin{definition}
Define $\BOX_{\ArgA}$ and $\dmap_{\ArgA}$ with types as above by
\begin{align*}
\BOX_{\ArgA}(P, Q, x) &\coloneqq %\\
    \{ y : \ArgA(\SigmaFAM (A, B)\ (P, Q))\ | \ArgA(\pi_0, \pi'_0)(y) = x\} \enspace , \\
\dmap_{\ArgA}(P, Q, f, g) &\coloneqq %\\
    \ArgA(\CwFsect{(f, g)}) \enspace .
\end{align*}
\end{definition}


\noindent Note that we have an isomorphism
\[
\varphi_{\ArgA} : \ArgA(\SigmaFAM (A, B)\ (P, Q)) \to \Sigma\,x\!:\! \ArgA(A, B) .\, \BOX_{\ArgA}(P, Q, x)
\]
defined by $\varphi_{\ArgA}(x) = \langle \ArgA(\pi_0, \pi'_0)(x), x\rangle$.

\begin{definition}
Given $P$, $Q$, $\stepind{c}$, $x$, $y$, $\stepindarg{x}$ as above, define
  \begin{enumerate}[(i)]
  \item $\SigmaDialg (A, B, c)\ (P, Q, \stepind{c}) \coloneqq (\SigmaFAM\ (A,
    B)\ (P, Q), \sigmamorph{c}{\stepind{c}} \circ \varphi_{\ArgA})$,
  \item $\BOX_{\ArgBpp}(P, Q, \stepind{c}, x, y, \stepindarg{x}) \coloneqq \\
\{ z : \ArgBpp((\SigmaDialg (A, B, c)\ (P, Q, \stepind{c})), \stepindarg{x}
)\ |\ \ArgBpp(\pi_0, \pi'_0, \stepindarg{x}, z) = y \}$,
\item $\dmap_{\ArgBpp}(P, Q, \stepind{c}, f, g) \coloneqq \ArgBpp(\CwFsect{(f, g)})$.
  \end{enumerate}
\end{definition}
We can now define what the eliminators for inductive-inductive definitions are:
\begin{definition}
  We say that $(A, B, c, d)$ in $\Eind{\ArgB}$ has an
  eliminator, if there exist two terms 
% \begin{align*}
%   \elim_{\ArgA} :\ & (P : A \to \Set)(Q : (x : A) \to B(x) \to P(x) \to \Set) \to \\
%   & \big(\stepind{c} : (x : \ArgA(A, B)) \to \BOX_{\ArgA}(P, Q, x) \to P(c(x))\big) \to \\
%   & \big(\stepind{d} : (x : \ArgA(A, B)) \to (y : \ArgBpp(A, B, c, x)) \to
%   (\stepindarg{x} : \BOX_{\ArgA}(P, Q, x))  \\
%   & \quad \to  \BOX_{\ArgBpp}(P, Q, c, \stepind{c}, x, y, \stepindarg{x}) \to 
%      Q(c(x), d(x, y), \stepind{c}(x, \stepindarg{x}))\big) \to \\
%   & (x : A) \to P(x) \\
% %\end{align*}
% %\begin{align*}
%   \elim_{\ArgBpp} :\ & (P : A \to \Set)(Q : (x : A) \to B(x) \to P(x) \to \Set) \to \\
%   & \big(\stepind{c} : (x : \ArgA(A, B)) \to \BOX_{\ArgA}(P, Q, x) \to P(c(x))\big) \to \\
%   & \big(\stepind{d} : (x : \ArgA(A, B)) \to (y : \ArgBpp(A, B, c, x)) \to
%   (\stepindarg{x} : \BOX_{\ArgA}(P, Q, x))  \\
%   & \quad \to  \BOX_{\ArgBpp}(P, Q, c, \stepind{c}, x, y, \stepindarg{x}) \to 
%      Q(c(x), d(x, y), \stepind{c}(x, \stepindarg{x}))\big) \to \\
% %  & \ldots \\
%   & (x : A) \to (y : B(x)) \to Q(x, y, \elim_{\ArgA}(P, Q, \stepind{c}, \stepind{d}, x))
% \end{align*}
\[
\deduce{\elim_{\ArgBpp}(P, Q, \stepind{c}, \stepind{d}) : (x : A) \to (y : B(x)) \to Q(x, y, \elim_{\ArgA}(P, Q, \stepind{c}, \stepind{d}, x))}
       {\infer{\elim_{\ArgA}(P, Q, \stepind{c}, \stepind{d}) : (x : A) \to P(x)}
       {\deduce{
          \deduce{
            \deduce{\qquad\qquad\qquad\qquad \to  \BOX_{\ArgBpp}(P, Q, c, \stepind{c}, x, y, \stepindarg{x}) \to 
      Q(c(x), d(x, y), \stepind{c}(x, \stepindarg{x}))}{\stepind{d} : (x : \ArgA(A, B)) \to (y : \ArgBpp(A, B, c, x)) \to
   (\stepindarg{x} : \BOX_{\ArgA}(P, Q, x))}}
                 {\stepind{c} : (x : \ArgA(A, B)) \to \BOX_{\ArgA}(P, Q, x) \to P(c(x))}}
          {\deduce{Q : (x : A) \to B(x) \to P(x) \to \Set}{P : A \to \Set}}}}
\]
with 
\begin{align*}
\elim_{\ArgA}(P, Q, \stepind{c}, \stepind{d}, c(x)) &= \stepind{c}(x, \dmap_{\ArgA}') \\
 \elim_{\ArgBpp}(P, Q, \stepind{c}, \stepind{d}, c(x), d(x, y)) &=
 \stepind{d}(x, y, \dmap_{\ArgA}',  \dmap_{\ArgBpp}')
\end{align*}
where 
\begin{align*}
\dmap_{\ArgA}' &= \dmap_{\ArgA}(\elim_{\ArgA}(P, Q, \stepind{c}, \stepind{d}), \elim_{\ArgBpp}(P, Q, \stepind{c}, \stepind{d}), x) \\
\dmap_{\ArgBpp}' &= \dmap_{\ArgBpp}(\stepind{c}, \elim_{\ArgA}(P, Q, \stepind{c}, \stepind{d}), \elim_{\ArgBpp}(P, Q, \stepind{c}, \stepind{d}), x, y) \enspace .
\end{align*}

\end{definition}

 \begin{example}[The eliminator for sorted lists]
   Recall from Example \ref{ex:SList-functor} that sorted lists were given by the functors $\ArgSList$, $\ArglessList$, where
\[ %begin{align*}
  \ArgSList(A, B) = \one + (\Sigma\, n\!:\!\Nat .\,\Sigma\,\ell\!:\!A.\, B(n,\ell))
%  \ArglessList(A, B, c, m, \inl(\oneelt)) &= \one \\
%  \ArglessList(A, B, c, m, \inr(\langle n, \ell, p\rangle)) &= \Sigma\,m \leq n.\, B(m, \ell) \enspace .
\] %end{align*}
   Thus, we see that e.g.\ 
   \begin{align*}
\BOX_{\ArgSList}(P, Q, \inl(\oneelt)) &=  
\{ y : \one + \ldots \ |\ (\id + \ldots)(y) = \inl(\oneelt) \} \cong \one \\
\BOX_{\ArgSList}(P, Q, \inr(\langle n, \ell, p\rangle)) &\cong \\
\{ y : \Sigma\, n'\!:\!\Nat .\,\Sigma\,\langle \ell', \overline{\ell}\rangle\!:\!(\Sigma A P).\, \Sigma p'\!&:\!B(n,\ell).\,Q(n', \ell', p', \overline{\ell})\ |\ \Sigma(\id, \Sigma(\pi_0, \pi'_0))(y) = \langle n, \ell, p\rangle\} \\
%\cong \{ y : \Sigma\, n'\!:\!\Nat .\,\Sigma\,\langle \ell', \overline{\ell}\rangle\!:\!(\Sigma A P).\, \Sigma p'\!&:\!B(n,\ell).\,Q(n', \ell', p', \overline{\ell})\ |\ \langle n', \ell', p'\rangle = \langle n, \ell, p\rangle\} \\
 & \cong \Sigma \overline{\ell}\!:\!P(\ell).\,Q(n, \ell, p, \overline{\ell})
   \end{align*}
   and similarly for $\BOX_{\ArglessList}$, so that the eliminators are equivalent to
%this one is hard to look good formatted as a rule -- leaving it like this for now
   \begin{align*}
  \elim_{\SortedList} :\ & (P : \SortedList \to \Set) \to \\
                     & (Q : (n : \Nat) \to (\ell : \SortedList) \to n \lessList \ell \to P(\ell) \to \Set) \to \\
                     & (\stepind{\nilList} : P(\nilList)) \to \\
                     & (\stepind{\consListbare} : (n : \Nat) \to (\ell : \SortedList) \to (p : n \lessList \ell) \to (\stepindarg{\ell} : P(\ell)) \\
                     & \qquad \to Q(n, \ell, p, \stepindarg{\ell}) \to P(\consList{n}{\ell}{p})) \to \\
                     & (\stepind{\nilLess{}} : (n : \Nat) \to Q(n, \nilList, \nilLess{n}, \stepind{\nilList})) \to \\
                     & (\stepind{\consLessbare} : (m : \Nat) \to (n : \Nat) \to (\ell : \SortedList) \to (p : n \lessList \ell) \\
                     & \qquad \to (q : m \leq n) \to (p' : m \lessList \ell) \to
                     (\stepindarg{\ell} : P(\ell)) \\ 
                     & \qquad \to (\stepindarg{p} : Q(n, \ell,p, \stepindarg{\ell}))
                     \to (\stepindarg{p'} : Q(m, \ell, p', \stepindarg{\ell})) \\
                     & \qquad \to Q(m,\consList{n}{\ell}{p}, \consLess{m}{n}{\ell}{p}{q}{p'}, \stepind{\consListbare}(n, \ell, p, \stepindarg{\ell}, \stepindarg{p}))) \to \\
                     & (\ell : \SortedList) \to P(\ell) \enspace , \\
%
  \elim_{\lessList} :\ & \ldots \to \\
% & (P : \SortedList \to \Set) \to \\
                     % & (Q : (n : \Nat) \to (\ell : \SortedList) \to n \lessList \ell \to P(\ell) \to \Set) \to \\
                     % & (\stepind{\nilList} : P(\nilList)) \to \\
                     % & (\stepind{\consListbare} : (n : \Nat) \to (\ell : \SortedList) \to (p : n \lessList \ell) \to (\stepindarg{\ell} : P(\ell)) \\
                     % & \qquad \to Q(n, \ell, p, \stepindarg{\ell}) \to P(\consList{n}{\ell}{p})) \to \\
                     % & (\stepind{\nilLess{}} : (n : \Nat) \to Q(n, \nilList, \nilLess{n}, \stepind{\nilList})) \to \\
                     % & (\stepind{\consLessbare} : (m : \Nat) \to (n : \Nat) \to (\ell : \SortedList) \to (p : n \lessList \ell) \\
                     % & \qquad \to (q : m \leq n) \to (p' : m \lessList \ell) \to
                     % (\stepindarg{\ell} : P(\ell)) \\ 
                     % & \qquad \to (\stepindarg{p} : Q(n, \ell,p, \stepindarg{\ell}))
                     % \to (\stepindarg{p'} : Q(m, \ell, p', \stepindarg{\ell})) \\
                     % & \qquad \to Q(m,\consList{n}{\ell}{p}, \consLess{m}{n}{\ell}{p}{q}{p'}, \stepind{\consListbare}(n, \ell, p, \stepindarg{\ell}, \stepindarg{p}))) \to \\
                     & (n : \Nat) \to (\ell : \SortedList) \to (p : n \lessList \ell) \\
                     & \qquad \to Q(n, \ell, p, \elim_{\SortedList}(P, Q, \stepind{\nilList}, \stepind{\consListbare},  \stepind{\nilLess{}}, \stepind{\consLessbare}, \ell)) \enspace .
  \end{align*}
 \end{example}



\subsection{The Equivalence Between Having an Eliminator and Being Initial}
\label{sec:elim-init-equiv}

We now prove the promised equivalence. In what follows, let $\ArgB = (\ArgA,
\ArgBpp)$ be functors for an inductive-inductive definition.

\begin{lemma}
\label{thm:isoArgAB}
  There is an isomorphism
\begin{multline*}
\varphi_{\ArgB} = (\varphi_{\ArgA}, \varphi_{\ArgBpp}) : \ArgB(\SigmaDialg (A, B, c)\ (P, Q, \stepind{c})) \\
  \to \SigmaFAM \ArgB(A, B, c)\ (\BOX_{\ArgB}(P, Q, \stepind{c})) %Todo: \BOX_{\ArgB} not defined -- clear?
\end{multline*}
such that $(\pi_0, \pi'_0) \circ \varphi_{\ArgB} =
\ArgB(\pi_0, \pi'_0)$ and
\[
\varphi_{\ArgB} \circ\ArgB(\CwFsect{(f, g)})
 = \CwFsect{(\dmap_{ArgA}(P, Q, f, g), \dmap_{\ArgBpp}(P, Q, \overline{c}, f, g))} \enspace .
\tag*{\qed}
\]
\end{lemma}

\begin{proposition}
\label{thm:init-to-elim}
Every initial object $(A, B, c, d)$ in $\Eind{\ArgB}$ has an eliminator.
\end{proposition}
\begin{proof}
  Let $P$, $Q$, $\stepind{c}$, $\stepind{d}$ as in the type signature for $\elim_{\ArgA}$
  and $\elim_{\ArgB}$ be given. Define $\inn_{\Sigma} :
  \ArgB(\SigmaDialg (A, B, c)\ (P, Q, \stepind{c})) \to \UDialgArgAU(\SigmaDialg
  (A, B, c)\ (P, Q, \stepind{c}))$ by $\inn_{\Sigma} = \sigmamorph{(c, d)}{(\stepind{c},
    \stepind{d})} \circ \varphi_{\ArgB}$. This makes
  $\SigmaDialg (A, B, c)\ (P, Q, \stepind{c})$ an object of
  $\Eind{\ArgB}$.
  % since the first component of $\sigmamorph{(c, d)}{(\stepind{c}, \stepind{g})} \circ
  % (\varphi_{\ArgA}, \varphi_{\ArgB})$ is $\sigmamorph{c}{\stepind{c}} \circ
  % \varphi_{\ArgA}$, which coincides with the ``constructor'' in
  % $\SigmaDialg (A, B, c)\ (P, Q, \stepind{c})$.

  Since $(A, B, c, d)$ is initial in $\Eind{\ArgB}$, we get a morphism ($h, h') :
  (A, B) \to \SigmaFAM (A, B)\ (P, Q)$ which makes the top part of the
  following diagram commute:
\[
\resizebox{1\textwidth}{!}{%
\xymatrix{
\ArgB(A, B, c) \ar^-{(c, d)}[rr] \ar_-{\ArgB(h, h')}[d] & & (A, B) \ar^-{(h, h')}[d] \\
\ArgB(\Sigma (A, B, c)\ (P, Q, \stepind{c})) \ar_{\varphi_{\ArgB}}[r] \ar_-{\ArgB(\pi_0, \pi'_0)}[d]
 & \Sigma \ArgB(A, B, c)\ (\BOX(P, Q, \stepind{c}))  \ar^-{\sigmamorph{(c,d)}{(\stepind{c},\stepind{d})}}[r] \ar^{(\pi_0, \pi'_0)}[dl]
 & \Sigma (A, B)\ (P, Q) \ar^-{(\pi_0, \pi'_0)}[d]\\
\ArgB(A, B, c) \ar_-{(c, d)}[rr] & & (A, B)
}
}
\]

The bottom part commutes by Lemma \ref{thm:isoArgAB} and
calculation. Hence $(\pi_0, \pi'_0) \circ (h, h')$ is a morphism in
$\Eind{\ArgB}$ and we must have $(\pi_0, \pi'_0) \circ (h, h')
= \id$ by initiality. Thus $\pi_1 \circ h : (x : A) \to P(x)$ and
$\pi_1(h'(x, y)) : Q(x, y, \pi_1(h(x)))$ for $x : A$, $y : B(x)$, so
we can define $\elim_{\ArgA}(P, Q, \stepind{c}, \stepind{d}) = \pi_1 \circ h$ and
$\elim_{\ArgBpp}(P, Q, \stepind{c}, \stepind{d}, x, y) = \pi_1(h'(x, y))$. 

To verify the computation rules, note that since $(\pi_0, \pi'_0)
\circ (h, h') = \id$, we have $(h, h') = \CwFsect{(\pi_1, \pi'_1)
  \circ (h, h')}$. We only show the calculation for $\ArgA$:
\begin{align*}
  \elim_{\ArgA}(P, Q, \stepind{c}, \stepind{d}, c(x)))
 &= \pi_1(h(c(x))) \\
 &= \stepind{c}(\varphi_{\ArgA}(\ArgA(h, h')(x))) \\
 &= \stepind{c}(\varphi_{\ArgA}(\ArgA(\CwFsect{(\pi_1, \pi'_1) \circ (h, h')})(x))) \\
 &= \stepind{c}(x, \dmap_{\ArgA}((\pi_1, \pi'_1) \circ (h, h'))(x)) \\
 &= \stepind{c}(x, \dmap_{\ArgA}(\elim_{\ArgA}(P, Q, \stepind{c}, \stepind{d}), \elim_{\ArgBpp}(P, Q, \stepind{c}, \stepind{d}), x))
\end{align*}
\mbox{} \qed %todo
\end{proof}

\begin{proposition}
\label{thm:elim-to-weak-init}
  Every $(A, B, c, d)$ with an eliminator is weakly initial in $\Eind{\ArgB}$.
\end{proposition}
\begin{proof}
  Let $(A', B', c', d')$ be another object in
  $\Eind{\ArgB}$. Notice that for $P(x) = A'$, $Q(x, y,
  \stepindarg{x}) = B'(\stepindarg{x})$, the usually dependent second
  projections $\pi_1, \pi'_1$ become non-dependent and make up a
  morphism $(\pi_1, \pi'_1) : \SigmaFAM (A, B)\ (P, Q) \to (A', B')$. 
  Since
   \[
   \pi_1 \circ \sigmamorph{c}{c' \circ \ArgA(\pi_1, \pi'_1) \circ \varphi_{\ArgA}^{-1}}
         \circ \varphi_{\ArgA} = c' \circ \ArgA(\pi_1, \pi'_1) \enspace ,
   \]
   this lifts to $(\pi_1, \pi'_1) : \SigmaDialg (A, B, c)\ (P, Q, c'
   \circ \ArgA(\pi_1, \pi'_1) \circ \varphi_{\ArgA}^{-1}) \to (A', B',
   c')$. By currying $(f, g) \coloneqq (c', d') \circ \ArgB(\pi_1,
   \pi'_1) \circ \varphi_{\ArgB}^{-1}$, we get
   \begin{align*} 
     \stepind{c} \coloneqq \widehat{f} &: (x : \ArgA(A, B)) \to \BOX_{\ArgA}(P, Q, x) \to A' \\ 
     \stepind{d} \coloneqq \widehat{g} &: (x : \ArgA(A, B)) \to (y : \ArgBpp(A, B, c, x))
     \to
     (\stepindarg{x} : \BOX_{\ArgA}(P, Q, x))  \\
     & \qquad\qquad \to \BOX_{\ArgBpp}(P, Q, c, \widehat{f}, x, y,
     \stepindarg{x}) \to B'(\widehat{f}(x, \stepindarg{x}))
   \end{align*}
   so that $(h, h') \coloneqq (\elim_{\ArgA}(P, Q, \stepind{c}, \stepind{d}),
   \elim_{\ArgBpp}(P, Q, \stepind{c}, \stepind{d})) : (A, B) \to (A', B')$.

   We have to check that $(h, h') \circ (c, d) = (c', d') \circ \ArgB(h, h')$.
   \begin{align*}
     (h, h') \circ (c, d)
     &= (\elim_{\ArgA}(P, Q, \widehat{f}, \widehat{g}),
          \elim_{\ArgBpp}(P, Q, \widehat{f}, \widehat{g})) \circ (c, d) \\
     &= (\widehat{f}, \widehat{g}) \circ \CwFsect{(\dmap_{\ArgA}(h, h'), \dmap_{\ArgBpp}(h, h'))} \\
     &= (\widehat{f}, \widehat{g}) \circ \varphi_{\ArgB} \circ \ArgB(\CwFsect{h, h'}) \\
     &= (c', d') \circ \ArgB(\pi_1, \pi'_1) \circ \ArgB(\CwFsect{h, h'}) \\
     &= (c', d') \circ \ArgB(h, h')
   \end{align*}
\mbox{} \qed %todo
\end{proof}

For strictly positive functors, we can say more, since we can argue by
induction over their construction:

\label{sec:elim-init-equiv-CSL}
\begin{theorem}
\label{thm:equiv-init-elim}
The functors $\Arggamma = (\ArggammaA, \ArggammaB)$ from the original axiomatisation as
described in Section \ref{sec:functors-CSL-article} have eliminators
if and only if $\Eind{\Arggamma}$ has an initial object.
\end{theorem}
\begin{proof}
%  By Lemma \ref{thm:CSL-functor-CNF}, it is enough to consider
%  functors of the form mentioned there.
Putting Proposition
  \ref{thm:init-to-elim} and Proposition \ref{thm:elim-to-weak-init}
  together, all that is left to prove is that given an eliminator, the
  arrow $(h, h')$ we construct is actually unique. Assume that $(k,
  k')$ is another arrow with $(k, k') \circ (c, d) = (c', d') \circ
  \Arggamma(k, k')$.

  We use the eliminator (and extensional equality) to prove that $(h, h') =
  (k ,k')$; let $P(x) = (h(x) = k(x))$ and $Q(x, y, \stepindarg{x})
  = (h'(x, y) = k'(x, y))$. It is enough to prove $P(c(x))$ and
  $Q(c(x), d(x, y), \_)$ for arbitrary $x : \ArggammaA(A, B)$, $y :
  \ArggammaB(A, B, c, x)$, given the induction hypothesis $\BOX_{\ArgA}(P,
  Q)$ and $\BOX_{\ArgBpp}(P, Q)$.
%But by by Proposition \ref{thm:CSL-functor-box}, they state exactly that 
By induction on the buildup of $\ArggammaA$ and $\ArggammaB$, we can prove that 
$\BOX_{\ArgA}(P,  Q)$ and $\BOX_{\ArgBpp}(P, Q)$ give that
$\ArgB(h, h') =  \ArgB(k, k')$ , and hence
  \[
  (h, h') \circ (c, d) = (c', d') \circ \ArgB(h, h')
                       = (c', d') \circ \ArgB(k, k')
                       = (k, k') \circ (c, d) \enspace .
  \]
  We conclude that $(h, h') =  (k ,k')$. \qed
\end{proof}
\section{Conclusions and Future Work}

We have shown how to give a categorical semantics for
inductive-inductive definitions, a principle for defining datatypes in
Martin-L\"of Type Theory. In order to do this, we generalised the
usual initial algebra semantics to a dialgebra setting and showed that
there is still an equivalence between this semantics and the more
traditional formulation in terms of elimination and computation rules.

Future work includes extending the notion of
containers~\cite{abbottAltenkirchGhani2005containers} to
inductive-inductive definitions, and generalising this work to a
unified setting including other forms of inductive definitions: let
$F, G: \C \to \D$ be functors between categories that are Categories
with Families~\cite{dybjer1996internalTT,hofmann1997syntaxsemantics}.
If $\C$ and $\D$ have $\Sigma$- and equality types, one can define the
concept of an eliminator for $F$ and $G$ and with some restrictions on
$\C$ and $G$ (that $\C$ admits all finite limits and $G$ is left exact
seems to be enough), one can show that having an eliminator and being
initial in (a subcategory of) $\Dialg{F}{G}$ is equivalent.

\bibliographystyle{splncs03}
\bibliography{all}

% \appendix

% \section{$\BOX_{ArgA}$ and $\BOX_{\ArgB}$ for functors arising from the axiomatisation}

% %However, since we are working in extensional type theory, there is
% %actually a convenient ``container-like''
% %\cite{abbottAltenkirchGhani2005containers} description of these functors:
% \begin{lemma} %todo: consistent with sigma / \times? %todo: fix formatting
% \label{thm:CSL-functor-CNF}
% For every pair $(\ArgA, \ArgBpp)$ of functors arising from a code
% $(\gamma_A, \gamma_B)$, there are $S : \Set$, $K : S \to \Set$, $L :
% (x : S) \to K(x) \to \Set$ and $S' : \Set$, $K' : S' \to \Set$,
% $L'_{n} : (x : S') \to \ArgA^i(K'(x), \vec{L'(x)}_{(i)} \to \Set$,
% $a_{\text{index}} : (x : S') \to \bigplus_{i=0}^{n}\ArgA^{i}(K'(s),
% \vec{L'_{i}(s)}_{(i)})$ such that
% \begin{align*}
% \ArgA(A, B) &\cong (s : S) \times (k : K(s) \to A) \times ((x : K(s)) \to
% L(s, x) \to B (k(x))) \\
% \ArgA(f, g)(\langle s, k, l\rangle) &= \langle s, (f, g) \circ (k, l)\rangle
% \end{align*}
% \begin{align*}
%   \ArgBpp(A, B, c, y) &\cong \{ (s : S') \times (k : K'(s) \to A)
%   \times (l_0 : (x : K'(s)) \to L'_0(s, x) \to B(k(x))) \\
%  &\times (l_1 : (x : \ArgA(K'(s), L'_0(s))) \to  L'_1(s, x) \to B(k_1(x)))) \\
%   \times \ldots 
%  &\times (l_n : (x : \ArgA^n(K(s), \vec{L'(s)}_{(n)})) \to L'_{n}(s, x) \to
%   B(k_n(x)))) \\
%  &|\ c(y) = \ \coprodmorph{k, k_1, \ldots}{k_n}(a_{\text{index}}(s)) \} \\
%   \ArgBpp(f, g)(\langle s, k, l_0, &\ldots, \l_n\rangle) = \langle s, f \circ k, \lambda x y .\, g(k(x), l_0(x, y)), \\
%  & \qquad\qquad\qquad\qquad\ \  \lambda x y .\, g(k_1(x), l_1(x, y)), \ldots, \\
%  & \qquad\qquad\qquad\qquad\ \  \lambda x y .\, g(k_n(x), l_n(x, y))\rangle
% \end{align*} 
% where $k_i = c \circ \ArgA^i(k, \vec{l}_{(i)})$.
% \qed
% \end{lemma}


% %For functors arising from our earlier axiomatisation, there is a nicer description of $\BOX_{\ArgA}$ and $\BOX_{\ArgB}$:
% \begin{proposition}
% \label{thm:CSL-functor-box}
% For functors $(ArgA, \ArgBpp)$ as in Lemma \ref{thm:CSL-functor-CNF},
% we have
% \begin{multline*}
% \BOX_{\ArgA}(P, Q)(\langle s, k, l\rangle)  \cong \\ (\overline{k} : (x : K(s)) \to P(k(s)) \times ((x : K(s)) \to (y : L(s, x)) \to Q(k(x), l(x, y), \overline{k}(x)))
% \end{multline*}
% \begin{align*}
%   &\BOX{\ArgB}(P, Q, c, \overline{c}, y, \langle s, k, l_0, \ldots, l_n\rangle, \overline{y}) \cong
% \{ (\overline{k} : K'(s) \to P(k(s))) \\
% &  \times (\overline{l_0} : (x : K'(s)) \to (y : L'_0(s, x)) \to Q(k(x), l_0(x, y), \overline{k}(x))) \\
% & \times (\overline{l_1} : (x : \ArgA(K'(s), L'_0(s))) \to  (y : L'_1(s, x)) \to Q(k_1(x), l_1(x, y) ,\overline{k_1}(x))) \\
% &  \times \ldots 
%  \times (\overline{l_n} : (x : \ArgA^n(K'(s), \vec{L'(s)}_{(n)})) \to (y : L'_{n}(s, x)) \to
%   Q(k_n(x), l_n(x, y), \overline{k_n}(x)))) \\
% & |\ \overline{c}(y, \overline{y}) = \ \coprodmorph{\overline{k}, \overline{k_1}, \ldots}{\overline{k_n}}(a_{\text{index}}(s)) \}
% \end{align*}
% where $k_i = c \circ \ArgA^i(k, \vec{l}_{(i)})$ and 
% \[
% \overline{k_i} : (x : \ArgA^i(K'(s) ,, \vec{L'(s)}_{(i)})) \to P(k_i(x))
% \]
% is defined by
% \begin{align*}
% \overline{k_i}(x) = \overline{c}(\ArgA^i(k, \vec{l}_{(i)})(x),
% \ArgA(\CwFmorph{&(\coprodmorph{k, k_1, \ldots}{k_{i-1}},
%                  \coprodmorph{l_0, l_1, \ldots}{l_{i-1}})}
%                          {\\&(\coprodmorph{\overline{k}, \overline{k_1}, \ldots}{\overline{k_{i-1}}},
%                  \coprodmorph{\overline{l_0}, \overline{l_1}, \ldots}{\overline{l_{i-1}}})})(x)).
% \end{align*}
% \qed
% \end{proposition}

\end{document}
